\subsection{Test Results}

  The most generic test that provides nearly ultimate performance measure
 for an IP network, is the \texttt{ping} test. No other particular formal
 tests were yet performed during the work on this project, however some
 possible test scenarios had been considered. Firstly the graphs obtained
 from several ping test shall presented. Other basic test described below
 were performed with a scripts written in Ruby programming language, the
 source code of these scripts can found in the project repository under
 directory path \git{projects/wmi/native/tests/}.

\subsubsection{Latency Measurement}

  \dots

  These results are deliberately presented on separate graphs for clarity.
 Even if simultaneous ping tests were to be attempted, those may not be
 combined in one-to-one graphs, unless the size of the network had been
 greater then 2 nodes. Then it would be appropriate to consider overlaying
 the datasets based on latency intervals, instead of sequential timebase
 as presented here.

  \dots

  Idle latency measurements had also been taken. The stability of the idle
 latency leads to consideration of previous results as being a subject to
 event-driven design of the \emph{uIP} stack and \ContikiOS\ in general.
 It is reasonable to propose that non-idle ping response can be disregarded,
 since there is no practical purpose for it to be prioritised.

\subsubsection{Other Test Scenarios}

  It had been desired to set-up a test, where one MIDI device would send
 data to the same host via a wireless link and a wired MIDI interface
 simultaneously. However, wireless transceiver would need to provide packet
 timestamping, otherwise no precise measurement can be taken. This facility
 could be added to \MCX\ driver code, though it has not yet been studied
 how this would need to be integrated. Additionally, synchronisation of the
 timer on the node and the clock of the host system is necessary, which
 largely extends the complexity of the problem. Without synchronisation
 protocol there is no particular use for hardware packet timestamps and
 vise versa.

 %% CAN WE DO A BASIC LOOPBACK TEST WITH A RUBY SCRIPT ?
 
 %% provide URL where ruby scripts can be found on github.

\subsubsection{Throughput Observations}

  An attempt has been made to measure relative throughput of wireless MIDI
 link, it would be quite difficult to measure this precisely and more work
 is required to find out the most accurate way to calculate exact figures
 of bytes arriving from the MIDI port and bytes received on the other end
 of the wireless link per unit time. As said above, precision is an issue
 due to the lack of support for packet timestamping in the current code.
 First estimations of the performance was certainly a difficult task.
 The reason why this has been difficult is because the estimated figures
 did not match any expectations and substantially difficult to compare
 results between subsequent test takes.
  A graph in figure \ref{fig:tests:bad1} shows the worst case of system
 throughput. This measurements were take once before improved algorithm
 with additional polling callback was implemented. That has shown some
 improvement visually, though according to statics mechanism that had be
 added to the program in form of three counters (for sent, lost and not
 yet processed bytes) only some small chunks of MIDI data were processed
 and delivered.


  With the help of these test results a bug was found and eliminated.
 This problem appeared to be due to needless and misused timer in the
 application code. Nevertheless, at first this was misunderstood as a
 limitation in the \emph{uIP} stack or the capabilities of the device
 under test. Further test has proven that performance of the system is
 adequate. Figures \cite{fig:tests:good1,fig:tests:good2} demonstrate
 this, though the time base is rather relative it serves the purpose
 for this test.
 
 

 % The current implementation can be viewed in the repository, it has
 % omitted from the program listing \ref{app:code:talker} for clarity.



\subsubsection{Conclusion}

 Further work is require to achieve suitable results with \Contiki\ and
 \emph{uIP}, in-depth study of the system core implementation is necessary

