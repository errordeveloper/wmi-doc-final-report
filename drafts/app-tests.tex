\subsection{Test Results}

  The most generic test that provides nearly ultimate performance measure
 for an IP network, is the \texttt{ping} test. No other particular formal
 tests were yet performed during the work on this project, however some
 possible test scenarios had been considered. Firstly the graphs obtained
 from several ping test shall presented.

\subsubsection{Latency Measurement}

  \dots

  These results are deliberately presented on separate graphs for clarity.
 Even if simultaneous ping tests were to be attempted, those may not be
 combined in one-to-one graphs, unless the size of the network had been
 greater then 2 nodes. Then it would be appropriate to consider overlaying
 the datasets based on latency intervals, instead of sequential timebase
 as presented here.

  \dots

\subsubsection{Other Test Scenarios}

  It had been desired to set-up such test, where a MIDI device would send
 data to a wireless link and to a wired with MIDI interface simultaneously,
 however wireless transceivers would need to provide packet timestamping.
 Otherwise no precise measurement can be taken. This facility can be added
 to \MCX\ driver code, though it has not yet been studied how this would
 need to be implemented. It would need to synchronise the timer with host
 clock, which adds more complexity to this problem. Without synchronisation
 protocol there no particular use for hardware packet timestamps.

 %% CAN WE DO A BASIC LOOPBACK TEST WITH A RUBY SCRIPT ?
 
 %% provide URL where ruby scripts can be found on github.

\subsubsection{Throughput Observations}

  An attempt has been made to measure the throughput of wireless MIDI link,
 nevertheless this appeared quite difficult to measure. Certainly more work
 is required to find out the most accurate way to calculate the statistics
 of bytes arriving from the MIDI port and bytes received on the other end
 of the wireless link. The reason why this has become difficult is because
 the estimate figures are worse then expected and do not match from one
 test to another. Use of dedicated hardware would be desired here, which
 can be done by utilising an additional microcontroller, though that could
 not be implemented by the deadline of this report.
  A graph in figure \ref{fig:tests:stat1} shows the worst case of system
 throughput. This measurements were take once before improved algorithm
 with additional polling callback was implemented. After this had been
 added to the program, the code which statistics were relying upon has
 evolved to more complex form and new statistics methods are needed now.
 The current implementation can be viewed in the repository, it has omitted
 from the program listing \ref{app:code:talker} for clarity.
 First attempt to implement those methods has not demonstrated the same
 level of improvements which has been observed, clearly there a few bugs
 due to race conditions between the counter variables. Most important is
 the fact that visual improvement had been made with the polling callback.
 However, It should be noted that this problem is not due to statics code.
 What appears to be the bottleneck of throughput is the scheduling of packet
 transmission. Further work is require to achieve suitable results with
 \Contiki\ and \emph{uIP}, in-depth study of the system core implementation
 is necessary to resolve this issue.

