
%% in introduction part:

  The major emphasis of this project had been in the area of development
 tools, platforms and code architecture for embedded software design.
 Though originally it was intended as application development project,
 study of various alternatives of how the task can be achieved lead to
 extended research in embedded operating systems and various programming
 tools (namely compiler toolchains, cross-platform integration techniques
 as well as languages).
  One of initial interest areas had been in the field of networking protocols
 for wireless sensor networks. Current best practice suggests that IP-enabled
 networking technology is most suitable for its global adoption, hence simple
 to integrate with various existing applications. It is rather difficult to
 imagine where in the modern technology world a non-IP network may be useful.
 The closes reference to this can be found in papers published by IPSO 
 \cite{IPSO_PAPERS} and general industry trends. Certainly non-IP networking
 technology is available on the market, however if such device are going to
 used for any application, there are several limitations imposed on integration
 into various existing infrastructures. That implies physical connectivity
 as well as software support. One example is the popular \emph{ZigBee}
 family of protocols, is apparently gaining popularity, thought no global
 scale of deployment has been observed. Instead most of research papers
 are oriented towards the use of IP-enabled networks for wireless sensing.

  As stated in the interim report, the task which was still to be
 completed then had been the development of radio transceiver driver
 code for the chosen platform (\Chip{ATmega128RFA1}). After spending
 several hours of work, it has been understood that there are various
 in portability of \emph{AVR} architecture code. The \emph{AVR} branch
 of \emph{Contiki OS} (in its current state) has almost no facility to
 aid porting task for various \emph{AVR} devices which may be of some
 interest. Comparing currently available \emph{Atmel AVR} microcontroller
 models \cite{atmel:avr:table}, it is rather clear that OS support for
 this architecture can be generalised to much higher degree then what is
 currently provided by the \emph{Contiki OS}. Due to the time limit for
 this project, various alternatives were considered.

\subsubsection{TinyOS}

  It has been discovered that a few members of \emph{TinyOS} community
 have recently worked on porting \emph{TinyOS} to run on \Chip{ATmega128RFA1}
 \cite{tinyos:arch:rfa1-p1,tinyos:arch:rfa1-p2}. Comparing to \emph{Contiki},
 \emph{TinyOS} has rather superior abstraction layer that provides a seamless 
 cross-platform integration \cite{tinyos:tepXXX,tinyos:tepYYY,tinyos:tepZZZ}.
 
  Various internal of \emph{TinyOS} had been studied, however it is
 certainly a very broad area to be described here. It is best described
 in the "TinyOS Programming" book by David Gay and Philip Levis (it is
 available in print as well as web-edition \cite{tinyos:book}). Two
 authors of this book are lead developer of the \emph{TinyOS} project
 and are currently working at the University of California, Berkley
 and Standford.

  The greatest achievement of \emph{TinyOS} is its specialised language,
 the benefits of which had been overlooked at the earlier stage of research
 for this project. Network 

  \emph{TinyOS} has become widely adopted and there are many researcher
 who contributed significant work in different application areas, one
 such area of great interest to WMI project is sensor device time
 synchronisation protocols. A number of papers are available and the
 mainstream repository of \emph{TinyOS} source code already contains
 implementations for some of those protocols.
 ( A few details with references needed )
  Further study and experimentation in this area are required to design
 a system which could cope with real-time constraints of stage control
 applications. It is important to note that \emph{P802.15.4} has addressed
 some real-time application requirements, however the status of software
 support for these features of \emph{P802.15.4} is currently uncertain.
 ( Perhaps mention NanoRX ? )

 

