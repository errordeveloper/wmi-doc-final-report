\section{Initial Hardware Consideration} \label{sec:RFA}

  At the initial stage of this project the hardware platform was chosen
 based on previous experience of using development tools for \emph{Atmel
 AVR} microcontrollers. The original orientation was towards using a chip
 with integrated radio transceiver, \emph{Atmel} advertises the \RFA\ chip
 as having the lowest power consumption and most appropriate link budget.

\subsubsection{Development Boards for ATmega128RFA1}

  As it was already mentioned, two choises of development boards were considered:

	\begin{itemize} \em
		\item Dresden Elektronik \cite{links:de:rcb,links:de:stb}
		\item SparkFun (Figure \ref{fig:sparkfun:atmega128rfa1:image})
	\end{itemize}

  \emph{DE} hardware features a useful set of components, including a spear 32 kB
 of memory and robust screew terminals for I/O connections. However, the board
 from \emph{SparkFun} has been chosen for it's low price.
  It has been found later that this board was inconvenient to use in various
 ways, for example the layout of serial port pins on the side of the board
 could be design to fit standard serial USB cables\footnote{\emph{SparkFun
 sells few different serial USB adaptors all of which have the same layout.
 These are know to be very popular among most of hobbists and professional
 engineers and therefore are considered to be a de-facto standard.}}.
 Also the layout of transciever side of the PCB was found to be quite primitive
 and, most importantly, has not included a suitable ground plane nor it has a
 shield. Nevertheless, this board is of a rather small footprint and could be
 used in a prototype product.

%% inlude: CIRCUIT & BOARD LAYOUTS
%% copy the section of interim report
%% where some stuff is said regarding
%% the circuit ... or may be not ???
