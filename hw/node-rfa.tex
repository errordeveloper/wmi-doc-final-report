\pagebreak
\section{Initial Hardware Consideration} \label{sec:RFA}

  At the initial stage of this project, the hardware platform was chosen
 based on previous experience of using development tools for \emph{Atmel
 AVR} microcontrollers. The original orientation was towards using a chip
 with integrated radio transceiver, \emph{Atmel} advertises the \RFA\ chip
 as having the lowest power consumption and most appropriate link budget.

\subsubsection{Development Boards for ATmega128RFA1}

\begin{figure}
\centering
\includegraphics[angle=90,scale=2]{../figures/SparkFun_ATmega128RFA1_board_top.pdf}
\includegraphics[angle=90,scale=2]{../figures/SparkFun_ATmega128RFA1_board_bot.pdf}
\caption{\emph{SparkFun development board layout (top and bottom)}} \label{fig:sparkfun:atmega128rfa1:layout}
\end{figure}

  As it was already mentioned, two choises of development boards were considered:

	\begin{itemize} \em
		\item Dresden Elektronik \cite{links:de:rcb,links:de:stb}
		\item SparkFun (Figure \ref{fig:sparkfun:atmega128rfa1:image})
	\end{itemize}

  \emph{DE} hardware features a useful set of components, including a spare 32 kB
 of memory and robust screew terminals for I/O connections. However, the board
 from \emph{SparkFun} has been chosen for it's low price and the GPIO pins of all
 ports are made available on the PCB edges.

  It has been found later that this board was inconvenient to use in various
 ways, for example the layout of serial port pins on the side of the board
 could had been designed to fit standard serial USB cables\footnote{\emph{%
 SparkFun sells few different serial USB adaptors all of which have the same
 layout. These are know to be very popular among most of hobbists and professional
 engineers and therefore are considered to be a de-facto standard.}}. As it
 can be seen in figure \ref{fig:sparkfun:atmega128rfa1:layout}, the layout of
 transciever side of the PCB was found to be quite primitive and, most importantly,
 has not included a suitable ground plane nor it has a shield. Nevertheless,
 this board is of a rather small footprint and could be used in a prototype product.

  Certainly some rework can be done to improve the layout, since \emph{SparkFun}
 published this design under the \emph{Creative Commons License}, which allows
 freedom of sharing and adoption of the circuit, provided that the original
 copyright notice is attributed \cite{links:license:cc:by:sa:3:0, links:wiki:cclic}.

\pagebreak

\begin{figure}
\includegraphics[scale=0.8]{../figures/SparkFun_ATmega128RFA1_circuit_monochrome.pdf}
\caption{\emph{SparkFun development board schematics}} \label{fig:sparkfun:atmega128rfa1:circuit}
\end{figure}
