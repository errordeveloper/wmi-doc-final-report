\section{DSP Host Hardware}

  A set of hardware platforms suitable for the DSP host were
  considered. Among those are the following:
 	\begin{itemize}
		\item \emph{Analog Devices SHARC}
		\item \emph{ARM:} \begin{itemize}
		\item \emph{Marvell Sheeva}
		\item \emph{Texas Instruments OMAP}
		\item \emph{Freescale i.MX}
		\end{itemize}
	\end{itemize}
 All of these processor architectures are fully supported and widely
 used in embedded systems. Various other architecture families were
 considered, but found rather inappropriate due to the price range
 dictated by the target of this project. Multi-core \emph{MIPS64}
 devices by \emph{NetLogic} \cite{links:netlogic:mips64}, for
 example, have great computational capacity, thought these are too
 expensive for this particular application.

\subsection{x86-based Embedded Devices}

  The list above does not include the \emph{x86}-based CPUs due to a
 fact that it had been very difficult to find a target board using
 Intel or any \emph{x86} CPU from other vendors. There are many boards
 from a large variety of manufacturers, hence the selection process
 becomes particularly time consuming. Quality evaluation is also a
 necessity\footnote{\emph{Due to the scale of production in this
 market, there is a great chance to obtain a device with a failure.}},
 when looking at \emph{x86} devices. Although, there are boards that
 would match the low-power and small form-factor requirements, most
 of these are with a handful of peripherals, for example low quality
 audio and video interfaces and, if these are not present, the board
 may have two or more RS-232 connectors. Another aspect of \emph{x86}
 system design is that it comes with legacy \emph{BIOS} technology,
 while most of \emph{ARM} systems, for instance, have rather more
 flexible facilities for bootloader and set-up. Although, in the class of
 \emph{single board computers (SBC)}, there are many boards which do
 not include the physical connectors for above mentioned peripherals,
 however such boards are designed for industrial use and require a
 specialised chassis \cite{links:linuxfordevices:guide}.

  Nevertheless, an \emph{x86} machine has been used for most of the
 development work on this project, which is for a rather transparent
 reason.


\subsection{OMAP}

  The \emph{OMAP} application processors form \emph{TI} has become
 popular in the open-source community, and in fact \emph{TI} promotes
 Linux and Android as most suitable operating systems for this platform.
 The \emph{OMAP} architecture is based around an \emph{ARM} processor
 (there are various models with different versions of \emph{ARM} core)
 and \emph{TI C64x} DSP block. However, there is a little problem
 associated with how the DSP unit is integrated within the SoC.
 Very limited documentation is provided on how it can be used in the
 \emph{Linux} environment \cite{ti:omap:wiki:dsp}.

 One most outstanding development platform that uses an \emph{OMAP}
 chips with a dual-core \emph{Cortex-A8} CPU clocked at 1GHz - is the
 \emph{PandaBoard} \cite{ti:omap:wiki:pb}. However, it is available
 for back-order only, the maker is producing these boards on demand
 and the lead time would be at least one month. Otherwise this board
 would have been purchased despite the fact that DSP unit would be
 difficult to utilise.

\subsection{Other ARM CPUs}

  \emph{ARM} CPUs are produced by almost any major semiconductor firm,
 including \emph{Freescale} and \emph{Marvell}. Not all fabricators
 make very high-performance \emph{ARM} chips, wich are clocked near to
 1 GHz, and just a few make multi-core SoCs. And only some chips have an
 FPU. \emph{ARM} floating-point instruction set is know as \emph{VFP}
 ("Vector Floating-Point"). There are different versions of \emph{VFP},
 though in this context that wasn't a concern \cite{links:arm:vfp}.
 It is needless to mention that both vendors fully support Linux OS.

\subsubsection{Marvell Sheeva}

  \emph{Marvell} has originally started offering \emph{ARM} processors
 since their purchase of Intel's \emph{PXA} devision. Now \emph{Marvell}
 produces a series of high-performance \emph{ARM} chips, most of which
 are multi-core. \emph{Sheeva} is the brand name for these SoC devices.
 The clock frequency ranges from 800 MHz to 1.6 GHz, gigabit Ethernet
 is one of the outstanding features, since \emph{Marvell} specialises
 in the networking and storage IC market. As mentioned above, most
 important feature of \emph{ARM} chips that is necessary for the design
 of DSP host for this project is \emph{VFP}. Some of \emph{ARMADA}
 SoCs include \emph{VFP}, namely \emph{ARMADA XP} series,
 \emph{ARMADA 510} and \emph{610} \cite{links:marvell:armada}.
 \emph{Kirkwood} series \cite{links:marvell:kirkwood} are not featuring
 a \emph{VFP}, thought there are some outstanding embedded development
 platforms available.

  \emph{Plug Computer} (also know as \emph{Sheeva Plug}) is small, yet
 very powerful computer in a form-factor of wall socket DC adaptor
 \cite{links:marvell:plug,links:plugcomp:homepage}. There many exciting
 application where these devices could be the best fit, thought due to
 above mentioned lack of \emph{VFP}, this CPU is not well suited as the
 DSP host.


\subsubsection{Freescale i.MX}

 \emph{Freescale Semiconductors} offers a range of \emph{ARM} processors
 branded \emph{i.MX} \cite{links:freescale:imx}. The datasheets had been
 examined to find out which chips include a floating-point unit. Among the
 mid-range, \Chip{i.MX31} devices (532 MHz ARM1136JF-S) appear to have
 support for vector floating-point instructions, however the clock rate of
 the \Chip{i.MX31} SoC would suite only a limited number of applications
 \cite{links:freescale:imx31}. In the upper-range of \emph{Freescale i.MX}
 solutions, there are single-core \emph{Conrtex-A8} devices with clock rates
 up to 1 GHz \cite{links:freescale:imx5}, namely the \Chip{i.MX535}  and
 \Chip{i.MX538}. The second SoC has better video coding and storage
 capabilities.

 It has been noted that a simple development board for \Chip{i.MX535} can
 be purchased for a relatively low price\footnote{\emph{There two basic
 options: one for 150 US dollars, and another for 200 with a touch-screen.}}.


\subsection{SHARC}

  \emph{Analog Device} is a well-known manufacturer of DSP chips, the
 \emph{SHARC} architecture is different from \emph{TI OMAP} and the
 signal processing instructions are handled directly on core\footnote{%
 \emph{As it was mentioned above, TI OMAP is a combination of ARM and
 C64x DSP, while SHARC is a monolithic SoC. Nevertheless, some designer
 chose to use SHARC with an ARM control CPU, e.g. the DMA1.}}.
 The clock frequency is not being the main performance factor of the
 \emph{SHARC} devices and range between 300 and 600 MHz, instead FLOPS
 (Floating-Point Operations per Second) are used to define the speed of
 these processors \cite{links:adi:sharc}. A \emph{SHARC} chip suitable
 for audio typically is rated at 2700 MFLOPS.
 Linux has been a popular platform for advanced \emph{SHARC}-based systems
 appearing in various application markets. % It should noted that the
%advertisment of \emph{SHARC} chips provides less details then products
%from ARM fabricators, hence an extensive evalution would be required for
%designing a system with \emph{SHARC}.


  \emph{SRARC} processors are commonly used in professional audio
 equipment. One very outstanding product has been released by
 a new British company \emph{Dark Matter Audio} this year, it is
 called \emph{DMA1} \cite{links:dma1}.
 It is a portable  system for interactive music performance with
 very novel design features and powerful \emph{SHARC} DSP engine.
 According to the product announcement it runs Linux and also
 uses an \emph{ARM} chip for graphical user interface. It would be
 possible to integrate \emph{DMA1} with current prototype system
 developed in the course of this project. The vendor has provided
 an SDK, thought a request has been sent to obtain more specific
 details needed to design this add-on solution appropriately.


  \emph{SHARC} development hardware is mostly available from the
 silicon vendor itself and very few third-party boards could be found.
 Since this architecture is more specialised, the prices of the 
 development boards are not as low as some of the \emph{ARM} boards,
 although the vendor offers a comprehensive free library of board
 design files \cite{links:adi:freepcb}.


