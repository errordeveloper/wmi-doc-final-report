\section{DSP Host Hardware}

  A set of hardware platforms suitable for DSP host were considered.
 Among those are the following:
 	\begin{itemize}
		\item \emph{Analog Devices SHARC}
		\item \emph{ARM:} \begin{itemize}
		\item \emph{Marvell Sheeva}
		\item \emph{Texas Instruments OMAP}
		\item \emph{Freescale i.MX}
		\end{itemize}
	\end{itemize}
 All of these processor architectures are fully supported and widely
 used in embedded systems. Various other architecture families were
 considered, but found rather inappropriate due to the price range
 dictated by the target of this project. Multi-core \emph{MISP64}
 devices by \emph{NetLogic} \cite{netlogic:mips64:multicore}, for
 example, have great computational capacity, thought these are too
 expensive for this particular application.

\subsection{x86-based Embedded Devices}

  The list above does not include \emph{x86}-based CPUs due to the
 fact that it had been very difficult to find a target board using
 Intel or any \emph{x86} CPU from other vendors. There are many boards
 from a large variety of manufacturers, hence the selection process
 becomes particularly time consuming. Quality evaluation would also
 be necessary\footnote{\emph{Due to the scale of production in this
 market, there is a great chance to obtain a device with a failure.}},
 when looking at \emph{x86} devices. Although, there are boards that
 would match the low-power and small form-factor requirements, most
 of these are with a handful of peripherals, for example low quality
 audio and video interfaces and, if these are not present, the board
 may have two or more RS-232 connectors. Another aspect of \emph{x86}
 system design is that it comes with legacy \emph{BIOS} technology,
 while most of \emph{ARM} systems, for example, have rather more
 flexible facilities for booloader and set-up. Also in the class of
 \emph{single board computers (SBC)} there are many boards which do
 not inlude physical connectors for above mentioned peripherals,
 however isuch boards are designed for industrial use require a
 specialised chasis \cite{links:linuxfordevices:guide}.

  Nevertheless, an \emph{x86} machine has been used for most of the
 development work on this project, that is for a rather transparent
 reason.

%% add to bibtex file:
% http://www.linuxfordevices.com/c/a/Linux-For-Devices-Articles/Single-Board-Computer-SBC-Quick-Reference-Guide/

\subsection{OMAP}

  The \emph{OMAP} application processors form \emph{TI} has become
 popular in the open-source community, and in fact \emph{TI} promotes
 Linux and Android as most suitable operating systems for this platform.
 The \emph{OMAP} architecture is based around an \emph{ARM} processor
 (there are various models with different versions of \emph{ARM} core)
 and \emph{TI C64x} DSP block. However, there is a little problem
 associated with how the DSP unit is integrated with the CPU.
 Very limited documentation is provided on how it can be used in the
 \emph{Linux} environment \cite{ti:omap:wiki:dsp}.

 One most outstanding development platform that uses an \emph{OMAP}
 chips with dual-core \emph{Cortex-A8} CPU clocked at 1GHz, is the
 \emph{PandaBoard} \cite{ti:omap:wiki:pb}, however it is available
 for back-order only, the maker is producing these boards on demand
 and the lead time would be at least one month. Otherwise this board
 would have been purchased despite the fact that DSP unit would be
 difficult to utilised.

 %http://www.omappedia.org/wiki/DSPBridge_Project
 %http://www.omappedia.org/wiki/PandaBoard

\subsection{Other ARM CPUs}

  \emph{ARM} CPUs are produced by almost any major semiconductor firm,
 so do \emph{Freescale} and \emph{Marvell}. Not all of those companies
 make very high performance \emph{ARM} chips, i.e. clocked near to 1GHz,
 and few make multi-core CPUs. And only some CPUs have an FPU.
 \emph{ARM} floating point instruction set is know as \emph{VFP}
 ("Vector Floating Point"). There are different versions of \emph{VFP},
 though in this context this wasn't a concern.

\subsubsection{Marvell Sheeva}

  \emph{Marvell} has originally started offering \emph{ARM} processors
 since their purchase of Intel's \emph{PXA} devision. Now \emph{Marvell}
 produces a series of high-performance \emph{ARM} chips, most of which
 are multi-core. \emph{Sheeva} is the brand name for these SoC devices.
 The clock frequency ranges from 800 MHz to 1.6 GHz, gigabit Ethernet
 is one of the outstanding features, since \emph{Marvell} specialises
 in the networking and storage IC market. As mentioned above, most
 important feature of \emph{ARM} chips that is necessary for the design
 of DSP host for this project is \emph{VFP}. Some of \emph{ARMADA}
 SoCs include \emph{VFP}, namely \emph{ARMADA XP} series,
 \emph{ARMADA 510} and \emph{610} \cite{links:marvell:armada}.
 \emph{Kirkwood} series are not featuring \emph{VFP}, thought there
 are some outstanding embedded development platforms available.

  \emph{Plug Commputer} (also know as \emph{Seeva Plug}) is small, yet
 very powerful computer in a form-factor of wall socket DC adaptor
 \cite{links:marvell:plug,links:plugcomp:homepage}. There many exciting
 application where these devices could be the best fit, thought due to
 above mentioned lack of \emph{VFP}, this CPU is not well suited as the
 DSP host.

% Marvell Semiconductor, Inc. - ARMAD Processor Family
% http://www.marvell.com/products/processors/armada.html
% http://www.marvell.com/platforms/plug_computer/
% http://www.plugcomputer.org/

\subsubsection{Freescale i.MX}
% Freescale i.MX family info:
% http://cache.freescale.com/files/32bit/doc/brochure/FLYRIMXPRDCMPR.pdf
%^% NO DETAILS ABOUT NEON/VFP specified !!

% i.MX31 - mid range (532 MHz ARM1136JF-S)
% i.MX31 has VFP:
% http://www.freescale.com/webapp/sps/site/prod_summary.jsp?code=i.MX31
% http://www.freescale.com/files/32bit/doc/ref_manual/MCIMX31RM.pdf
%^% NOT MUCH APART FROM THE VFP.

% i.MX53 - top end of the family (1 GHz ARM Cortex-A8)
% i.MX535 and i.MX538 Applications Processors Fact Sheet 
% http://www.freescale.com/webapp/sps/site/overview.jsp?code=IMX53_FAMILY
% http://cache.freescale.com/files/32bit/doc/fact_sheet/IMX5CNFS.pdf
%^% IT SAYS THAT A BASIC DEV BOARD IS JUST $150, WITH TOUCHSCREEN - $200
% i.MX538 has NEON and VFP + SATA:
% http://www.freescale.com/webapp/sps/site/prod_summary.jsp?code=i.MX538
% http://www.freescale.com/files/32bit/doc/ref_manual/iMX53RM.pdf

\subsection{SHARC}

  The DSP instruction can be used directly and a library is provided
 which \dots blah \dots blah.

300MHz - 600MHz .. there dual-core Blackfin chips ..


  \emph{Analog Device SHARC} development hardware is mostly available
 from the vendor and very few third-party board could be found. Since
 this hardware architecture is more specialised, the prices of the 
 development boards are not as low as some of the \emph{ARM} boards.
  Although the vendor offers a comprehensive free library of board
 design file \cite{links:adi:freepcb}.

% http://blackfin.uclinux.org/gf/project/stamp/frs

  \emph{SRARC} processor are commonly used in professional audio
 equipment. One very outstanding product has been released by
 a new British company \emph{Dark Matter Audio}, the device called
 \emph{DMA1} \cite{links:dma1} has been released this year.
 It is an advanced system for interactive music performance with
 very novel design features and powerful \emph{SHARC} DSP engine.
 According to the product announcement it runs Linux OS and also
 uses an \emph{ARM} chip for graphical user interface. It would be
 possible to integrate \emph{DMA1} with current prototype system
 developed in the course of this project. The vendor has provided
 an SDK, thought a request has been sent to obtain more specific
 details needed to design this add-on solution appropriately.
 
