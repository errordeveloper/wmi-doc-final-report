\subsection{Building Embedded Linux OS}

\subsubsection{Background}

  Until quite recently, it had been rather more difficult to achieve
 the task of building embedded (custom) Linux system. Traditionally,
 the engineer who desired to do so, would need to follow instruction
 provided in the reference book \emph{"Linux from Scratch"}, commonly
 known as \emph{LFS} \cite{book:lfs}. This text provides details on
 how to utilise various tools for building an embedded Linux kernel
 and the file system from source, it discusses how to tweak various
 features at build time and configure appropriate runtime services.

\subsubsection{Build Utilities}
 
  More recently an number of project emerged, which do a
 great job of extending the flexibility by automating some of the
 simpler task, e.g. package dependency tracking and build version
 control (enabling the maintainer to revert to previous builds).
 One of the most outstanding and widely used projects in the areas
 is \emph{OpenEmbedded} \cite{links:oe} it has recently been adopted
 by a major software company \emph{Mentor Graphics}. \emph{Mentor
 Embedded Linux} extends \emph{OpenEmbedded} framework with a set
 of tools which are useful in a large-scale development project.
 
  There a few alternative approaches which lead to similar results,
 one may wish to use \emph{Gentoo Linux} meta-distribution
 \cite{links:gentoo:embedded}.
 Generally, there is no big difference between \emph{Gentoo} and
 \emph{OpenEmbedded}. Many internals are know to be very similar,
 tough \emph{Gentoo} doesn't provide some specialised tools which 
 are provided in \emph{OpenEmbedded}, hence \emph{Gentoo} has been
 originally designed for custom desktop and server systems.
 The conclusion is that \emph{OpenEmbedded} is most like to be
 more convenient to use for an embedded target.
 
  A little different approach can be taken with \emph{BuildRoot}
 \cite{links:buildroot:homepage}, which is a build system for
 embedded Linux based around the same framework which was design
 for configuring the Linux kernel builds. This tool is a set of
 scripts controlled via a console menu. Although, it appears to
 have a simple to use front-end, the underlying configuration
 system is certainly behind the competition with \emph{OpenEmbedded}.

  A project which have a slightly different orientation is
 \emph{ScratchBox} \cite{links:sbox:homepage}. It provide a
 cross-compilation toolkit for an application developers.
 However it is rather limited in various ways, i.e it a
 distribution rather then a build system.
 
  Another project in this area of interest is \emph{Linaro Foundation}
 \cite{links:linaro:about}, the initiative has been originated by
 London-based \emph{Canonical Limited} (the company behind now most
 popular Linux distribution - \emph{Ubuntu}).
 The purpose of \emph{Linaro} is to improve various issues related to
 embedded Linux and provide a higher grade platform for Android and
 other multimedia and consumer oriented distribution. \emph{Linaro} is
 still an emerging project and has not produced any significant output
 \cite{links:linaro:homepage}. It has specific orientation in terms of
 hardware, being currently involved only with \emph{ARM} platform and
 in terms software it is bound to \emph{Ubuntu} methodology.

\subsection{Conclusion}

  It had been desired to build an embedded DSP host OS as one of
 additional project targets, however due to various limitations, most
 importantly - the time frame, the project has not achieved this at
 the time of writing of the final report. The above section had included
 to provide the evidence of research in this area.

% Mentor Embedded Linux - Technical Brief
% by Mentor Graphics Corporation
% http://www.mentor.com/embedded-software/upload/embedded-linux-brief.pdf

% OpenEmbedded - User Manual
% http://docs.openembedded.org/usermanual/usermanual.pdf

% BuildRoot - Home Page
% http://buildroot.uclibc.org/

% Linaro - Foundation Home Page
% http://www.linaro.org/about-linaro/

% ScratchBox - Project Home Page
% http://www.scratchbox.org/

% Gentoo Linux - Embedded Handbook
% http://www.gentoo.org/proj/en/base/embedded/handbook/
