

\section{Overview of Hardware Options}

  Regarding the choice of development platform, there had been extended research
  carried out and final decision was taken due to a number of factors. One of
  which was familiarity with development tools for \emph{8-bit AVR}. \emph{GCC}
  cross-compiler and \emph{AVR LIBC} were used in some earlier work with
  \emph{Arduino}. Also the porting of driver code was considered as challenging
  opportunity. More details will be given in further sections of this report,
  for the purpose of this section some alternatives are discussed.  Another
  factor narrowing the selection, was integrated MCU core and RF transceiver
  in a single chip.

\subsection{Platforms and Cores}

 Currently semiconductor market presents a wide variety
 of single-chip MCUs with \WPAN\ interfaces. There is a
 dedicated section on the WMI website for more information
 and references to product pages \cite{wmi:wiki:chips1, wmi:wiki:chips2}.

\subsubsection{32-bit ARM Cortex}

 Two different integrated \WPAN\ microcontrollers based
 on \emph{ARM Cortex} architecture were found. Most notable
 chips are \emph{Freescale \texttt{MC1322x}}
 \emph{platform-in-package} family, which combine all transceiver
 components in one chip, providing a very simple design solution.
 Only one external RF component that \emph{\texttt{MC1322x}}
 requires is the antenna. Alternative \emph{Cortex-M3} product is
 \emph{Ember \texttt{EM300}-series}, these don't achieve the same
 scale of integration on the transceiver's side and still requires
 an external RF circuit.

 The \Chip{MC1322x} gives a great benefit to a designer, as it
 is known that layout of a radio circuit for frequencies in
 gigahertz range is particularly challenging task and indeed
 this is a specialist area of board design.

\subsubsection{Other Microcontrollers}

 \emph{ST Microelectronics} offers a 16-bit single-chip \WPAN\
 microcontrollers \Chip{SN250} and \Chip{SN260}, both are based
 on \emph{XAP2b} processor architecture. \emph{Ember} has \Chip{EM260}
 \emph{network co-processor} and \Chip{EM250} \emph{system-on-chip}
 which also use 16-bit \emph{XAP2} cores.

 \emph{NEC} (now \emph{Renesas}) has a few 16-bit RF controller chips
 available but the selection of development tools for that platform is
 rather poor.

 \emph{Freescale} has a range of 8-bit \WPAN\ microcontrollers
 \Chip{MC13213}, there are currently three devices with 16k/32k/60k
 of flash memory. These processors belong to well known \emph{68HC08}
 architecture family.

 A UK company \emph{Jennic} (now acquired by \emph{NXP}) produces a
 range of wireless micorocontrollers and sub-assembly modules. These
 are utilizing \emph{OpenRISC} 32-bit processor core combined with a
 proprietary transceiver. The \emph{OpenRISC} core had been developed
 as community effort of \emph{OpenCores.org}, however \emph{Jennic}
 software source code is not available and a rather low quality binary
 library distribution is offered. This software had been analysed, but
 the results are irrelevant to this report.

 \emph{Texas Instruments} currently offer a \Chip{CC2530} based on
 \emph{8051} MCU with 2.4GHz radio, which is most like a product for
 designers who prefer to use \emph{8051} and this chip doesn't feature
 anything else in particular. Italian company \emph{Telit Communications}
 sells sub-assembly RF and USB dongles featuring the \Chip{CC2530} chip.
 Another interesting product from \emph{Texas} is \Chip{CC8520} which
 allows low-power compressed audio transmission in the ISM band.

 At the time of writing of this report the product range of \emph{TI}
 wireless ICs has expanded. It now has \Chip{CC2511} \emph{SoC}
 that is based on \Chip{CC2530} with a USB interface added and
 yet another integrated family of \emph{TI's} \emph{MSP430} 16-bit
 core and sub-gigahertz \Chip{CC1101} transceivers (the 2.4GHz devices
 are expected soon).

\subsubsection{8-bit Atmel AVR ATmega128RFA1}

 The \Chip{ATmega128RFA1} \cite{atmel:atmega128rfa1:datasheet} chip
 has 16MHz clock and contains 128kB of flash, 4kB of EEPROM and 16kB
 of SRAM memories, 32 general purpose registers, 35 GPIO lines, 8
 channels of 10-bit ADC and 6 configurable timers, counters and PWM
 as well as USART, SPI and JTAG interfaces. \Chip{ATmega1281} core
 of the \Chip{ATmega128RFA1} is a generic 8-bit microcontroller.
 Apart from integrated \Chip{AT86RFA231} \cite{atmel:at86rf231:datasheet}
 low-power 2.4GHz \emph{P802.15.4} transceiver, there are no other
 distinct features, it is a stock-standard AVR chip, but is us first
 Wireless MCU from \emph{Atmel}.

 A choice of software stacks is provided \cite{atmel:avr2070,
 atmel:avr2025, atmel:avr2102, atmel:zbpro}, licensing and architectural
 aspects are being discussed in dedicated section of the WMI website
 (\ref{sec:atmellic}) \cite{wmi:wiki:atmelsw}.

\subsection{Boards and Peripherals}

 \emph{Note: Product URLs are provided on dedicated page of the WMI
 website \cite{wmi:wiki:devhw}.}
 \newline

 For a radio controller board (RCB) one of the major factors is the
 type of antenna. Type of sensor connectors and serial interface,
 as well as presence of JTAG header are also take in consideration.

\subsubsection{\texttt{"32-bit"}}

 \emph{Freescale} offers a variety of evaluation boards featuring
 \Chip{MC1322x} chips and some interesting sensor ICs. The sensors
 from \emph{Freescale} may be evaluated in a later stage of this
 project, it could had been a good option to chose \emph{Freescale's}
 evaluation boards as a platform solution for our task, however the
 price of development hardware from original vendors is quite high
 and does not fit into the budget. There is number of \Chip{MC1322x}
 RCBs from 3rd-party suppliers, an up-to-date list is available
 from \emph{Contiki} \Chip{MC1322x} port website
 \cite{links:contiki:rcb:mc1322x}.

\subsubsection{\texttt{"16-bit"}}

 An interesting device that is using TI MSP430 16-bit MCU with
 discrete transceiver chip is \emph{Zolertia Z1} board. The node
 PCB also features 3-axis accellerometer and two 3-pin sensor
 connectors (compatible with \emph{Phidgits} sensors).
 USB serial interface chip and two anennas are other distinct
 features of this device. A chip antenna is soldered on to the
 \emph{Z1} PCB and second micro-FL socket is provied for
 alternative antennae options. A \emph{Z1} version with
 extra peripherals (JTAG, battery holder and external antenna)
 as well as compact plastic enclosure will be soon available
 from \emph{Zolertia} sore \cite{links:zolertia:store}.
 
\subsubsection{\texttt{"8-bit"}}

 \emph{Atmel} offers a range of RCB evaluation kits
 \cite{links:atmel:rcb}, there a few \emph{ZigBit-series}
 modules which feature an external RF power amplifier,
 although those could be great board to use the 3rd-party
 solutions are more suitable for the budget.

 \begin{figure}
 \includegraphics[scale=2.0]{../figures/images/sparkfun_atmega128rfa1_img00.jpg}
 \caption{\emph{SparkFun} \Chip{ATmega128RFA1} \emph{Development Board}} \label{fig:sparkfun:atmega128rfa1:image}
 \end{figure}

 Two development board options were considered for this project.

 First \Chip{ATmega128RFA1} board that had been looked at 
 was \emph{Dresden Elktronik Radio Controller Board RCB128RFA1}
 \cite{links:de:rcb}. In combination with \emph{Sensor Terminal
 Board} \cite{links:de:stb} it is a very suitable development 
 and prototyping solution. The \emph{RCB} has external antenna
 connector and the MCU chip is under metal shielding and battery
 holder. The \emph{STB} also appears to be of very good quality
 and has screw sensors and JTAG, ISP, FIFO USB and DC power
 connectors, as well as external 32kB SRAM chip.

 The \emph{SparFun} board pictured in Figure
 \ref{fig:sparkfun:atmega128rfa1:image}
 is a much more minimalistic and compact device,
 therefore is could fit into a slim enclosure with
 \emph{I/O} breakout connections arranged, or otherwise
 just a few sensors inside of enclosure for a basic
 application. The circuit diagram for this device is
 shown in Figure \ref{fig:sparkfun:atmega128rfa1:circuit}.

\subsection{Popular Solutions}

 The \emph{Xbee}\cite{links:digi:xbee} modules from \emph{Digi}
 are very popular among hobbyists, however it is rather a
 drop-in solution, and is not appropriate for a new design.
 \emph{Xbee} modules can also be used without external MCU
 \cite{links:misc:xbeemidi}, but the functionality is fixed
 to what is implemented in module firmware. Also a recently
 published \titleof{noble2009programming} book suggests to
 use the \emph{Xbee} modules.
 One product to mention is \emph{IcludeTech WiMi} \cite{links:includetech:wimi}
 development board which uses an \emph{Xbee} module with whip antenna and a
 \emph{Microchip PIC} MCU. This is appropriate as a drop-in module
 for a MIDI device which already is being produced and wireless
 connectivity needs to be offered as an expansion option, engineers
 who are familiar with \emph{PIC} microcontrollers may find these
 devices suitable for their designs.
