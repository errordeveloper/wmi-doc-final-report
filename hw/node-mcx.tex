\section{Selecting the Development Platform for Contiki OS} \label{sec:MCX}

  Examining the code for various hardware platforms in the source
 tree of \ContikiOS\, it was understood that source code for
 \Chip{MC1322x} devices from \emph{Freescale} is organised in a
 much clearer way. Current implementation appears to support most
 of important features of these chips and, in fact, these are
 very robust devices. As mentioned earlier, \Chip{MC1322x} is an
 \Chip{ARM7TDMI} microcontroller with fully integrated radio and
 the only component external to the chip itself is the antenna.
 Apart from this, the \emph{Freescale} device has an outstanding
 set of peripheral and rather large amount of memory (128KB of
 flash, 96kB of RAM and 80kB of ROM) the processor clock frequency
 is 24MHz. Comparing to the minimum amount of memory required
 to run \Contiki (2kB of RAM and 40kB of ROM) there is a very
 large headroom available for the application and some additional
 drivers. This also makes it a great development platform, during
 debugging stages compiler optimisation often needs to be disabled
 and additional memory may be also populated with debug data.

\subsection{MC1322x Architecture Overview}

 Below is the diagram of \MCX\ \emph{Platform in Package (PiP)}.

\begin{figure}
% http://cache.freescale.com/files/graphic/block_diagram/2150_MC1322X_BD.gif
\caption{\emph{The block diaghram of Freescale MC1322x}}
\end{figure}

  In addition to standard peripherals (UART, 12-bit A/D converters,
 SPI and I2C) this device has \emph{Synchronous Serial Interface (SSI)}
 that would allow communication with I2S devices as well as other
 synchronous serial peripherals. Being a 32-bit microcontroller
 it can be used for some basic audio signal processing. For example,
 with \emph{Analog Devices} \Chip{ADMP441} omnidirectional microphone
 with 24-bit I2S digital output \cite{datasheet:adi:admp441} an
 acoustic measurement sensor node can be designed. According to the
 datasheet, \Chip{ADMP441} has a very linear frequency response in
 the band between 100Hz to 15kHz. Such wireless node would be
 suitable for field noise measurement at popular music festivals
 or construction sites and industrial areas.
 Another sensor that could be included on such board would be an
 accelerometer, according to \emph{Freescale Application
 Note AN3751} \cite{appnote:freescale:AN3751} by utilising standard
 DSP techniques on accelerometer signal data, various vibration
 frequency analysis results can be produced. Another feature of
 the \MCX\ is it's accelerated MAC unit, called \emph{MACA}.

\subsection{MC1322x Development Hardware}

  The homepage of \Chip{MC1322x} port of \contiki has a detailed
 overview of what development boards are available \cite{homepage:mc1322x:hw}.
 Two boards which were chosen for this project are \emph{Freescale
 1322xUSB} and \emph{Redbee Econotag}.

  A minimum of two devices is required to implement a \WPAN\ network.
 At the early stage, the \emph{SparkFun} boards were supposed to be
 used the sensor node and gateways device. When the decision had been
 made to use \MCX\ platform, the \emph{Freescale} dongle was found to
 be the most appropriate device to run \Contiki border router program,
 which implements end-to-end IP communication\footnote{\emph{The source
 code is located in: \Blob{examples/ipv6/rpl-border-router/}}}.

\subsubsection{Freescale USB Dongle}

 \emph{1322xUSB} is a very small USB device that doesn't provide the
 access to any of the pin and therefore is well suited as a network
 gateway. The circuit diagram is shown in figure \ref{cir:freescale-usb}
 demonstrate the minimalistic layout of this board.

\subsubsection{Redbee Econotag}

 \emph{Econtag} by \emph{Redbee LLC} is highly regarded open-source
 development board. It is specifically design to meet all important
 requirenements of a development board with very constrained bill of
 materials. The board provides access to all of the I/O pins and
 and is very convenient for programming and debugging. Both of these
 functions are achived with the help of dual-port USB UART chip,
 \emph{FTDI} \Chip{FT22232HL} \ref{datasheet:ftdi:dual}. One of the two
 ports is connected to the primary UART interface and the second port
 provides access to the JTAG. The \Chip{FT2232HL} IC also facilitates
 bit-bang software reset, the \emph{Freescale} dongle cannot be reset
 in this way.

\subsection{Hardware Programming Facilities}

  To program the \Chip{MC1322x}, the first UART port is used. It also
 used for standard serial port output. The program is localed into
 RAM and doesn't wear-off the flash memory each time. Thought, for
 flashing the \MCX\ devices, a two-stage procedure is needed. First,
 a self-flasher code has to be loaded, which in turns loads the
 main program code into the flash memory.
 
  \Contiki\ source code contains the self-flasher implementations as
 well a set of utilities for programming the \MCX\ devices. These are
 located in \Blob{cpu/mc1322x/tools/}. However, this is not the only
 way to update the software. Dunkels describes the design of \Contiki
 \emph{Executable and Linkable Format} \cite{dunkels06runtime} loader.
 There are a few implecation of using ELF loader, there has to be
 a certain amount of spare memory for the code to be downloaded into.
 However, it gives a great advantage of reprogramming over-the-air,
 without a need for physical access to the node.

% http://mc1322x.devl.org/doc/redbee-econotag/redbee-econotag-r2.pdf
% http://www.ftdichip.com/Products/ICs/FT2232H.htm
% http://www.ftdichip.com/Support/Documents/DataSheets/ICs/DS_FT2232H.pdf
