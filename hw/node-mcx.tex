\section{Selecting the Development \\Platform for Contiki OS} \label{sec:MCX}

  Examining the code for various hardware platforms in the source
 tree of \Contiki, it was understood that source code for
 \Chip{MC1322x} devices from \emph{Freescale} is organised in a
 much clearer way. Current implementation appears to support most
 of important features of these chips and, in fact, these are
 very robust devices. As mentioned earlier, \Chip{MC1322x} is an
 \Chip{ARM7TDMI} microcontroller with fully integrated radio and
 the only component external to the chip itself is the antenna.
 Apart from this, the \emph{Freescale} device has an outstanding
 set of peripheral and rather large amount of memory\footnote{\emph{%
 An average size of compiled code for the Talker program was
 reported to be 79 kB.}} (128KB of flash, 96kB of RAM and 80kB
 of ROM) the processor clock frequency is 24MHz. Comparing to
 the minimum amount of memory required to run \Contiki\ (2kB of RAM
 and 40kB of ROM) there is a very large headroom available for the
 application and some additional drivers. This also makes it a great
 development platform, during debugging stages compiler optimisation
 often needs to be disabled and additional memory may be also
 populated with debug data.

\subsection{MC1322x Architecture Overview}

 %Below is the diagram of \MCX\ \emph{Platform in Package (PiP)}.

\begin{figure}
\centering
\includegraphics[angle=-90,scale=0.76]{../figures/images/freescale_mc1322x_bd.png}
\caption{\emph{The block diagram of Freescale MC1322x}} \label{fig:mc1322x}
\end{figure}

  In addition to standard peripherals (UART, 12-bit A/D converters,
 SPI and I2C) this device has \emph{Synchronous Serial Interface (SSI)}
 that would allow communication with I2S devices as well as other
 synchronous serial peripherals. Being a 32-bit microcontroller
 it can be used for some basic audio signal processing. For example,
 with \emph{Analog Devices} \Chip{ADMP441} omnidirectional microphone
 with 24-bit I2S digital output \cite{datasheet:adi:admp441} an
 acoustic measurement sensor node can be designed. According to the
 datasheet, \Chip{ADMP441} has a very linear frequency response in
 the band between 100Hz to 15kHz. Such wireless node would be
 suitable for field noise measurement at popular music festivals
 or construction sites and industrial areas.
 Another sensor that could be included on such board would be an
 accelerometer, according to \emph{Freescale Application
 Note AN3751} \cite{appnote:freescale:AN3751} by utilising standard
 DSP techniques on accelerometer signal data, various vibration
 frequency analysis results can be produced. Another feature of
 the \MCX\ is it's accelerated MAC unit, called \emph{MACA}.

\subsection{MC1322x Development Hardware}

  The homepage of \Chip{MC1322x} port of \Contiki\ has a detailed
 overview of what development boards are available
 \cite{links:contiki:rcb:mc1322x}.
 Two boards which were chosen for this project are \emph{Freescale
 1322xUSB} and \emph{Redbee Econotag}.

  A minimum of two devices is required to implement a \WPAN\ network.
 At the early stage, the \emph{SparkFun} boards were supposed to be
 used as the sensor node and the gateway. When the decision had been
 made to use \MCX\ platform, the \emph{Freescale} dongle was found to
 be the most appropriate device to run the \Contiki\ \emph{Border Router}
 program, which implements end-to-end IP communication\footnote{\emph{The
 source code is located in: \Blob{examples/ipv6/rpl-border-router/}}}.

\subsubsection{Freescale USB Dongle}

 \emph{1322xUSB} is a very small USB device that doesn't provide the
 access to any of the pin and therefore is well suited as a network
 gateway. The circuit diagram is shown in figure \ref{fig:freescale-usb}
 to demonstrate the minimalistic layout of this board. One the bottom
 side of the board, there is a header for a JTAG connector and a few
 test points. It has appeared that the board has already been flashed
 in the factory, and the firmware had to be erased when the board was
 used for the first time to run the \Contiki\ code. The information on
 how to erase the firmware was obtained from the datasheet
 \cite{links:freescale:1322xusb}. This device was found rather primitive,
 its only benefit is the size, though it also serves as the minimal reference design.

\begin{figure}
\centering
\includegraphics[scale=0.5]{../figures/images/freescale_1322xusb.png}
\caption{\emph{The circuit layout of Freescale 1322xUSB}} \label{fig:freescale-usb}
\end{figure}

\subsubsection{Redbee Econotag}

 \emph{Econotag} by \emph{Redbee LLC} is highly regarded open-source
 development board. It is specifically design to meet all important
 requirenements of a development board with very constrained bill of
 materials. The board provides access to all of the I/O pins and
 and is very convenient for programming and debugging. Both of these
 functions are achived with the help of dual-port USB UART chip,
 \emph{FTDI} \Chip{FT22232HL} \cite{datasheet:ftdi:dual}. One of the
 two ports is connected to the primary UART interface and the second
 port provides access to the JTAG. The \Chip{FT2232HL} IC also facilitates
 bit-bang software reset, the \emph{Freescale} dongle cannot be reset
 in this way.

\subsection{Hardware Programming Facilities}

  To program the \Chip{MC1322x}, the first UART port is used. It also
 used for standard serial port output. The program is localed into
 RAM and doesn't wear-off the flash memory each time. Thought, for
 flashing the \MCX\ devices, a two-stage procedure is needed. First,
 a self-flasher code has to be loaded, which in turns loads the
 main program code into the flash memory.
 
  \Contiki\ source code contains the self-flasher implementations as
 well a set of utilities for programming the \MCX\ devices\footnote{%
 \emph{These are located in \Blob{cpu/mc1322x/tools/}.}} However, this
 is not the only way to update the software. Dunkels describes the design
 of \Contiki \emph{Executable and Linkable Format} \cite{dunkels06runtime}
 loader.  There are a few implication of using ELF loader, e.g. there
 has to be a certain amount of spare memory for the code to be downloaded
 into. However, it gives a great advantage of reprogramming over-the-air,
 without a need for physical access to the node.


\begin{figure}
\centering
\includegraphics[angle=90]{../figures/econotag_circuit_sheet1_monochrome.pdf}
\caption{\emph{Econotag schematics (sheet 1)}}\label{fig:econotag:circuit:s1}
\end{figure}

\begin{figure}
\centering
\includegraphics[angle=90]{../figures/econotag_circuit_sheet2_monochrome.pdf}
\caption{\emph{Econotag schematics (sheet 2)}}\label{fig:econotag:circuit:s2}
\end{figure}

\pagebreak
\subsection{Econotag Circuit}


  The schematics show in figures \ref{fig:econotag:circuit:s1},
  \ref{fig:econotag:circuit:s2} and
  the board layout in \ref{fig:econotag:layout} had been taken
  from the design package \cite{links:econotag:design} that is
  distribute by \emph{RedWire} under \emph{Creative Commons}
  license. These are included here to approve the simplicity
  of circuit layout with highly integrated \MCX\ chip.

   Comparing to the \emph{SparkFun} board, the \emph{Econotag}
  demonstrates much superior development solution with the
  inclusion of dual USB chip, that provides software reset and
  JTAG facilitates. Also many other minor aspects of this board
  provide a greater benefit then the earlier board. Althought,
  the layout shown above does not include the ground plane, by
  by visual comparison, the \emph{Econotag} has a very solid
  ground plane, unlike the \emph{SparkFun} board.

   The area where more experiments are still to be done, is the
  power consumption measurement. It is know however, that the
  \emph{Econtag} will run of two cell batteries for up to 48
  hours without utilising any of the sleep modes, however this
  information still has to be proven.


\begin{figure}
\centering
\includegraphics[angle=90,scale=1.75]{../figures/econotag_board_top.pdf}
\includegraphics[angle=90,scale=1.75]{../figures/econotag_board_bot.pdf}
\caption{\emph{Econotag development board layout (top and bottom)}} \label{fig:econotag:layout}
\end{figure}

\pagebreak
\subsection{Photographs of the Hardware in Use}

  The photograph in figure \ref{fig:photo:1} shows the top view
 of the \emph{Econotag} with an \Chip{ADXL345}\footnote{\emph{%
 More details on the use of the sensor are to follow.}}.
 break-out board soldered to the SPI pins. The second image
 (\ref{fig:photo:2}) is the bottom view of the Econotag with
 MIDI cable soldered to the UART2 pins, the DIN5 MIDI connector
 and the \emph{Freescale} dongle. It can observed the size of
 the dongle is very close to the size of DIN5 connector,
 therefore a device can be designed which would be of a size
 little bigger then the DIN5 connector body, though the power
 would need to be supplied. The standard midi connector
 has 5 pins and only two are used by most of the devices,
 hence, theoretically, this could be implemented. However,
 in practice, a battery will be required, since the modification
 of MIDI device to provide a power supply is not an option.
 The lack of bus power is another reason to consider MIDI
 being a legacy protocol.

\begin{figure}
\centering
\includegraphics[scale=0.8]{../figures/images/photos/1c.png}
\caption{\emph{Econotag (top view)}}\label{fig:photo:1}
\end{figure}

\begin{figure}
\centering
\includegraphics[scale=0.8,angle=90]{../figures/images/photos/2c.png}
\caption{\emph{Freescale USB dongle and the Econotag (bottom view)}}\label{fig:photo:2}
\end{figure}
