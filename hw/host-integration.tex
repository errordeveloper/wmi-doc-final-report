\section{Key System Components: \\ Specification \& Requirements} \label{sec:SPECS}

  This section outlines the specification which had been proposed
 after initial research phase has been accomplished. Diagram shown
 in figure \ref{fig:sketch1} illustrates the function blocks of the
 complete system that meets this specification, however it is now
 considered suitable only for the development system and not the
 final product design.

\subsection*{Sensor Node}

\emph{Sensor Node} is a device featuring a microcontroller chip
that contains following function blocks and peripherals:

\begin{itemize}
	\item \emph{internal:}
	\begin{itemize}
		\item 2.4Ghz RF transceiver,
		\item A/D converters,
		\item GPIO,
		\item SPI, and
		\item USART
	\end{itemize}
	
	\item \emph{external:}
	\begin{itemize}
		\item digital or analogue sensors,
		\item external connectors for MIDI,
		\item wired remote sensors, and
		\item serial port header
	\end{itemize}
\end{itemize}

The \emph{Sensor Node} runs software which consists of:

\begin{itemize}
	\item \emph{operating system:}
	\begin{itemize}
		\item communication protocol stack,
		\item devices drivers, and
		\item application task management,
		\item service tasks, and
		\item the main program
	\end{itemize}
	\item \emph{bootloader:}
	\begin{itemize}
		\item loads new software
	\end{itemize}
\end{itemize}

\begin{figure}
\centering
\includegraphics[scale=0.4]{../figures/sketch1.pdf}
\caption{\emph{Abstract Sketch of the Development System}} \label{fig:sketch1}
\end{figure}


\pagebreak
\section{Aspects of Final Product Design}

   Many commercial uses of \WPAN in the field of audio control were
 considered. The block diagrams in figure \ref{fig:products} briefly
 illustrate a few interesting solutions. Some of the device proposed
 here can be used in a variety of applications and others are rather
 specific to live sound and stage performance.

  Current implementation can be used to some extend, however one
 very important modification of hardware needs to be considered.
 The connectivity between the host and border router, in current
 development system, is implement via the USB serial interface.
 This involves unnecessary hardware and software components and
 should eliminated from a commercial design.
  One Solution would to utilise SPI bus, which is available on
 most of the SoCs. A transceiver can be connected directly to
 the host SoC and there will need to be a Linux driver for it.
 The second solution is to implement an abstract Linux/Contiki
 framework, such that would use SPI for the datapath and GPIO
 for the interrupts. In this way, a very robust device can be
 designed (see the \emph{"Wired Gateway Board"} diagram in figure
 \ref{fig:products}).

  Another important idea presented in \ref{fig:products}, is the
 \emph{"Tiny Node"}. It proposes a general-purpose boar, which
 would suite a wide range of application. Provided that an
 appropriate physical connection is defined, this board may
 have removable sensor part for each of the possible uses.

\begin{figure}
\centering
\includegraphics[scale=0.8]{../../poster/figures/figure1.pdf}
\caption{\emph{Block diagrams of some possible commercial solutions}} \label{fig:products}
\end{figure}
