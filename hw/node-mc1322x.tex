\section{Selecting the Development Platform for Contiki OS}

  Examining the code for various hardware platforms in the source
 tree of \ContikiOS\, it was understood that source code for
 \chip{MC1322x} devices from \emph{Freescale} is organised in a
 much clearer way. Current implementation appears to support most
 of important features of these chips and, in fact, these are
 very robust devices. As mentioned earlier \Chip{MC1322x} is an
 \Chip{ARM7TDMI} microcontroller with fully integrated radio and
 the only component external to the chip itself is the antenna.
 Apart from this, the \emph{Freescale} device has an outstanding
 set of peripheral and rather large amount of memory (128KB of
 flash, 96kB of RAM and 80kB of ROM) the processor clock frequency
 is 24MHz. Comparing to the minimum amount of memory required
 to run \Contiki (2kB of RAM and 40kB of ROM) there is a very
 large headroom available for application and additional drivers.
 This also makes it a great development platform, during debugging
 stages compiler optimisation often needs to disable and additional
 memory may be also populated with debug data.

\subsection{MC1322x Architecture Overview}

 Below is the diagram of \MCX\ \emph{Platform in Package (PiP)}.
% INCLUDE THIS IMAGE:
% http://cache.freescale.com/files/graphic/block_diagram/2150_MC1322X_BD.gif

  In addition to standard peripherals (UART, 12-bit A/D converters,
 SPI and I2C) this device has \emph{Synchronous Serial Interface (SSI)}
 that would allow communication with I2S devices as well as other
 synchronous serial peripherals. Being a 32-bit microcontroller
 it can be used for some basic audio signal processing. For example,
 with \emph{Analog Devices} \Chip{ADMP441} omnidirectional microphone
 with 24-bit I2S digital output \cite{datasheet:adi:ADMP441} an
 acoustic measurement sensor node can be designed. According to the
 datasheet, \Chip{ADMP441} has a very linear frequency response in
 the band between 100Hz to 15kHz. Such wireless node would be
 suitable for field noise measurement at popular music festivals
 or construction sites and industrial areas.
 Another sensor that could be included on such board would be an
 accelerometer, according to \emph{Freescale Application
 Note AN3751} \cite{appnote:freescale:AN3751} by utilising standard
 DSP techniques on accelerometer signal data, various vibration
 frequency analysis results can be produced.

%% to be added to the bibtex file:

 %http://www.freescale.com/files/rf_if/doc/ref_manual/MC1322xRM.pdf
 %http://www.freescale.com/files/rf_if/doc/app_note/AN3751.pdf
 %http://www.analog.com/static/imported-files/data_sheets/ADMP441.pdf
 
%% write about freescale usb stick and the Econotag, include circuits

\subsection{MC1322x Development Hardware}

  The homepage of \Chip{MC1322x} port of \contiki has a detailed
 overview of what development boards are available \cite{homepage:mc1322x:hw}.
 Two boards which were chosen for this project are \emph{Freescale
 1322xUSB} and \emph{Redbee Econotag}.
 \emph{1322xUSB} is a very small USB device that doesn't provide the
 access to any of the pin and therefore is well suited as a network
 gateway. \emph{Redbee Econotag} is an open-source development board.
  Circuit diagrams are shown in figures \ref{cir:econotag} and 
 \ref{cir:freescale-usb}.

\subsubsection{Redbee Econotag}

  The \emph{Econotag} is specifically design to meet all important
 requirenements of a development board with very constrained bill of
 materials. The board provides access to all of the I/O pins and
 and is very convenient for programming and debugging. Both of these
 functions are achived with the help of dual-port USB UART chip,
 \emph{FTDI} \Chip{FT2??} \ref{datasheet:ftdi:dual}. One of the two
 ports is connected to the primary UART interface and the second port
 provides access to the JTAG.

 

