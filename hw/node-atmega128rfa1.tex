\section{Initial Hardware Consideration}

  At the initial stage of this project the hardware platform was chosen
 based on previous experience of using development tools for \emph{Atmel
 AVR} microcontrollers. The original orientation was towards using a chip
 with integrated radio transceiver, \emph{Atmel} advertises the \RFA\ chip
 as having the lowest power consumption and most appropriate link budget.

\subsubsection{Development Boards for ATmega128RFA1}

  Two choises of development boards were considered:

	\begin{itemize}
		\item Dresden
		\item SparkFun
	\end{itemize}

  \emph{DE} board features a useful set of components, including spear flash
 memory and robust screew terminals for I/O connections. However, the board
 from \emph{SparkFun} has been chosen for it's low price.
  It has been found later that this board was inconvenient to use in various
 ways, for example the layout of serial port pins on the side of the board
 (REFERENCE THE PIN NAMES HERE) could be design to fit standard serial USB
 cables\footnote{\emph{SparkFun sells few different serial USB adaptors all
 of which have the same layout. These are know to be very popular among most
 of hobbists and professional engineers and therefore are considered to be
 a de-facto standard.}}. Also the layout of transciever side of the PCB was
 found to be quite primitive and, most importantly, has not included a
 suitable ground plane. Nevertheless, this board is of a rather small footprint
 and could be used in a prototype product.

%% inlude: PICTURE, CIRCUIT & BOARD LAYOUTS
