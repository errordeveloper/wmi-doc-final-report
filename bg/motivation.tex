\section{Motivation}

  Recent advances in low-power microcontroller and radio frequency data communications
 technology, together with enormous growth in the general purpose embedded computer
 market challenges several industries around the world. However, the music instrument
 industry haven't fully taken all the advantages of the most recent electronic devices
 and standards.
 
 Wireless sensor networks will very soon appear in numerous application areas,
 which may include practically any market (consumer, industrial, medical and many other).
 
 This project aims to implement a networked system for stage performance which uses
 wireless sensor devices as a creative user interface. Despite the title, this also
 concerns any theatrical or dance performance as well as various monitoring and
 control uses in entertainment systems.

\subsection{System Specification \& Requirements}

   This section shall outline the background for technical
 requirements for the system which is about to be discussed
 in this report. Some essential non-technical information
 will be introduced briefly. Below is a very general outline
 of what the requirements are.

  \begin{itemize} \em
	  \item	Inexpensive hardware
	  \item	Low-power components
	  \item	Small form-factor
	  \item	Most recent radio technologies

  \item The system design also should:

  \begin{itemize}
	  \item	be suitable for various similar applications
	  \item	and avoid theoretical limits in scalability 

  \end{itemize}
  \end{itemize}


 Observing contemporary arts and music scene it appears that
 that technology becomes increasingly popular among artists
 and to some surprise there are individuals who attempt to
 introduce non-trivial aspects of technology into visual and
 performed arts. Some are utilising commodity devices, such
 as mobile phones, while others desire to learn microcontroller
 programming and simple circuit design for physical interaction
 with sensors as a creativity instrument.
  As an example, Ryan Jordan in his MFA thesis \cite{paper:ryan09}
 states that \emph{"A fragile DIY hardware and software system has
 been created with various sensors which are attached directly to
 the performers body."}. The word "fragile" atracts the attention
 in this sentense, hence if a system was made such that an artist
 could apply for their performance (or display) in a flexible,
 robus and relieble manner, there is, probably, a niche market
 for it. Ryan Jordan has probably found certain intsperation in
 the actual wiring of his system, wireless provides a diffrent
 benefit to the artist.


