
\section{Organisation}

  The very initial research and preparation phases of this project had taken place
 prior the start of the academic year \emph{2010/11}. When the project has started
 in October, one of the first steps taken was a creation of website, which provides
 a number of important facilities for project management, file keeping and issue
 tacking. The website can be found at the URL below:
 {\URL{http://wmi.new-synth.info}}
 It will be referred to throughout this report as \emph{The WMI Website}. A current
 source code repository is also available on-line and linked to the website as well
 as work in progress notes and other supplementary information and data.

\subsection{Report Structure}

  The report had been divided into two parts. Part I introduces some
  general concepts and gives a summary of the initial phase of the
  research. Part II contains all the details related to development
  process, presents problems and discusses various related solutions.

  There are several chapters in Part II, these were organised in
  the logical order. The best effort had been made to clearly
  describe all necessary details and progress towards the final
  implementation chapter, without any degree of obscurity. It is,
  however, up to the reader whether some references should be
  followed and analysed. A few assumptions had been made with
  the hope that the reader has certain background to follow
  the ideas which are being presented.

  It is important to note, that most of the software which had
  been used and/or written throughout the project, is located
  in the on-line repository (see section \ref{sec:conv}). That
  also provides the logging facility, hence no handwritten
  log book will be submitted with the report. However, a
  supplementary disk is being submitted as a form of evidence.
  Some of the data on the disk is a subject to the terms and
  conditions of a respective copyright licenses, those components
  were included with the believe that the disk serves the purpose
  of a personal back-up copy which is not a distribution media.


\pagebreak
\subsection{Typographic Conventions} \label{sec:conv}

  Unfamiliar names and acronyms mentioned in the report are typeset in \emph{italic},
 as well as vendor brands when referred to a product from that vendor. Names of device
 models as well as commands and programming language keywords or statements are typed
 in \emph{\texttt{bold-italic}}. Filenames are \texttt{mono-spaced} and in the PDF
 version of the document are hyper-linked to the source code repository\footnote{%
 \emph{The path is aways relative to the source code root directory, unless specified
 otherwise.}}. There will appear a special sub-section titled \emph{"Tracked Issues"}
 in some sections, it shows reference to issue tickets on the website\footnote{\emph{%
 The tickets are all enumerated in one sequence starting from 1 and classified by the
 type of issue each ticket relates to (i.e. "bug", "feature", "task"). Best effort
 was made to file these issue and there are very few remaining undocumented.}}.
 The tickets can be accessed via URL of the following form (replace the \Symb{`\#'}
 symbol with the given number): \URL{http://wmi.new-synth.info/issues/\#}

  The base URL to access any references to source code is:
  \URL{https://github.com/errordeveloper/contiki-wmi/blob/wmi-work/}

  For example, to see the file \texttt{README} this URL should be used:
  \URL{https://github.com/errordeveloper/contiki-wmi/blob/wmi-work/README}

  Similarly, to view the commit log for a certain file use:
  \URL{https://github.com/errordeveloper/contiki-wmi/commits/wmi-work/}

