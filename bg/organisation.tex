
\section{Organisation}

  The very initial research and preparation phases of this project had taken place
 prior the start of the academic year \emph{2010/11}. When the project has started
 in October, one of the first steps taken was a creation of website, which provides
 a number of important facilities for project management, file keeping and issue
 tacking. The website can be found at the URL below:
 {\URL{http://wmi.new-synth.info}}
 It will be referred to throughout this report as \emph{The WMI Website}. A current
 source code repository is also available on-line and linked to the website as well
 as work in progress notes and other supplementary information and data.

\subsection{Report Structure}

\subsection{Typographic Conventions}

  Unfamiliar names and acronyms mentioned in the report are typeset in \emph{italic},
 as well as vendor brands when referred to a product from that vendor. Names of device
 models as well as commands and programming language keywords or statements are typed
 in \emph{\texttt{bold-italic}}. Filenames are \texttt{mono-spaced} and in the PDF
 version of the document are hyper-linked to the source code repository\footnote{%
 \emph{The path is aways relative to the source code root directory, unless specified
 otherwise.}}. There will appear a special sub-section titled \emph{"Tracked Issues"}
 in some sections, it shows reference to issue tickets on the website\footnote{\emph{%
 The tickets are all enumerated in one sequence starting from 1 and classified by the
 type of issue each ticket relates to (i.e. "bug", "feature", "task"). Best effort
 was made to file these issue and there are very few remaining undocumented.}}.
 The tickets can be accessed via URL of the following form (replace the \Symb{`\#'}
 symbol with the given number): \URL{http://wmi.new-synth.info/issues/\#}
