\chapter{Field \& Market Trends}

 This chapter intends to review the variety of available
 technologies and discuss why some desitions and preference
 had been made.

 \emph{The terminology of the OSI 7-layer model will be used,
 therefore table \ref{tab:osi} will illustrate the stacking
 of all layers in comparison to the Internet Protocol model.
 The reader is expected to be already familiar with this
 terminology, hence it is shown only for their reference.}

\begin{table}[h]
\begin{center}
 \begin{tabular}{|c||c|}
 \hline
 {\bf OSI model}&{\bf IP model}\\
 \hline \hline
 \ \ \  Application \ \ \   & \multirow{2}{*}{ \ \ \   Application \ \ \ }   \\
 \cline{1-1}
 Presentation & \\
 \hline
 Session & \multirow{2}{*}{TCP, UDP,\  \dots }\\
 \cline{1-1}
 Transport & \\
 \hline
 Network & IP\\
 \hline
 Data Link & Data Link\\
 \hline
 Physical & Physical\\
 \hline
 \end{tabular}
 \end{center}
\caption{\emph{Open System Interconnection Model
	and The Internet Protocol}}\label{tab:osi}
\end{table}


\section{Music Control Protocols}

   One of additional motivations to work on low-power wireless
  network for music control, had been a desire to experiment in
  the area of high-level representation of control data, specific
  to live music performance and audio in general.
  Started by considering to extend, not so recently proposed, OSC
  ("Open Sound Control") protocol \cite{paper:osc11}, with a set
  features which it appears to be missing. This is a subject to
  more extensive experimentation and therefore has not been
  included in the scope of the project itself.

   OSC is effectively just an application layer data format and
  is mostly used with UDP. It had been proposed by a group of
  researchers at the Centrer for New Music \& Audio Technologies
  (CNMAT), UC Berkley \cite{links:cnmat}. It was first presented
  in 1997 \cite{paper:osc97} and has received a rather limited
  adoption. Although, it was observed that there is a tendency
  towards a wider adoption of OSC - a number of interesting
  hardware product had been released which use OSC as primary
  protocol. Some of these devices are listed below.

  \begin{itemize}
  	\item \emph{Livid Block64 \cite{links:livid:block64}}
	\item \emph{Jazz Mutant Lemur \cite{links:jazzmutant:lemur}}
	\item \emph{Monome \cite{links:monome, links:monome:osc}}
  \end{itemize}

  It has to be said that OSC seems to be intended as a candidate
  to replace the MIDI protocol (which had been define in 1983 and
  therefore considered in need of a replacement). The greatest
  limitation seen in MIDI today is the size of values it can
  represent, i.e. most control values a bound in 0-127 range.
  Nevertheless, wireless transmission of MIDI was chosen to
  be the initial target for this project. To avoid going
  outside of the scope of this report, it shall be defined
  that MIDI is quite likely to be most appropriate to implement
  at this stage, since OSC format is certainly not suitable.
  The reason for this is that microcontrollers do not have
  the capabilities to handle large amount of data represented
  as character strings. Therefore a new protocol needs to be
  designed, which would overcome most these limitations; though,
  that is already in the scope of another project.



\section{Low-power Digital Radio}

  This section gives an overview of currently available low-power
 wireless communication technologies in general terms, then some of
 the important concepts are briefly introduced to support further
 discussion of these devices and software with appropriate terminology.

  One of the main subjects of the initial research was wireless data
 communication standards for sensing and control applications and it
 should be noted that the transmission of audio signals has not been
 taken into consideration. Also as outlined in the requirenments,
 radio protocols such as UWB (e.g. WUSB, WiMAX) and \emph{IEEE 802.11}
 (i.e. WiFi) are not low-power and therefore are not applicable for the
 purpose of this project. Although, short-range\footnote{\emph{Recent
 amendments in P802.15.4a specify alternative physical layer options
 that include sub-gigahertz UWB modes \cite{links:wiki:p:wpan}.}} UWB
 could be of great use for its potential throughput capabilities, the
 cost and availability of transceivers are yet unknown.
 
  The main interest is in low-power radio of 2.4GHz range. The semiconductor
 market is currently flooded with a variety of inexpensive devices that
 implement either \emph{IEEE 802.15.4} standard or patented protocols such
 as \emph{ZigBee}, \emph{RF4CE} or other that are based on \emph{P802.15.4}.

 A few more different low-power wireless networking technologies exist, such
 as \emph{DASH7} and \emph{ANT}. \emph{DASH7} is an active RFID protocol for
 extended range and it is very specific for certain applications, it operates
 in 434MHz band \cite{links:wiki:p:dash7}. \emph{ANT} is a proprietary standard
 which uses 2.4GHz band \cite{links:wiki:p:ant}. Both of these technologies
 are support only by small groups of silicon chip vendors.

 Multiple standards exist which are using the same hardware functions provided
 by \emph{P802.14.5}-compliant devices\footnote{\emph{This implies that all of
 these protocols are effectively defined only by software.}}, some of these are
 \emph{ZigBee}, \emph{RF4CE} and \emph{6loWPAN} \cite{links:wiki:p:6lowpan}.
 \emph{RF4CE} belongs to the \emph{ZigBee} \cite{links:wiki:p:zigbee} family
 together with a number of other application area specific variations.
 \emph{6loWPAN} \cite{links:wiki:p:6lowpan} is most interesting patent-free
 protocol and it is transparent to existing software, since it is compliant
 to the Internet Protocol. %\IETF

  Nevertheless, all of these technologies are not yet widely available in
 the consumer market, where \emph{BlueTooth} \cite{links:wiki:p:mrwpan} and
 simplistic sub-gigahertz serial radio transceivers or infra-red are commonly
 found. Although, it is most likely that \emph{P802.15.4} transceivers will
 soon dominate low-power wireless application markets.

 Further in this report \emph{P802.15.4} will be referred to as \WPAN\ 
 (stands for Low-Rate\footnote{\emph{Medium-rate (MR-WPAN) are the BlueTooth
 (P802.15.1) networks. Also there is HR-WPAN (P802.15.3) standard defining
 high-rate (UWB) networks. All classes of WPAN together with WLAN are
 are referred to as short-range wireless networks.}} Wireless Personal
 Area Network). A number of amendments to \emph{IEEE 802.15.4} had been
 published, the latest version is \emph{P802.15.4-2006}. In this report
 it shall be looked at as a de-facto solution \cite{links:wiki:p:wpan,
 links:wiki:p:lrwpan, links:ieee:802:15}.

 \emph{Some general concepts of the \WPAN\ hardware described below
 are considered to be absolutely complete for understanding the
 system operation from the software design perspective.}

\subsection{\emph{IEEE 802.15.4} - Low-Rate WPAN}

  This standard was first proposed by the IEEE in 2003 and has
 evolved since. As far as the concepts essestial for application
 development are concerned, there is no major difference between
 the revisions.

 It is important to understand at this point that the concept of
 low-power consumption applies to all layers, so the application
 layer indeed is required to co-operate in order to preserve the
 energy. However, the task of this project is to get maximum
 throughput on \WPAN\ network and attempt to reduce the latency
 and maximize quality of service, hence power preservation
 techniques are considered very briefly. However, the low-power
 requirement for the design is met, since the avarage power rating
 of the device remains relatively low without applying these
 techniques.
 

 \WPAN\ defines two layers of the OSI model, the \emph{Physical (PHY)}
 and \emph{Media Access Control (MAC)} layers. Network and Applications
 layers are defined by other standards mentioned above.


\subsubsection{Physical Layer (PHY)}

\begin{flushright} \small{
\emph{Source: "ZigBee Wireless Networks and Transceivers" \\
	by Shahin Farahani (2008) \cite{b:zigbee}}}
	\end{flushright}


% This layer functions for transmission and reception of radio packets,
%provides control facilities for channel selection and power management.

 "The PHY layer is the closest layer to hardware and directly controls
 and communicates with the radio transceiver.
 The PHY layer is responsible for the following:
 \begin{itemize}

        \item Activating and deactivating the radio transceiver.

	\item Transmitting and receiving data.

	\item Selecting the channel frequency.

	\item Performing Energy Detection (ED).\\
	\small{\emph{The ED is the task of estimating the signal energy within the
	frequency band of interest. This estimate is used to understand
	whether or not a channel is clear and can be used for transmission.}}

	\item Performing Clear Channel Assessment (CCA).

	\item Generating a Link Quality Indication (LQI).\\
	\small{\emph{The LQI is an indication of the quality of the data packets
	received by the receiver. The signal strength can be used as
	an indication of signal quality.}}"
	\end{itemize}

 The \WPAN\ standard defines use
 for a number of bands in different geographical regions, the 2.4GHz
 band can be used anywhere in the world.  The details regarding RF
 modulation techniques and various regulations are outside of the
 subject of this report. There are 27 channels in different bands, 
 2.4GHz band has been assigned with channel numbers 11 to 26.

  Power regulations apply depending on geographical region, the
 measures are transceiver output power and duty cycle. The global
 ISM band can be utilized at 100\% duty on approximately 10mW level.

 Two modulation techniques can be used in 2.4GHz band:
 \begin{description}
 \item[\emph{Offset-QPSK}]- Offset Quadrature Phase-Shift Keying
 \cite{links:wiki:qpsk}
 \item[\emph{DSSS}]- Direct-Sequence Spread Spectrum (was used in 2003 revision)
 \cite{links:wiki:dsss}
 \end{description}

 Alternative modulations techniques are defined in amendment
 \emph{P802.15.4a}, these include Ultra-Wide Band \emph{(UWB)}
 and Chirp Spread Spectrum \emph{(CSS)}.

 Common basic data rate is 250kbps and distance coverage is from 10m
 to 100m, but higher limits can be achieved (up to 2Mbps).

\subsubsection{Media Access Control (MAC)}

\begin{flushright} \small{
\emph{Source: "ZigBee Wireless Networks and Transceivers" \\
	by Shahin Farahani (2008) \cite{b:zigbee}}}
	\end{flushright}

  "The MAC provides the interface between the PHY
 and the next higher layer above the MAC."

% {This layer is above physical and therefore it controls what is being
% transmitted and how it is done.
% Some of the most important concepts are:
%	\begin{itemize}
% 		\item Super Frames and Timeslots\\
%		\small{\emph{These provide mechanism for real-time data transmission.}
%		\item Encryption\\
%		\small{\emph{Most of commercially available devices use 128-bit AES}
%		\dots Beacon Frames\\
%		\small{\emph{Essentially beacons can be seen as advertisement messages.}

 "The IEEE 802.15.4 defines four MAC frame structures:
\begin{itemize}
       \item \texttt{Beacon Frame}
	\small{\emph{ --- used by a coordinator to transmit beacons.\\
       			The beacons are used to synchronize the clock\\
			of all the devices within the same network.}}
       \item \texttt{Data Frame}
	\small{\emph{ --- used to transmit data.}}

       \item \texttt{Acknowledge Frame}
	\small{\emph{ --- used to acknowledge the successful reception of a packet.}}
       \item \texttt{MAC Command Frame}
	\small{\emph{ --- are used for commands such as requesting the data\\
			  and association or disassociation with a network.}}

\end{itemize}

It is important to understand that frames from each network layer are
encapsulated into each other, i.e. the MAC frame are encapsulated into
PHY frames on transmission and on reception these data structure is
being decoupled.


\subsubsection{\WPAN\ Classification: Nodes and Topologies}

  The \WPAN\ device hierarchy is defined in terms of \emph{full-}
  and \emph{reduced-function} devices (FFD and RFD for short).
  
  \begin{itemize}
 	\item Routers (FFD) 
	\begin{itemize}
		\item Network Coordinator
		\item Branch Coordinator
		\item Border Router
	\end{itemize}
	\item Participant Clients (RFD or FFD)
  \end{itemize}

  \WPAN\ Topologies are:
  \begin{itemize}
  	\item Point-to-Point
	\item Star
	\item Mesh
  \end{itemize}

\subsubsection{Feature Outline} \label{sec:P802154}
\emph{Source: IEEE 802.15 Task Group 4 Website}\cite{links:ieee:802:15:4}.

\begin{itemize}
	\item Data rates of 250 kbps, 40 kbps, and 20 kbps.

	\item Two addressing modes: 16-bit short and 64-bit.

	\item Support for critical latency devices, such as joysticks.

	\item \emph{Carrier Sense Multiple Access with Collision
		Avoidance \\(CSMA-CA)} channel access mode \cite{links:wiki:csma}.

	\item Automatic network establishment by the coordinator.

	\item Fully hand-shaked protocol for transfer reliability.

	\item Power management to ensure low power consumption.

\end{itemize}

\subsection{Conclusions}

  It has been found that some of the concepts defined in the \emph{IEEE}
 paper are of very little practical use. For example, the hierarchy
 specification is only used in \emph{ZigBee} and does not apply to
 \emph{6loWPAN} and the entire MAC layer specification is disregarded
 by the \emph{6loWPAN} user community and many papers where presented
 which propose various improvements, for example \cite{paper:amac}
 (One most highly regarded papers presented at the ACM conference
 on Embedded Networked Sensor Systems in November 2010).

 It does appear that Media Access Control for \WPAN\ as well as IP
 Routing Over Low-power and Lossy networks (ROLL) \cite{ietf:draft:roll}
 are subject to very extensive research at present, as well as time
 synchronisation \cite{paper:ts4,paper:ts2,paper:ts3,paper:ts1,
 Lenzen2010Clock, Lenzen2009Optimal, Sommer2009Gradient, Sommer2008Symmetric}.
 Many papers appeared on this subject and it is of greatest concert
 to the music instruments application, however it was found too complex
 to cover in the course of this project.



\subsection{Higher Layers and the Application Layer}

  The greatest benefit that the OSI model gives, is that any layer
 can be re-implemented without any changes to other layers above or
 below. Also translation between different implementations would a
 trivial task. \emph{6loWPAN} allows for connectivity of low-power
 wireless nodes to the Internet, enabling the future paradigm of
 \emph{"the Internet of Things"}. This has been developed and
 popularised by an alliance of Internet Protocol for Smart Objects
 (IPSO) \cite{links:ipso:homepage}. \emph{6loWPAN} takes care of
 several aspects such as compute resources limits and address space
 requirements due to large amount of participant nodes by providing
 \emph{IPv6} addressing with additional features such as \emph{header
 compression}, reduced functionality as well as other miscellaneous
 enhancements \cite{pubs:ipso:wp3, pubs:ipso:wp1}.

  By enabling Internet connectivity for the sensor nodes, the application
 developers are presented with lesser challenge, because existing algorithms,
 packages and libraries can be used. Care still needs to be taken due
 to limitations such as bandwidth and data structure complexity.

 For example, a Representational State Transfer Application Programming
 Interface (\emph{RESTful API}) \cite{links:wiki:rest} would make a
 system much simpler to integrate into existing environments,
 there won't be a need to develop tools for any particular computer platform.
 \emph{RESTful} approach brings a cross-platform compatibility out-of-the-box
 by utilising web-client software with very little of extra work to be done.
 \emph{RESTful API} is largely used in modern web-application services
 (also known as "the cloud"), it uses human readable URL strings for
 addressing functions on remote servers via HTTP protocol (the handling
 of HTML or XML response is optional and not applicable here, due to above
 mentioned data structure complexity issue).

 Another feature of \emph{IPv6} that is very important for complex wireless
 networks, is \emph{neighborhood discovery}, which eliminates any need
 for what DHCP served in \emph{IPv4}. Regarding the scalability,
 it is true that there is no particular use in this project for 128-bit
 address space that \emph{IPv6} provides. Nevertheless, this network
 may need to extend to a larger number of nodes if a orchestra of these
 is going to be built, also including various stage automation and
 safety sensors all in one area.


\subsection{Alternatives and Concerns}

  It is very appropriate to discuss a few alternative approaches which
 could be taken to implement a radio network specific to the application
 of interactive objects for live performance.

  An alternative to using \WPAN\ radio would be a different, rather
 simpler protocol and, perhaps, in a different frequency band.
  The reason for this is obviously because of application-specific
 concerns. The data transmission for musical performance is constrained
 to very strict real-time latency measures. In other words, when a
 musician is playing on stage the audience will perceive the music as
 very odd if the latency is too high. It would be even worse if several
 participants exhibit random latency, while not playing an improvised
 piece of music.

  Using a radio protocol which is in its own \emph{clear} frequency
 band and uses a minimal communication abstraction stack, should be
 more appropriate for latency optimisation. However, the cross-platform
 networking capabilities would be lost. That might lead to incompatibilities
 with market trends as well as radio band regulations.
  Since \emph{IEEE 802.15.4} is the current global technology trend and
 the 2.4GHz ISM band is accepted world-wide, the alternatives are not
 quite appropriate.

  There are a few solutions for \emph{non-P802.15.4} 2.4GHz ISM radio,
 but these would impose a vendor locking for the transceiver interfaces
 to such manufacturers as \emph{Nordic Semiconductors} \cite{links:nordic:rf2400}
 or \emph{Hope Microelectronics} \cite{links:hoperf:rf2400}.
 There might be even model-specific incompatibilities, while 
 it is quite certain that all \emph{P802.15.4} transceiver
 chips will be compatible in the future.

\break

\section{Study of Earlier Work \\and Commercial Solutions}

  Earlier works in the field of wireless performance devices (i.e. music instruments and
  devices to assist dance and theater performance) were studied. A number of links to
  other sites and publication are listed in the WMI website sections "Links" \cite{wmi:wiki:links}
  and "Bibliography" \cite{wmi:wiki:refs}. Below only the most relevant information is discussed.

\subsection{Publications}

  One publication by Paulo Jorge Bartolomeu (Univ. of Aveiro, 2005)
 \cite{pub:bartolomeu2005} which has also been submitted to the IEEE,
 evaluates the use of \emph{BlueTooth} for a wireless MIDI network.
 Paulo Bartolomeu in his dissertation considered either \emph{ZigBee} or
 \emph{UWB} as a possible alternative, however in 2005 these technologies
 still were immature and expensive. A number of interesting results and
 figures are provided, proving that \emph{BlueTooth} exhibits a definite
 byte latency of, on average, 20ms and a maximum of 100ms, these figures
 were reduced with improved network configuration to an average byte latency
 below 10ms. A quite interesting "command aggregation algorithm" had also
 been proposed in this paper.

 A surprisingly small amount of white papers have been found on this subject, 
 which indeed makes this study more interesting. Though, there are various
 articles available on-line, these are listed on the WMI website. There is
 no significant information to discuss in regards to those articles
 \cite{links:misc:xbeemidi}.
 
 A few short reports demonstrating \emph{BlueTooth} implementations were
 found on CCRMA (Centre for Research in Music and Acoustics at UC Stanford)
 website \cite{links:ccrma}, these appear to be homework reports by CCRMA
 students \cite{pub:ccrma:muggling, pub:ccrma:brbi}. Other related articles
 are listed in the bibliography \cite{pub:ccrma:sm, pub:ccrma:bb} appeared
 in 2005 and only one recent publication describing another \emph{BlueTooth}
 controller for \emph{PureData} \cite{links:wiki:pd} written by a researcher
 from Plymouth University \cite{pub:plym:go}.


\subsection{Commercial Wireless MIDI Products}

  Currently there are four commercial solution available from different
 manufacturers. It is most likely that the radio communication protocol
 used by these devices is unique to each particular manufacturer, since
 no information on compatibility is provided.

\emph{
\begin{itemize}
 	\item Kenton MidiStream \cite{links:kenton:midistream}
	\item M-Audio MidAir \cite{links:maudio:midair}
	\item CME WIDI \cite{links:cme:widi}
	\item PatchMan MidiJet \cite{links:pmm:midijet}
\end{itemize}}

 The \emph{Kenton} devices operate in the two sub-gigahertz ranges, 869MHz
 and 433Mhz, while the other three manufactures use the 2.4GHz ISM band.
 It may be desired to evaluate these devices before proposing any solution
 of a commercial nature; there is no particular interest to obtain any
 of these product otherwise.


\subsubsection{Other Related Hardware}

 The \emph{MiniBee} project by the Canadian group of artists, \emph{Sense/Stage}
 \cite{links:sensestage:homepage}, is of particular interest, it utilizes
 high-level frameworks such as \emph{Pure Data}, \emph{Max MSP} and
 \emph{Super Collider} \cite{links:sensestage:host}. The \emph{MiniBee} hardware
 \cite{links:sensestage:node} is an \emph{Atmel AVR} 8-bit MCU and \emph{Digi
 Xbee} transceiver module, which is not the most optimal design solution.
 Again, \emph{ZigBee} is a patented technology and is not appropriate for an
 open-source project. Also, the \emph{Xbee} module itself is already using an
 MCU (\Chip{HCS08}) to run its software stack \cite{links:xbee:wiki:prog}.

 
 In the area of sensor products, there is one interesting company called
 \emph{Phidgets, Inc.} \cite{links:phidgets:homepage}, it manufactures rather
 wide variety of sensor devices, ranging from base single-board computer (SBC)
 to sensor and actuators sharing same type of 3-pin connector. The SBC board
 runs custom-built \emph{Linux OS} on \emph{ARM9} 266MHz CPU with 64MB of
 SDRAM and 64MB of flash memory. \emph{Phidgets SBC}
 \cite{links:phidgets:products:sbc} has USB and 100MB Ethernet peripherals as
 well as a number of screw terminals and 3-pin connectors for sensors.
 \emph{Phidgets, Inc.} supplies numerous easy-to-use sensor and actuator
 devices. The cost of their products is quite high, but the reason of
 this is the target market. Buyers of these easy-to-use devices are not
 professional electronics engineers and are ready to pay these prices.
 Their website also has source code for reference in a wide selection of
 programming languages and frameworks.
 
 However the top product in the hobbyist marker remains \emph{Arduino}
 \cite{links:arduino:homepage}, of course the low price factor is key to its
 popularity. A huge number of semi-clone products emerged, all featuring
 various additions that some users may find more useful. The most recent
 interesting semi-clone of \emph{the Arduino} is \emph{Seeeduino Film}
 \cite{links:seeed:products:film}, made by \emph{Seeed Studio} in China.
 It appears to be the only product of its class that uses flexible film
 instead of regular circuit board. For example, this could be quite
 useful for integration into a music instrument or attached on performer's
 body in combination with capacitive flex sensors. Even the above mentioned
 \emph{MiniBee} in its design is also a semi-clone of \emph{the Arduino},
 in order to support convenient \emph{Arduino C++ Library}
 \cite{links:arduino:library}, which is in no doubt appreciated by the
 target audience.
