\section{Outlook Trial for the TinyOS} \label{sec:TINYOS}

  It has been discovered that a few members of \TinyOS\ community
 have recently worked on porting \TinyOS\ to run on \Chip{ATmega128RFA1}
 \cite{links:tinyos:rfa1:p1,links:tinyos:rfa1:p2}. Comparing to \Contiki,
 \TinyOS\ has rather superior abstraction layer that provides a seamless
 cross-platform integration methods [\TEP{2}, \TEP{108}, \TEP{109}, \TEP{131}].

 
  Various internals of \TinyOS\ had been studied, however it is 
 certainly a very broad area to be described here. It is best described
 in the \emph{"TinyOS Programming"} book by David Gay and Philip Levis
 (it is available in print as well as a web-edition \cite{levis2009tinyos}).
 Two authors of this book are lead developer of the \TinyOS\ project
 and are currently working at the University of California, Berkley
 and Standford.


  \TinyOS\ currently had been ported to a variety of 8-, 16- and
 32-bit microcontrollers, most of the peripherals (such as I2C and SPI)
 have unified high-level access mechanisms, unlike in \Contiki\footnote{
 \emph{Most of peripheral drivers in Contiki OS are rather specific to
 a certain platform and only some particular drives share a similar
 interface.}}. Most of development documentation is provided in the
 the format of \emph{"TinyOS Enhancement Proposal"} (known as
 \emph{TEP}\footnote{\emph{These documents are commonly referred to as
 TEP\texttt{n}, where \texttt{n} is a number. \TEP{1} defines the format
 of the TEP documents.}}), there is also documentation generated from
 the source code comments (\emph{nescdoc}) as well as various on-line
 resources \cite{tinyos:docs} and the textbook mentioned above. It is
 noticeable that \TinyOS\ documentation is more extensive then what is
 currently present for the \ContikiOS.


\subsection{Programming with TinyOS:\\Compilers and Abstractions}

  The greatest achievement of \emph{TinyOS} is its specialised language,
 benefits of which had been overlooked at the earlier stage of research
 for this project. \emph{"Network Embedded Systems C" (nesC)} is a
 C-based language, it provides abstractions for components, interfaces
 and configurations. It is not an object-oriented language, though
 it is rather component and interface oriented. As it was defined in
 the previous section that \Contiki\ can be seen as a collection of
 processes and events are communicated between those processes, then
 \TinyOS\ can be defined as a collection of components and interfaces
 which are wired together in configuration abstraction layer, tasks,
 events and the data are communicated between the components via the
 interfaces.

  The control structures and all of C constructs are still valid in
 \emph{nesC}. This can be seen as if \emph{nesC}-specific statements
 are replaced by appropriate C source code that defines required behaviour.
 Nevertheless, this is only a brief description of the relation between
 \emph{nesC} and its ancestor. Just as if C could be defined by stating
 a ratio of lines of code in C versus lines of assembly code.

 It is important to note that current implementation of the compiler
 translates higher-level nesC code into C language. The resulting
 programs are specifically suited for embedded devices. Direct
 compilation would be possible and, if desired, there is an interesting
 platform to look into. \emph{LLVM} ("Low-level Virtual Machine") is a
 new generation compiler technology \cite{links:wiki:llvm}. It is know
 that using \emph{LLVM} and its family member \emph{clang} implementation
 of a new compiler for a C-like language would be simplified (comparing
 to more traditional techniques). One good example of industrial grade
 compiler based on \emph{LLVM} is the \emph{XC} toolchain for \emph{XS-1}
 devices from a UK semiconductor company \emph{XMOS} \cite{links:xmos:tools}.

 Nevertheless, currently \emph{nesC} compiler is known to be fully
 functional and it's task is not as complex as it may seem. Programming
 in any language is always done by applying code patterns, general
 to some degree, to accomplish desired behaviour of a program.
 The purpose of \emph{nesC} abstraction layer can be seen as making
 the details of complex coding - with which high degree of modularity
 can be achieved - rather hidden away from the programmer.


\subsection{Network Protocols in TinyOS}
 
  \TinyOS\ has become widely adopted and there are many researchers
 who contributed significant work in different application areas, one
 such area of great interest to WMI project is sensor node timer
 synchronisation protocols. A number of papers are available and the
 mainstream repository of \emph{TinyOS} source code already contains
 implementations for some of those protocols.

 \underline{\emph{( A few details with references needed )}}

  Further study and experimentation in this area are required to design
 a system which could cope with real-time constraints of stage control
 applications. It is important to note that \emph{P802.15.4} has addressed
 some real-time application requirements, however the status of software
 support for these features of \emph{P802.15.4} is currently uncertain.
 \underline{\emph{( Perhaps mention NanoRX ? )}}

  After several aspects of \emph{TinyOS} were studied, it was clear
 that its current implementation of \emph{6loWPAN} is fully compatible
 with \emph{Contiki} and it had been desired to prove this in practice,
 but this has not been achieved at the time of writing of this paper.
  
  A problem exists however, there was no fully working IP-enabled driver
 for \emph{ATmega128RFA1} transceiver. The IP layers are provided by
 \emph{BLIP} stack (this stands for Berkley Lightweight IP), the stack
 is currently undergoing major development. Details on what is required
 are available on \emph{TinyOS} wiki page \cite{tinyos:wiki:blip20},
 implementing it in \emph{nesC} was not as trivial due to the learning
 curve. Therefore this had be postponed.

  This problem requires further explanation. \emph{TinyOS} has been
 developed since 2001, while \emph{Contiki} first dates back to two
 years later - 2003\footnote{\emph{No exact information has been
 found, these are the earliest dates which appear in the source code.}}. 
 In the early days of WSN research \emph{P802.15.4} and \emph{ZigBee}
 where still emerging, the \emph{6loWPAN} RFCs appeared at IETF more
 recently\footnote{\emph{The first revision of P802.15.4 was released
 in 2003 (current revision is from 2006) and first draft of 6loWPAN
 is dated April 2007.}}. \emph{TinyOS} has originally used its own
 protocol called \emph{ActiveMessage}.

  Currently \TinyOS\ incorporates \emph{BLIP} stack, which initially
 was released by UC Berkley Wireless Embedded Systems research group
 in the summer 2008, know as \emph{bl6lowpan} at that time \cite{ucb:webs:blip}.
 As mentioned below, major code changes are currently still in progress.
 The driver code which fully implements all new features of the \emph{BLIP}
 stack (including point-to-point tunnelling for simple UART wired
 connectivity) is only for the most popular \Chip{CC2420} radio chips.

\subsection{Hardware Resources}

  It is difficult to compare the two sets of hardware platforms which
 \TinyOS\ and \Contiki\ had been ported to, since the status of support
 for some platforms is uncertain. It is probably most appropriate to
 say that these two sets are almost equal, however, some platforms
 which appear in \Contiki\ source code, had not been ported to \TinyOS\
 and vise versa. Two devices of interest to WMI project are \RFA\ and
 \MCX. The issue with the first chip is already described above.
 The second chip has not been ported to \TinyOS\ yet. This should not
 be too difficult task to achieve considering that driver implementation
 could be derived from \Contiki\ code. There also very noticeable
 progress towards \emph{ARM Cortex} support\footnote{\emph{The homepage
 of TinyOS Cortex project can be found at:
 \URL{http://code.google.com/p/tinyos-cortex/}
 The code structure was found to be well organised and most probably
 can be relied upon. However it has not been desired to take this
 challenge in the near future.}}, hence there is base for \emph{ARM}
 devices (i.e. toolchain support and core architecture code).
 

  There is a large repository of board design files featuring so called
 \emph{Epic Mote} platform, that is base around \Chip{CC2420} and the
 \Chip{MSP430} \cite{epic:homepage}. The most interesting designed by
 Prabal Duta of UC Berkley \cite{duta:homepage} is shown in Figure
 \ref{fig:mote:quanto}. The reason why it is so interesting is because
 it features \emph{Digi Connect-ME} microprocessor system block that is
 of standard RJ45 form-factor \cite{digi:connect-me}. This tiny device
 has 55MHz \Chip{ARM7TDMI} CPU, 4MB of flash and 8MB of SDRAM as well as
 hardware cryptographic unit and is capable to run Linux kernel and a
 small subset of standard software in userspace, hence the advantage of
 scripting languages can be utilised. Currently it is the smallest
 device available on the market featuring such capabilities and
 the unique form-factor. It is a very appropriate device to use for
 wireless network edge gateway system, for example a hardware crypto
 unit could accelerate secure tunneling of the network traffic. It
 could be considered as a better solution instead of using a large
 Ethernet chip, unless there would be a chip with a combination of
 Ethernet and RF hardware in a single package.

% INCLUDE THIS IMAGE:
% http://www.cs.berkeley.edu/~prabal/projects/epic/epic-quanto.jpg


\subsection{Conclusion}

  Several steps were made in attempt to bring up \emph{TinyOS} on the
 hardware chosen for this project earlier, though it appeared rather
 unmanageable within the give time frame. This is still a subject of
 interest and shall be looked into at a later time. An evidence of
 work carried out with \TinyOS\ code can be viewed on WMI website
 issue tracker.

\TrackerList
 \IssueX{30} Porting Radio Interface to BLIP 2.0
\TrackerEnd

%\URL{https://github.com/errordeveloper/tinyos-wmi/commits/wmi-work?author=errordeveloper}
