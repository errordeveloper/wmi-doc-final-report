\section{Introducing the Contiki Operating System}\label{sec:CONTIKI}

  \Contiki\ operating system can be seen as collection of processes.
 In the same way as the UNIX operating system is seen as collection of
 files and processes. However, files are not a necessary component, in
 fact \emph{Coffee File System (CFS)} is only needed for some specific
 applications which require a non-volatile storage abstraction. 
  The inter-process communication (IPC) in UNIX has not been implement
 in the very early releases, although Contiki processes wouldn't work
 without its event-driven IPC mechanism\footnote{\emph{Though it is very
 primitive comparing to the usual meaning of IPC as an acronym.}}, which also
 drives the core scheduler. There is no notion of users in \Contiki\,
 however, and at this point comparison with UNIX becomes apparently
 impossible. Nevertheless, such a high-contrast comparison will hopefully
 deliver a good picture of what \ContikiOS\ is and what it is not.

% check the date in the slides from WSN course

  The lead developer of \Contiki\, Adam Dunkels of Swedish Institute of
 Computer Science (SICS), has revolutionised embedded system world by
 implementing \emph{lwIP} in 2000 and later, in 2002, even further optimised
 \emph{uIP} stack. It is known that until \emph{lwIP}, no complete IP stack
 existed which could run on a microcontroller device \cite{dunkels03full}.

  The key technique utilised in implementing \Contiki\ and the \emph{uIP}
 stack is based on a very obscure behaviour which the C language's
 \texttt{switch/case} statement exhibits in certain context. Dunkels
 wrote a few papers \cite{dunkels05protothreads,dunkels06protothreads}
 describing how it had been achieved, and summarises it in his doctoral
 thesis \cite{dunkels07programming}. The threading technique has been
 named \emph{protothreads (PT)}. Processes in \Contiki\ are designed as
 an extension to the \emph{protothreads}. Another extension is called
 the \emph{protosockets}, as the name suggest it provides the abstraction
 of network sockets.

  Most of functionality of all three key elements is delivered by the use
 of C pre-processor macros. The \emph{protothreads} are platform
 independent\footnote{\emph{It is not completely true for Contiki, there
 are few additions which utilise facilities that some hardware platforms provide.}}
 and can be integrated into any C program by including only one header file.
 \emph{Protothreads} can be used with virtually any C compiler toolchain and
 as described in \cite{dunkels06protothreads}, the overhead is rather
 insignificant, considering the powerful functionality; the only limitation
 is due to the use of \emph{\texttt{switch/case}} statements by the \emph{PT}
 macros, these control structures will lead to undefined behaviour of the
 code in the \emph{PT} context.

  \emph{Protosockets} are designed specifically for \emph{uIP} and therefore 
 cannot be included with just one header. The only limitation of the
 \emph{protosockets} is due to design decision, there is no UDP communication
 facility, only TCP is provided. The \emph{protosockets} greatly simplify the
 application code, since all necessary functions are provided, including
 handling of strings. Thought no packet flushing is currently possible, which
 may be desired in some application.

\subsection{Initial Development Phase}

  \Contiki\ source code includes a large set of stable drivers for various
 RCBs which were mentioned previously, such as \emph{Zolertia Z1}, \emph{Tmote
 Sky}, \emph{Atmel Raven} and \Chip{RF230}\emph{-based} boards, \emph{TI}
 \Chip{CC2430} and \emph{MSP430} as well as \emph{ARM Conrtex-MX} and many
 older devices, including \emph{Comondore 64} and \emph{Apple II} computers.

 Details regarding the \Chip{MC1322x} port \cite{links:contiki:port:mc1322x}
 are to follow in \ref{sec:APPDEV}, since \Chip{MC1322x} was chosen as the
 target platfrom for this project. However, the \RFA\ platform had been the
 original preference. Major work during the inital phase had been done to
 port the existing driver code for \Chip{RF230} transceiver to take the
 advantage of the single chip device. This work had been described in the
 interim report \cite{wmi:irep}.

