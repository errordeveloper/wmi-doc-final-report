\section{Working with Contiki}
  
  After an appropriate compiler toolchain and debugger packages for \MCX\
 had been installed on the development host, a few steps were taken to
 simplify the work-flow in \Contiki\ application development environment.
 It may be noted here, that an integrated development environment (IDE)
 could be used and some programmers do prefer to use an IDE, such as
 \emph{Eclipse} \cite{links:contiki:eclipse, links:mc1322x:eclipse},
 nevertheless the command line tools are known to be the most efficient approach.
 It should be noted that this section is rather brief description of what
 has been done and was not intended to provide a detailed guidance on how
 to reproduce the results.

  Apart from the revision control tools\footnote{\emph{This project used
 git system, however the details on how that has been done are considered
 to be irrelevant to the subject of this report.}}, a text editor and the
 GCC toolchain for ARM \cite{links:mc1322x:gcc}, there are three essential
 command-line tools which were utilised during the development process:

\begin{description}
	\item [\MAN{gdb}] - GNU Debugger (source-level) \cite{docs:gdb:manual}
	\item [\MANX{1}{make}] - GNU make program \cite{docs:make:manual}
	\item [\emph{\texttt{OpenOCD}}] - On-chip Debugger \cite{links:mc1322x:ocd}
\end{description}


  To enhance the work-flow \emph{"Makefiles"}\footnote{\emph{These are
 the file which specify a set of rules for the make program on how to
 compile the source code and also perform administrative tasks and run
 debugger or other tools.}} were amended throughout the development
 process. Generally there is one \emph{Makefile} in each subdirectory
 of the source tree, thought most of these inherit rules specified in
 the main \emph{Makefile} (in \Contiki\ there are two of these - one
 in the root directory and one for each processor architecture).

 Most of the changes were made in \Blame{cpu/mc1322x/Makefile.mc1322x}
 to provide a few of shorter commands for setting-up the debugger
 and sending the program to run on a development board\footnote{
 \emph{For example, to set-up the WPAN router on the Freescale board - run
 \texttt{`make.f1 router'} and to compile 'example.c', load and print
 serial output on the console for the Econotag  -
 \texttt{`make.e1 example.load-print'}; in case if 'example.c' does
 not behave as expected - run \texttt{`make.e1 example.ocd-screen'}.}}
 
\section{Application Prerequisites}

  The first step in application development was to add a driver for
 the second UART (UART2) port. This has been done by copying an
 existing driver code for UART1, though enhancements were required
 at a later stage. The history of changes to the code can be viewed
 in at the repository by utilising the commit log filter. The files
 shown here had been modified.

 \begin{itemize}
 \item \Contrib{cpu/mc1322x/lib/include/uart1.h} \\
 	\emph{normal and weak prototypes, register pointers and macros}
 \item \Contrib{cpu/mc1322x/lib/uart2.c} \\
 	\emph{driver interrupt handlers}
 \item \Contrib{cpu/mc1322x/src/default\_lowlevel.h} \\
	\emph{prototypes}
 \item \Contrib{cpu/mc1322x/src/default\_lowlevel.c} \\
	\emph{initialisation functions}
 \end{itemize}

  A set of macros was defined to aid the application code, though
 most of those macros were used only for initial debugging.

  Soon, it has been understood that the original implementation
 was missing handler functions for the UART interrupts. With
 the help of communication on the \Contiki\ mailing list, the
 appropriate methods were realised. One of the key techniques
 was to use \emph{"weak"} function linking attribute. Two macros
 named \texttt{U2\_RXI\_POLLHANDLER()} and \texttt{U2\_TXI\_CALL()}
 are appearing in the program listing \ref{code:talker}. These
 macros define the function which is called by the interrupt
 routine in \Blob{cpu/mc1322x/lib/uart2.c}\footnote{\emph{For
 the definition of the macros, see: \begin{itemize}
	\item \Blob{cpu/mc1322x/lib/include/uart1.h}
	\item \Blob{projects/wmi/mc1322x/midi/uart2-midi.h}
 	\end{itemize}}}.

\TrackerList\em
\IssueX{27} Compiling various example programs
\IssueX{31} Wireless transmission of MIDI is proven working
\IssueX{32} Requirenments for the MIDI UART driver
\TrackerEnd

\section{Designing the MIDI Talker Program}

  A variety of constructs had been tried for handling the
 FIFO  buffer into the \emph{uIP} packet buffer. This was
 rather challenging and many complicated bugs appeared
 throughout out the process. The JTAG debugger had been
 utilised a number of times to find these bugs.

  The macro named \texttt{PSOCK\_GENERATOR\_SEND()} had
 been found as the most direct way to transfer the data
 from FIFO into the packet buffer. This macro is defined
 and documented in \Blob{core/net/psock.h}, it is a part
 of \Contiki\ \emph{protosockets} library.

\subsection{Protosockets}

  This section gives more detail on \emph{protosockets},
 which were introduced in \ref{sec:CONTIKI}.

  The \emph{protosockets}  is a set of C macros (and just
 a few C functions) defined and documented in:
 \begin{itemize}	
	\item\Blob{core/net/psock.h}
	\item\Blob{core/net/psock.c}
 \end{itemize}

 These are indeed simple methods for a TCP application to
 utilise. Nevertheless, it is commonly known that UDP
 communication is most appropriate for a type of program
 that is discussed here, the results included included in
 this chapter will show that TCP can be used for a simple
 program like this. If UDP was to be utilised, additional
 algorithms for retransmission and stream data ordering
 would need to be developed. As discussed in section
 \ref{sec:MIDI}, it should be rather more appropriate to
 develop a library for \emph{RTP-MIDI} instead of a simple
 UDP application. There is a great benefit to use TCP at
 this stage, as it provides the guarantees of data being
 transmitted correctly and in a sequential order on the
 socket level.

 For detailed documentation of the function which will
 appear in the program listing, see the \emph{Contiki}
 \cite{contik:docs} and the \emph{uIP} \cite{contik:uip}
 manuals. It would be misleading to included extracts
 of documentation here, since lower-level functions
 and algorithms would need to be listed.
