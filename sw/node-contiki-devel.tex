\section{Working with Contiki}
  
  After an appropriate compiler toolchain and debugger packages for \MCX\
 had been installed on the development host, a few steps were taken to
 simplify the work-flow in \Contiki\ application development environment.
 It may be noted here, that an integrated development environment (IDE)
 could be used and some programmers do prefer to use an IDE, such as
 \emph{Eclipse} \cite{links:contiki:wiki:eclipse}, nevertheless the
 command line tools are known to be the most efficient approach.
 It should be noted that this section is rather brief description of what
 has been done and was not intended to provide a detailed guidance on how
 to reproduce the results.

  Apart from the revision control tools\footnote{\emph{This project used
 git system, however the details on how that has been done are considered
 to be irrelevant to the subject of this report}} and the GCC toolchain
 for ARM \cite{links:mc1322x:gcc}, there are four key command-line tools
 which were utilised during the development process.

\begin{description}
	\item [\MAN{gdb}] - GNU Debugger (source-level) \cite{docs:gdb:manual}
	\item [\emph{\texttt{OpenOCD}}] - On-chip Debugger \cite{links:mc1322x:ocd}
	\item [\emph{\texttt{Vim}}] - a text editor
	\item [\MANX{1}{make}] - GNU make program \cite{docs:make:manual}
\end{description}

% http://mc1322x.devl.org/eclipse.md
% http://www.sics.se/contiki/wiki/index.php/Setting_up_Eclipse_for_Contiki_Development

% http://mc1322x.devl.org/openocd.html

% http://mc1322x.devl.org/toolchain.md

% http://www.gnu.org/software/make/manual/make.pdf
% http://www.gnuarm.com/pdf/gdb.pdf

  To enhance the work-flow \emph{"Makefiles"}\footnote{\emph{These are
 the file which specify a set of rules for the make program on how to
 compile the source code and also perform administrative tasks and run
 debugger or other tools.}} were amended throughout the development
 process. Generally there is one \emph{Makefile} in each subdirectory
 of the source tree, thought most of these inherit rules specified in
 the main \emph{Makefile} (in \Contiki\ there are two of these - one
 at in the root directory and one for each processor architecture).

 Most of the changes were made in \Blame{cpu/mc1322x/Makefile.mc1322x}
 to provide a variety of short command for connecting the debugger
 and sending the program to run on a development board\footnote{
 \emph{For example, to set-up the WPAN router on the Freescale board - run
 \texttt{`make.f1 router'} and to compile 'example.c', load and print
 serial output on the console for the Econotag  -
 \texttt{`make.e1 example.load-print'}; in case if 'example.c' does
 not behave as expected - run \texttt{`make.e1 example.ocd-screen'}.}}
 
\section{Application Prerequisites}

  The first step in application development was to add a driver for
 the second UART (UART2) port. This has been done by copying an
 existing driver code for UART1, though enhancements were required
 at a later stage. The history of changes to the code can be viewed
 in at the repository by utilising the commit log filter. The files
 shown here had been modified.

 \begin{itemize}
 \item \Contrib{cpu/mc1322x/lib/include/uart1.h} \\
 	normal and weak prototypes, register pointers and macros
 \item \Contrib{cpu/mc1322x/lib/uart2.c} \\
 	driver interrupt handlers
 \item \Contrib{cpu/mc1322x/src/default\_lowlevel.h} \\
	prototypes
 \item \Contrib{cpu/mc1322x/src/default\_lowlevel.c} \\
	initialisation functions
 \end{itemize}

