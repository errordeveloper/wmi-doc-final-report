\section{Implementation Decisions} \label{sec:MIDI}

  Two programs where written to test transmission of MIDI. The concept of
 server/client communication could be applied, nevertheless it had been
 found more appropriate to name the parties as a \emph{Talker} and a
 \emph{Listener}. At the time of writing of this report the \emph{Talker}
 program had been fully tested and debugged. The second program can be
 considered only as a prototype and does require further work.

  The \emph{protosocket} macros had been utilised for the implementation,
 hence the communication protocol imposed by \emph{protosocket} was TCP.
 There are various improvements that are still to be considered.
 The specification of TCP includes a notion for \emph{urgent data}
 (\RFC{0793}), however this feature has received a very limited adoption
 and is known to be handled differently by various implementations of TCP
 \cite{ietf:draft:tcpm:urgdata}. It has been found that the \emph{uIP}
 stack does implement some mechanisms for the \emph{urgent data}, however
 the documentation on how it can be applied in \Contiki\ application code,
 could not be found. 


\subsection{UDP Multicasting}

  A variety of packages exist which utilise UDP multicasting for streaming
 of the MIDI data on LAN. These include \emph{ipMIDI} \cite{links:ipmidi} for
 Microsoft Windows and Apple Mac OS X platforms as well as compatible
 packages for Linux - \emph{multimidicast} \cite{links:multimidicast} and
 \emph{qmidinet} \cite{links:qmidinet}. This two Linux packages had been
 tested with current implementation of the \emph{Talker} program and have
 shown a suitable performance\footnote{\emph{This has been achived by
 piping the data from the Talker node to a multicast client on the host and
 then receiving this at another host on the same LAN. This was done with
 a few standard network testing utilities available on Linux, however for
 a deployment, a specialised program needs to be developed.}}. The multicasting
 approach is considered to be quite appropriate for streaming of MIDI signals
 on IP networks, since no configuration is required, i.e. none of the
 participants have to set-up connection between each other and only need
 to send data to the multicast IP address\footnote{\emph{For compatibility
 with ipMIDI, the multicast IP
 address should be 225.0.0.37 and the port numbers from 21928 and above.}}.
 However the \emph{ipMIDI} is not a formal protocol, it simply streams the
 MIDI signals to a UDP socket, i.e. there are no any error correcting
 mechanisms or hand-shaking.

  There had been no implmentation of UDP multicast in \emph{uIP} and there 
 are certain difficulties associated with this, since \emph{P802.15.4}
 only defines broadcasting and doesn't define multicasting. However,
 with the help of \Contiki\ mailing list, a patch had been received from
 Anuj Sehgal and tested. There had been no feasible test results of using
 this implementation.

 \TrackerList\em
 \IssueX{33} Testing UDP Multicasting (with uIP)
 \TrackerEnd

  One simple test has also been made with UDP broadcast. However, this
  is not appropriate to use, since that would be considered as flooding
  of the entire network. More suitable use of broadcasting was shown in
  the  previous section (figures \ref{sm:advlistener} \& \ref{sm:advtalker}).

%\pagebreak

\subsection{MIDI payload for Real-time Transport Protocol}

 Two relevant Internet standards had been published by the IETF in 2006:
 \begin{itemize}
  \item\emph{"RTP Payload Format for MIDI"} (\RFC{4695})
  \item\emph{"An Implementation Guide for RTP MIDI"} (\RFC{4696})
 \end{itemize}

 This two papers had been studied and considered as a possible solution
 for encapsulating MIDI data stream on a \emph{6loWPAN} networks, however
 the implementation would require a library for the \emph{Real-time
 Transport Protocol} as well a large set of algorithms to handle various
 aspects of the \emph{RFC4695}. It would be a very challenging task to
 design \emph{RTP-MIDI} library which could run on a microcontroller.

  One very important aspect that is defined in \RFC{4695}, is the
 \emph{recovery journal}. The journal has to be maintained by each
 participant and it is of a concern, whether the journal could be
 implemented on a microcontroller or it requires a larger amount
 of RAM. The function of the \emph{recovery journal} in \emph{RTP-MIDI}
 is to keep the history of events and, if a packet had been lost, prevent
 \emph{stuck notes} in the musical performance.
 
 This protocol is currently implemented as part of \emph{Apple CoreMidi}
 \cite{links:wiki:rtpmidi} and an open-source library exists for Linux
 \cite{links:rtpmidi}. However, there had been major adoption for the
 \emph{RTP-MIDI} in the field.

\subsection{MIDI Networking in Linux}

  On Linux, networking of MIDI clients can be implemented in user-space
 with \emph{ALSA ("Advanced Linux Sound Architecture")} utilities package
 \cite{links:linux:alsa}.
 The documentation on \emph{ALSA Sequencer Network} is limited to the
 comments in the source code of the \MAN{aseqnet} program. It is design
 using UDP and a specific data structure. The two programs mentioned in
 the previous section do not communicate with \MAN{aseqnet}, but use
 the \emph{ALSA Sequencer} interface directly.

  It would reasonable to implement the same protocol which \emph{ALSA
 Sequencer Network} is using, however this will be incompatible with
 other operating systems.

\section{Conclusion}

  The final decision had been to apply the simples data structure, i.e.
 \emph{raw MIDI bitstream}, since it is more important to prove the concept
 before progressing towards an implementation of a more advanced protocol.

