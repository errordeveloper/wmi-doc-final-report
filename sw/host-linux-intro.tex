\section{Building Embedded Linux OS}

\subsection{Background}

  Until quite recently, it had been rather more difficult to achieve
 the task of building (custom) embedded Linux system. Traditionally,
 the engineer who desired to do so, would need to follow instructions
 provided in the reference book \emph{"Linux from Scratch"}, commonly
 known as \emph{LFS} \cite{book:lfs}. This text provides details on
 how to utilise various tools for building an embedded Linux kernel
 and the file system from source, it discusses how to tweak various
 features at build time and configure appropriate runtime services.

\subsection{Build Utilities}
 
  More recently, a number of projects emerged, which do a great job
 of extending the flexibility by automating some of the simpler
 tasks, e.g. package dependency tracking and build version control
 (enabling the maintainer to revert to previous builds).
 One of the most outstanding and widely used projects in this areas
 is \emph{OpenEmbedded} \cite{links:oe} it has recently been adopted
 by a major software company \emph{Mentor Graphics}. \emph{Mentor
 Embedded Linux} extends \emph{OpenEmbedded} framework with a set
 of tools which are useful in a large-scale development project
 \cite{links:mentor:linux}.
 
  There are a few alternative approaches which lead to similar results,
 one may wish to use \emph{Gentoo Linux} meta-distribution, there
 is only one source of documentation - \emph{"Gentoo Linux Embedded
 Handbook"} \cite{links:gentoo:embedded}. Generally, \emph{Gentoo}
 framework (named {\emph{Portage}) has all necessary components
 which a system designer may need and there is no big difference
 between \emph{Gentoo} and \emph{OpenEmbedded}. Many internals
 are known to be very similar, tough \emph{Gentoo} doesn't provide
 some specialised tools which are provided in \emph{OpenEmbedded},
 hence \emph{Gentoo} has been originally designed for custom
 desktop and server systems. Major aspect which is different is
 that \emph{Gentoo} is rather bound to package releases, while
 \emph{OpenEmbedded} is more flexible when a development revision
 number has to be specified for a certain source code package.
 The conclusion is that \emph{OpenEmbedded} is most likely to be
 more convenient to use for an embedded target.
 
  A little different approach can be taken with \emph{BuildRoot}
 \cite{links:buildroot:homepage}, which is a build system for
 embedded Linux. Based around the same framework that was design
 for configuring the Linux kernel builds. This tool is a set of
 scripts controlled via a console menu. Although, it appears to
 have a simple to use front-end, the underlying configuration
 system is certainly behind the competition with \emph{OpenEmbedded}.

  A project which have a slightly different orientation is
 \emph{ScratchBox} \cite{links:sbox:homepage}. It provides
 a cross-compilation toolkit for application developers.
 It can be used to provide an software development kit (SDK)
 for a 3rd-party developer with appropriate toolchain and
 hardware emulator. However, \emph{OpenEmbedded} includes
 support for building SDK and \emph{ScratchBox} is rather
 limiting in various ways, i.e. it is a static distribution,
 rather then a build system.
 
  Another project in this area of interest is \emph{Linaro Foundation}
 \cite{links:linaro:homepage}, the initiative has been originated by
 London-based \emph{Canonical Limited} (the company behind now most
 popular Linux distribution - \emph{Ubuntu}).
 The purpose of \emph{Linaro} is to improve various issues related to
 embedded Linux and provide a higher grade platform for Android and
 other multimedia and consumer oriented distributions. \emph{Linaro} is
 still an emerging project and has not produced any significant output
 \cite{links:linaro:homepage}. It has specific orientation, in terms of
 hardware, being currently involved only with \emph{ARM} platforms,
 and in terms software, it is bound rather strictly to \emph{Ubuntu}
 methodology\footnote{\emph{That implies it is being directed by the
 founder company, Canonical.}}.

\section{Conclusion}

  It had been desired to build an embedded DSP host OS as one of
 additional project targets, however due to various\footnote{\emph{%
 Another aspect was due to hardware selection and purchasing issues.}}
 limitations, most importantly - the time frame, the project has not
 achieved this at the time of writing of the final report. The above
 chapter was included to provide the evidence of research in this area.

\section{Synthesis Software}

  Another target which has not been achieved, since it was dependant
 on the design of the DSP host, was to use visual DSP modeling package
 \emph{Pure Data} \cite{links:wiki:pd, links:sw:pd}. It would be a
 simple task and there is no particular concern of how to integrate
 the current implementation with \emph{Pure Data}\footnote{\emph{%
 The author has extensive experience of using this package}}. Another
 reason for disregarding this target, was that the emphasis of the
 entire project has changed towards various aspects of wireless
 sensor network implementation and audio synthesis is outside the
 general scope of this report.

