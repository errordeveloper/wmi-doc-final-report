\section{Overview of Embedded Operating Systems} \label{sec:RTOS}

		%% REWRITE !!

   A subject of extensive research for this project has been into the
  area of Real-Time Operating Systems for embedded devices, RTOS for short.
  This family of operating systems is not like any other general
  purpose OS family. The distinct feature that any RTOS has is the task
  management in constrained time-scale, the term \emph{real-time} is relatively
  abstract, it can be classified either as \emph{soft-} or \emph{hard-real-time}.
  The classification is also an abstract matter, some operating systems provide
  various approaches and it is up to the engineer to utilise it correctly for
  any given application. A modern RTOS should also provide a communication
  protocol stack and file storage drivers, support a set of platforms and
  a variety of peripheral devices.
 
 It is commonly known that RTOS is a rather expensive element in embedded design.
 This is because until very recently most commonly used RTOS implementations were
 provided only by specialised commercial vendors, brand names like \emph{Micrium},
 \emph{VxWorks} and perhaps \emph{QNX} are popular in specific application fields
 and there are also more obscure RTOS products offered at premium prices. Another
 more recent commercial RTOS is \emph{Nucleus}, currently owned by \emph{Mentor
 Graphics}. This repor does not focus on commercial products, hence these will
 not be refer to, more information can be found in \cite{links:wiki:rtos,
 links:wiki:rtos:list}.


 Currently the open sources community does present a number of very interesting
 RTOS projects, some of which are briefly introduced below. More details can be
 found on the WMI website \cite{wmi:wiki:rtos} (all the general project URLs are
 listed in that section of the website, and therefore omitted from the text below).
 It may be important to note that many of the project which are listed here are
 originating from major universities, some of which previously gave birth to very
 well know UNIX software and general innovations in the computer science.


 \subsubsection{RTEMS} \label{sec:rtos:rtems}

 The \emph{Real-Time Executive for Multiprocessor Systems} or \emph{RTEMS} is a
 full featured RTOS that supports a variety of open API and interface standards.
 It originates from US military and space exploration organisations. By definition
 \emph{RTEMS} is a highly scalable system, i.e. it is designed for multi-processor 
 deployment and is an interesting system to use for various possible applications,
 \emph{RTEMS} appears to be the oldest open-source RTOS. Unlike other systems below,
 the \emph{RTEMS} doesn't belong to sensor network OS sub-family on which this
 project had been focused. 


\subsubsection{TinyOS} \label{sec:rtos:tinyos}

 \emph{TinyOS} originates from the University of California, Berkeley and
 Stanford and the Intel Research Laboratory at Berkeley \cite{links:wiki:tinyos,
 links:tinyos:webs, links:tinyos:homepage}. The approach of \emph{TinyOS}
 is very different from all other systems, it uses a higher level abstraction
 language based on \emph{C}, which is called \emph{nesC}. It has a very
 modular semantics with degredd some similarity to hardware description
 languages (HDL), such as \emph{VHDL} or \emph{Verilog}. The name \emph{nesC}
 stands for "Network Embedded Systems C", in analogy with HDL it describes
 application level connectivity for modules, protocols and drivers. Despite
 this high level of abstraction, compiled source code is targeted for devices
 with limited compute resources \cite{links:tinyos:nesc}. The development
 ecosystem of \TinyOS\ also includes a network simulation package \emph{TOSSIM}
 \cite{links:tinyos:tossim}, which is a rather trivial element to design when
 such level of abstraction is in place. At the start of the project \TinyOS\
 was of some interest, however it had been opted out in preference to \Contiki\
 for it's more tradition \emph{"plain C"} language abstractions. However,
 \TinyOS\ had been evaluated during the development phase and more details
 can be found under section \ref{sec:TINYOS}.

 %http://en.wikipedia.org/wiki/TinyOS


\subsubsection{Nano-RK} \label{sec:rtos:nanork}

 \emph{Nano-RK} is a fully preemptive reservation-based RTOS, unlike
 any others in this section it already contained source code for \RFA\
 chip. The testing of \emph{Nano-RK} had been added to the project agenda.
 This OS is developed by Carnegie Mellon University and there are various
 interesting research papers available from the website
 \cite{links:nanork:pubs, pubs:nrk07, pubs:nrk06b, pubs:nrk06a, pubs:nrk06c}.
 Some particular features to mention are an implementation of real-time
 wireless communication protocol for voice signal transmission \cite{pubs:nrk06a}
 and a presence of an audio device driver within \emph{Nano-RK} kernel.

\TrackerList
	 \IssueX{21} Test Nano-RK on \RFA
\TrackerEnd



\subsubsection{Contiki - "The OS of Things"} \label{sec:rtos:contiki}

 {\emph{"Contiki is an open source, highly portable, multi-tasking operating
 system for memory-efficient networked embedded systems and wireless
 sensor networks. Contiki is designed for microcontrollers with small
 amounts of memory. A typical Contiki configuration is 2kB of RAM
 and 40kB of ROM.}}
 
 {\emph{"Contiki provides IP communication, both for IPv4 and IPv6. (\dots)}}
 
 {\emph{"Many key mechanisms and ideas from Contiki have been widely adopted in
 the industry. The uIP embedded IP stack, originally released in 2001, is today
 used by hundreds of companies in systems such as freighter ships, satellites
 and oil drilling equipment. (\dots) Contiki's protothreads, first released
 in 2005, have been used in many different embedded systems, ranging from
 digital TV decoders to wireless vibration sensors.}}
 
 {\emph{"Contiki introduced the idea of using IP communication in low-power
 sensor networks. This subsequently lead to an IETF standard and
 the IPSO Aliance (\dots)}}
 
 {\emph{"Contiki is developed by a group of developers from industry and
 academia lead by Adam Dunkels from the Swedish Institute of Computer
 Science."} \cite{contiki:home}

 Many publications by Adam Dunkels can be found on the SICS website
 \cite{dunkels04contiki, tsiftes09enabling}. He is also an author of
 two recent textbooks on IP WSN subject \cite{dunkels09operating,
 vasseur10interconnecting}. In addition, there is a rich set of simple
 code examples illustrating basic applications. The source code has
 well-defined directory hierarchy, build system and programming style
 \cite{contiki:code:style}.

 \emph{Contiki} provides a very simple API for advanced functions,
 such as real-time task scheduling, \emph{protothread} multitasking,
 filesystem access, event timers, TCP \emph{protosockets} and other
 networking features.

 More detailed description \emph{Contiki} is provided the following
 section (\ref{sec:CONTIKI}).
  
