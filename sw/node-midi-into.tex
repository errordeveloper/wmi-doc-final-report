
  To programs where written to test transmission of MIDI. The concept of
 server/client communication could be applied, nevertheless it had been
 found more appropriate to name the parties as the \emph{Talker} and the
 \emph{Listener}. At the time of writing of this report the \emph{Talker}
 program had been fully tested and debugged. The second program can be
 cosidered only as a prototype and does require further work.

  The \emph{protosocket} macros had been utilised for the implementation,
 hence the communication protocol imposed by \emph{protosocket} was TCP.
 There are various improvements that are still to be considered.
 The specification of TCP includes a notion for \emph{urgent data}
 \rfc{0793}, however this feature has received a very limited adoption and
 is known to be handled differently by various implementations of the TCP
 \cite{ietf:draft:tcpm-urgent-data}. It has been found that the \emph{uIP}
 stack does implement some mechanism for the \emph{urgent data}, however
 the documentation on how it can be applied in \Contiki application code
 could not be found. 


% http://tools.ietf.org/pdf/draft-ietf-tcpm-urgent-data-07
 

\subsection{UDP Multicasting}

  A variety of packages exist which utilise UDP multicasting for streaming
 of MIDI data on LAN. These include \emph{ipMIDI} \cite{links:ipmidi} for
 Microsoft Windows and Apple Mac OS X platforms as well as compatible
 packages for Linux - \emph{multimidicast} \cite{links:multimidicast} and
 \emph{qmidinet} \cite{links:qmidinet}. This two Linux packages had been
 tested with current implementation of the \emph{Talker} program and were
 shown suitable performance. The multicasting approach is considered to be
 quite appropriate for streaming of MIDI signals on IP networks, since no
 configuration is required, i.e. none of the participants have to set-up
 connection between each other and only need to send data to the multicast
 IP address\footnote{\emph{For compatibility with ipMIDI the multicast IP
 address should be 225.0.0.37 and the port numbers from 21928 and above.}
 However the \emph{ipMIDI} has not formal protocol and simply streams the
 MIDI signals to a UDP socket, i.e. there is no any error correcting or
 detecting mechanisms.

% http://nerds.de/
% http://llg.cubic.org/tools/multimidicast/
% http://qmidinet.sourceforge.net/qmidinet-index.html

\emph{MIDI Networking in Linux}

  On Linux networking of MIDI clients has been implemented in user-space
 by \emph{ALSA ("Advanced Linux Sound Architecture") utilities package
 \cite{links:linux:alsa}.
 The specification of \emph{ALSA Sequencer Network} is also limited to
 comments in the source code of the \man{1, aseqnet} program. It is design
 using UDP and a specific data structure. The two programs mentioned in
 the previous section do not communicate with \man{aseqnet}, but use
 the \emph{ALSA Sequencer} inteface directly.

% http://alsa-project.org/main/index.php/Developer_Zone
