\newcommand{\Chip}[1]{\emph{\texttt{#1}}}
\newcommand{\Symb}[1]{\emph{\texttt{#1}}}
\newcommand{\MCX}[0]{\Chip{MC1322x}}
\newcommand{\RFA}[0]{\Chip{ATmega128RFA1}}
\newcommand{\Port}[1]{\emph{\texttt{Port #1}}}
\newcommand{\Comp}[1]{\emph{\texttt{#1}}}
\newcommand{\WPAN}[0]{\emph{LR-WPAN}}

\newcommand{\TrackerList}[0]{\subsubsection{\emph{Tracked Issues}}\begin{description}}
\newcommand{\TrackerEnd}[0]{\end{description}}

\newcommand{\Contiki}[0]{\emph{Contiki}}
\newcommand{\ContikiOS}[0]{\emph{Contiki OS}}
\newcommand{\TinyOS}[0]{\emph{TinyOS}}


\documentclass[a4paper]{report}

\usepackage{multirow}
\usepackage{graphicx}

%02:36 <kahrl> errordeveloper: http://physics.wm.edu/~norman/latexhints/conditional_macros.html
%01:37 <kahrl> so \makeatletter \ifx \PRINTING \@empty \renewcommand{\href}{FAKE_VERSION_OF_HREF} \fi \makeatother
%01:39 <kahrl> or \makeatletter \ifx \PRINTING \@empty \relax \else \renewcommand{\href}{FAKE_VERSION_OF_HREF} \fi \makeatother

% `latex '\def\NOHREF{}\input{foo.tex}'`

\ifdefined \NOHREF

	\usepackage{url}
	\newcommand{\URL}[1]{\[ \texttt{\emph{#1}} \]}
	%% \newcommand{\href}[2]{#2 (\texttt{\emph{\url{#1}}})} %% not needed probably
	
	\newcommand{\Href}[2]{{#2}} %% fake version for printing
\else

\usepackage{hyperref}
\newcommand{\Href}[2]{\href{#1}{#2}} %% use this for proper href
\newcommand{\URL}[1]{\[ \Href{#1}{\texttt{\emph{#1}}} \]}

\fi



%% use \Href  and for printing we use \cite{something}
%% so we end up with link to the URL and bibliography
%% reference number as well. If \Href is fake there
%% will be no URL in the text.
\newcommand{\AltRef}[3]{\Href{#2}{#1} \cite{#3}}



\newcommand{\Issue}[1]{\Href{http://wmi.new-synth.info/issues/#1}{Issue \##1}}
\newcommand{\IssueX}[1]{\item[\Href{http://wmi.new-synth.info/issues/#1}{\Issue{#1}}]:}

\newcommand{\TEP}[1]{\AltRef{\emph{TEP#1}}{http://www.tinyos.net/tinyos-2.x/doc/pdf/tep#1.pdf}{links:tinyos:tep:#1}}

%% usage:
%	\MAN{gcc}
%	\MAN{2, open}

%\newcommand{\MAN}[1]{\AltRef{\emph{\texttt{#1}}}{http://manpages.debian.net/cgi-bin/man.cgi?query=#1&sektion=0&format=pdf}{doc:linux:man:0:#1}}
%\newcommand{\MANX}[2]{\AltRef{\emph{\texttt{#2}}}{http://manpages.debian.net/cgi-bin/man.cgi?query=#2&sektion=#1&format=pdf}{doc:linux:man:#1:#2}}

\newcommand{\MAN}[1]{\AltRef{\emph{\texttt{#1}}}{http://manpages.debian.net/cgi-bin/man.cgi?query=#1&sektion=0&format=ascii}{doc:linux:man:0:#1}}
\newcommand{\MANX}[2]{\AltRef{\emph{\texttt{#2}}}{http://manpages.debian.net/cgi-bin/man.cgi?query=#2&sektion=#1&format=ascii}{doc:linux:man:#1:#2}}

%% For this commands we need to use \Href explicitly
%% so if Href is proper - the display the path and link it
%% otherwise - just display the path. Because there's no
%% reason to show various refences to this in the bibtex

\newcommand{\Blob}[1]{\Href{https://github.com/errordeveloper/contiki-wmi/blob/wmi-work/#1}{\texttt{#1}}}
\newcommand{\Tree}[1]{\Href{https://github.com/errordeveloper/contiki-wmi/tree/wmi-work/#1}{\texttt{#1}}}
\newcommand{\Blame}[1]{\Href{https://github.com/errordeveloper/contiki-wmi/blame/wmi-work/#1}{\texttt{#1}}}
\newcommand{\Logs}[1]{\Href{https://github.com/errordeveloper/contiki-wmi/commits/wmi-work/#1}{\texttt{#1}}}
\newcommand{\Contrib}[1]{\Href{https://github.com/errordeveloper/contiki-wmi/commits/wmi-work/#1?author=errordeveloper}{\texttt{#1}}}

\usepackage{tikz}
\usetikzlibrary{automata,positioning,shapes,arrows}

\tikzstyle{C} = [diamond,
			draw, fill=blue!20,
			text width=4.5em, text badly centered,
			node distance=3cm, inner sep=0pt]

\tikzstyle{F} = [rectangle,
			draw, fill=green!20,
			text width=5em, text centered,
			rounded corners, minimum height=2.5em]

\tikzstyle{E} = [ellipse,
			draw, fill=red!20,
			node distance=3cm, minimum height=3em]

\tikzstyle{L} = [draw, -latex']

\usepackage{pstricks-add}

%\DeclareGraphicsExtensions{.eps,.epsi,.pdf,.png,.jpg}

\title{WMI: Wireless Music Instruments\\ \emph{BSc Final Year Project} \\ Final Report}
\author{Ilya Dmitrichenko \\ \\ Department of Computing\\ London Metropolitan University}

\date{\today}



\begin{document}
\maketitle


\abstract
{

   This is the end of year report for a BSc degree project which had originally
   targeted a design of a complete wireless system for streaming of music control
   data collected from sensor devices and MIDI hardware to a host machine which
   would synthesise an audio signal to be delivered into loudspeakers or recorded
   on disk. Despite rather simple design idea, the research and development had
   taken extended time period and the emphasis of this report is on a variety of
   aspects related to wireless sensor network design and implementation. These
   aspects include: cross-platform operating system architecture, development
   hardware, application design and the abstraction layer requirements for
   interfacing hardware functions to the software as well as other miscellaneous
   subjects intersecting these areas. Two popular operating systems are presented
   in this report - \emph{Contiki} and \emph{TinyOS}. Two common microcontroller
   architectures had been used in the course of this project - 32-bit \emph{ARM}
   and 8-bit \emph{AVR}. \emph{IEEE 802.15.4} devices where considered as the
   only suitable radio harware option, due to the wide use and availability of
   the chips which implement this protocol. Another key assumption has been that
   the complice to the Internet Protocol (IP) is neccessary. A simple prototype
   has been implemented and a set of solutions proposed for a complete product
   design. Several advanced subjects for further work are also addressed in this
   report.

}

\tableofcontents
\listoffigures
%\listoftables

%%%%%%%%%%%%%%%%%%%%%%%%%%%%%%%%%%%%%%%%%%%%%%%%%%%%%%%%%%%%%%%%%%%%%%%%

\part{Introduction \& Research}
\chapter{General Background}

\section{Motivation}

  Recent advances in low-power microcontroller and radio frequency data communications
 technology, together with enormous growth in the general purpose embedded computer
 market challenges several industries around the world. However, the music instrument
 industry haven't fully taken all the advantages of the most recent electronic devices
 and standards.
 
 Wireless sensor networks will very soon appear in numerous application areas,
 which may include practically any market (consumer, industrial, medical and many other).
 
 This project aims to implement a networked system for stage performance which uses
 wireless sensor devices as a creative user interface. Despite the title, this also
 concerns any theatrical or dance performance as well as various monitoring and
 control uses in entertainment systems.

\subsection{System Specification \& Requirements}

   This section shall outline the background for technical
 requirements for the system which is about to be discussed
 in this report. Some essential non-technical information
 will be introduced briefly. Below is a very general outline
 of what the requirements are.

  \begin{itemize} \em
	  \item	Inexpensive hardware
	  \item	Low-power components
	  \item	Small form-factor
	  \item	Most recent radio technologies

  \item The system design also should:

  \begin{itemize}
	  \item	be suitable for various similar applications
	  \item	and avoid theoretical limits in scalability 

  \end{itemize}
  \end{itemize}


 Observing contemporary arts and music scene it appears that
 that technology becomes increasingly popular among artists
 and to some surprise there are individuals who attempt to
 introduce non-trivial aspects of technology into visual and
 performed arts. Some are utilising commodity devices, such
 as mobile phones, while others desire to learn microcontroller
 programming and simple circuit design for physical interaction
 with sensors as a creativity instrument.
  As an example, Ryan Jordan in his MFA thesis \cite{paper:ryan09}
 states that \emph{"A fragile DIY hardware and software system has
 been created with various sensors which are attached directly to
 the performers body."}. The word "fragile" atracts the attention
 in this sentense, hence if a system was made such that an artist
 could apply for their performance (or display) in a flexible,
 robus and relieble manner, there is, probably, a niche market
 for it. Ryan Jordan has probably found certain intsperation in
 the actual wiring of his system, wireless provides a diffrent
 benefit to the artist.




\section{Organisation}

  The very initial research and preparation phases of this project had taken place
 prior the start of the academic year \emph{2010/11}. When the project has started
 in October, one of the first steps taken was a creation of website, which provides
 a number of important facilities for project management, file keeping and issue
 tacking. The website can be found at the URL below:
 {\URL{http://wmi.new-synth.info}}
 It will be referred to throughout this report as \emph{The WMI Website}. A current
 source code repository is also available on-line and linked to the website as well
 as work in progress notes and other supplementary information and data.

\subsection{Report Structure}

\subsection{Typographic Conventions}

  Unfamiliar names and acronyms mentioned in the report are typeset in \emph{italic},
 as well as vendor brands when referred to a product from that vendor. Names of device
 models as well as commands and programming language keywords or statements are typed
 in \emph{\texttt{bold-italic}}. Filenames are \texttt{mono-spaced} and in the PDF
 version of the document are hyper-linked to the source code repository\footnote{%
 \emph{The path is aways relative to the source code root directory, unless specified
 otherwise.}}. There will appear a special sub-section titled \emph{"Tracked Issues"}
 in some sections, it shows reference to issue tickets on the website\footnote{\emph{%
 The tickets are all enumerated in one sequence starting from 1 and classified by the
 type of issue each ticket relates to (i.e. "bug", "feature", "task"). Best effort
 was made to file these issue and there are very few remaining undocumented.}}.
 The tickets can be accessed via URL of the following form (replace the \Symb{`\#'}
 symbol with the given number): \URL{http://wmi.new-synth.info/issues/\#}

\chapter{Field \& Market Trends}

 This chapter intends to review the variety of available
 technologies and discuss why some desitions and preference
 had been made.

 \emph{The terminology of the OSI 7-layer model will be used,
 therefore table \ref{tab:osi} will illustrate the stacking
 of all layers in comparison to the Internet Protocol model.
 The reader is expected to be already familiar with this
 terminology, hence it is shown only for their reference.}

\begin{table}[h]
\begin{center}
 \begin{tabular}{|c||c|}
 \hline
 {\bf OSI model}&{\bf IP model}\\
 \hline \hline
 \ \ \  Application \ \ \   & \multirow{2}{*}{ \ \ \   Application \ \ \ }   \\
 \cline{1-1}
 Presentation & \\
 \hline
 Session & \multirow{2}{*}{TCP, UDP,\  \dots }\\
 \cline{1-1}
 Transport & \\
 \hline
 Network & IP\\
 \hline
 Data Link & Data Link\\
 \hline
 Physical & Physical\\
 \hline
 \end{tabular}
 \end{center}
\caption{\emph{Open System Interconnection Model
	and The Internet Protocol}}\label{tab:osi}
\end{table}


\section{Music Control Protocols}

   One of additional motivations to work on low-power wireless
  network for music control, had been a desire to experiment in
  the area of high-level representation of control data, specific
  to live music performance and audio in general.
  Started by considering to extend, not so recently proposed, OSC
  ("Open Sound Control") protocol \cite{paper:osc11}, with a set
  features which it appears to be missing. This is a subject to
  more extensive experimentation and therefore has not been
  included in the scope of the project itself.

   OSC is effectively just an application layer data format and
  is mostly used with UDP. It had been proposed by a group of
  researchers at the Centrer for New Music \& Audio Technologies
  (CNMAT), UC Berkley \cite{links:cnmat}. It was first presented
  in 1997 \cite{paper:osc97} and has received a rather limited
  adoption. Although, it was observed that there is a tendency
  towards a wider adoption of OSC - a number of interesting
  hardware product had been released which use OSC as primary
  protocol. Some of these devices are listed below.

  \begin{itemize}
  	\item \emph{Livid Block64 \cite{links:livid:block64}}
	\item \emph{Jazz Mutant Lemur \cite{links:jazzmutant:lemur}}
	\item \emph{Monome \cite{links:monome, links:monome:osc}}
  \end{itemize}

  It has to be said that OSC seems to be intended as a candidate
  to replace the MIDI protocol (which had been define in 1983 and
  therefore considered in need of a replacement). The greatest
  limitation seen in MIDI today is the size of values it can
  represent, i.e. most control values a bound in 0-127 range.
  Nevertheless, wireless transmission of MIDI was chosen to
  be the initial target for this project. To avoid going
  outside of the scope of this report, it shall be defined
  that MIDI is quite likely to be most appropriate to implement
  at this stage, since OSC format is certainly not suitable.
  The reason for this is that microcontrollers do not have
  the capabilities to handle large amount of data represented
  as character strings. Therefore a new protocol needs to be
  designed, which would overcome most these limitations; though,
  that is already in the scope of another project.



\section{Low-power Digital Radio}

  This section gives an overview of currently available low-power
 wireless communication technologies in general terms, then some of
 the important concepts are briefly introduced to support further
 discussion of these devices and software with appropriate terminology.

  One of the main subjects of the initial research was wireless data
 communication standards for sensing and control applications and it
 should be noted that the transmission of audio signals has not been
 taken into consideration. Also as outlined in the requirenments,
 radio protocols such as UWB (e.g. WUSB, WiMAX) and \emph{IEEE 802.11}
 (i.e. WiFi) are not low-power and therefore are not applicable for the
 purpose of this project. Although, short-range\footnote{\emph{Recent
 amendments in P802.15.4a specify alternative physical layer options
 that include sub-gigahertz UWB modes \cite{links:wiki:p:wpan}.}} UWB
 could be of great use for its potential throughput capabilities, the
 cost and availability of transceivers are yet unknown.
 
  The main interest is in low-power radio of 2.4GHz range. The semiconductor
 market is currently flooded with a variety of inexpensive devices that
 implement either \emph{IEEE 802.15.4} standard or patented protocols such
 as \emph{ZigBee}, \emph{RF4CE} or other that are based on \emph{P802.15.4}.

 A few more different low-power wireless networking technologies exist, such
 as \emph{DASH7} and \emph{ANT}. \emph{DASH7} is an active RFID protocol for
 extended range and it is very specific for certain applications, it operates
 in 434MHz band \cite{links:wiki:p:dash7}. \emph{ANT} is a proprietary standard
 which uses 2.4GHz band \cite{links:wiki:p:ant}. Both of these technologies
 are support only by small groups of silicon chip vendors.

 Multiple standards exist which are using the same hardware functions provided
 by \emph{P802.14.5}-compliant devices\footnote{\emph{This implies that all of
 these protocols are effectively defined only by software.}}, some of these are
 \emph{ZigBee}, \emph{RF4CE} and \emph{6loWPAN} \cite{links:wiki:p:6lowpan}.
 \emph{RF4CE} belongs to the \emph{ZigBee} \cite{links:wiki:p:zigbee} family
 together with a number of other application area specific variations.
 \emph{6loWPAN} \cite{links:wiki:p:6lowpan} is most interesting patent-free
 protocol and it is transparent to existing software, since it is compliant
 to the Internet Protocol. %\IETF

  Nevertheless, all of these technologies are not yet widely available in
 the consumer market, where \emph{BlueTooth} \cite{links:wiki:p:mrwpan} and
 simplistic sub-gigahertz serial radio transceivers or infra-red are commonly
 found. Although, it is most likely that \emph{P802.15.4} transceivers will
 soon dominate low-power wireless application markets.

 Further in this report \emph{P802.15.4} will be referred to as \WPAN\ 
 (stands for Low-Rate\footnote{\emph{Medium-rate (MR-WPAN) are the BlueTooth
 (P802.15.1) networks. Also there is HR-WPAN (P802.15.3) standard defining
 high-rate (UWB) networks. All classes of WPAN together with WLAN are
 are referred to as short-range wireless networks.}} Wireless Personal
 Area Network). A number of amendments to \emph{IEEE 802.15.4} had been
 published, the latest version is \emph{P802.15.4-2006}. In this report
 it shall be looked at as a de-facto solution \cite{links:wiki:p:wpan,
 links:wiki:p:lrwpan, links:ieee:802:15}.

 \emph{Some general concepts of the \WPAN\ hardware described below
 are considered to be absolutely complete for understanding the
 system operation from the software design perspective.}

\subsection{\emph{IEEE 802.15.4} - Low-Rate WPAN}

  This standard was first proposed by the IEEE in 2003 and has
 evolved since. As far as the concepts essestial for application
 development are concerned, there is no major difference between
 the revisions.

 It is important to understand at this point that the concept of
 low-power consumption applies to all layers, so the application
 layer indeed is required to co-operate in order to preserve the
 energy. However, the task of this project is to get maximum
 throughput on \WPAN\ network and attempt to reduce the latency
 and maximize quality of service, hence power preservation
 techniques are considered very briefly. However, the low-power
 requirement for the design is met, since the avarage power rating
 of the device remains relatively low without applying these
 techniques.
 

 \WPAN\ defines two layers of the OSI model, the \emph{Physical (PHY)}
 and \emph{Media Access Control (MAC)} layers. Network and Applications
 layers are defined by other standards mentioned above.


\subsubsection{Physical Layer (PHY)}

\begin{flushright} \small{
\emph{Source: "ZigBee Wireless Networks and Transceivers" \\
	by Shahin Farahani (2008) \cite{b:zigbee}}}
	\end{flushright}


% This layer functions for transmission and reception of radio packets,
%provides control facilities for channel selection and power management.

 "The PHY layer is the closest layer to hardware and directly controls
 and communicates with the radio transceiver.
 The PHY layer is responsible for the following:
 \begin{itemize}

        \item Activating and deactivating the radio transceiver.

	\item Transmitting and receiving data.

	\item Selecting the channel frequency.

	\item Performing Energy Detection (ED).\\
	\small{\emph{The ED is the task of estimating the signal energy within the
	frequency band of interest. This estimate is used to understand
	whether or not a channel is clear and can be used for transmission.}}

	\item Performing Clear Channel Assessment (CCA).

	\item Generating a Link Quality Indication (LQI).\\
	\small{\emph{The LQI is an indication of the quality of the data packets
	received by the receiver. The signal strength can be used as
	an indication of signal quality.}}"
	\end{itemize}

 The \WPAN\ standard defines use
 for a number of bands in different geographical regions, the 2.4GHz
 band can be used anywhere in the world.  The details regarding RF
 modulation techniques and various regulations are outside of the
 subject of this report. There are 27 channels in different bands, 
 2.4GHz band has been assigned with channel numbers 11 to 26.

  Power regulations apply depending on geographical region, the
 measures are transceiver output power and duty cycle. The global
 ISM band can be utilized at 100\% duty on approximately 10mW level.

 Two modulation techniques can be used in 2.4GHz band:
 \begin{description}
 \item[\emph{Offset-QPSK}]- Offset Quadrature Phase-Shift Keying
 \cite{links:wiki:qpsk}
 \item[\emph{DSSS}]- Direct-Sequence Spread Spectrum (was used in 2003 revision)
 \cite{links:wiki:dsss}
 \end{description}

 Alternative modulations techniques are defined in amendment
 \emph{P802.15.4a}, these include Ultra-Wide Band \emph{(UWB)}
 and Chirp Spread Spectrum \emph{(CSS)}.

 Common basic data rate is 250kbps and distance coverage is from 10m
 to 100m, but higher limits can be achieved (up to 2Mbps).

\subsubsection{Media Access Control (MAC)}

\begin{flushright} \small{
\emph{Source: "ZigBee Wireless Networks and Transceivers" \\
	by Shahin Farahani (2008) \cite{b:zigbee}}}
	\end{flushright}

  "The MAC provides the interface between the PHY
 and the next higher layer above the MAC."

% {This layer is above physical and therefore it controls what is being
% transmitted and how it is done.
% Some of the most important concepts are:
%	\begin{itemize}
% 		\item Super Frames and Timeslots\\
%		\small{\emph{These provide mechanism for real-time data transmission.}
%		\item Encryption\\
%		\small{\emph{Most of commercially available devices use 128-bit AES}
%		\dots Beacon Frames\\
%		\small{\emph{Essentially beacons can be seen as advertisement messages.}

 "The IEEE 802.15.4 defines four MAC frame structures:
\begin{itemize}
       \item \texttt{Beacon Frame}
	\small{\emph{ --- used by a coordinator to transmit beacons.\\
       			The beacons are used to synchronize the clock\\
			of all the devices within the same network.}}
       \item \texttt{Data Frame}
	\small{\emph{ --- used to transmit data.}}

       \item \texttt{Acknowledge Frame}
	\small{\emph{ --- used to acknowledge the successful reception of a packet.}}
       \item \texttt{MAC Command Frame}
	\small{\emph{ --- are used for commands such as requesting the data\\
			  and association or disassociation with a network.}}

\end{itemize}

It is important to understand that frames from each network layer are
encapsulated into each other, i.e. the MAC frame are encapsulated into
PHY frames on transmission and on reception these data structure is
being decoupled.


\subsubsection{\WPAN\ Classification: Nodes and Topologies}

  The \WPAN\ device hierarchy is defined in terms of \emph{full-}
  and \emph{reduced-function} devices (FFD and RFD for short).
  
  \begin{itemize}
 	\item Routers (FFD) 
	\begin{itemize}
		\item Network Coordinator
		\item Branch Coordinator
		\item Border Router
	\end{itemize}
	\item Participant Clients (RFD or FFD)
  \end{itemize}

  \WPAN\ Topologies are:
  \begin{itemize}
  	\item Point-to-Point
	\item Star
	\item Mesh
  \end{itemize}

\subsubsection{Feature Outline} \label{sec:P802154}
\emph{Source: IEEE 802.15 Task Group 4 Website}\cite{links:ieee:802:15:4}.

\begin{itemize}
	\item Data rates of 250 kbps, 40 kbps, and 20 kbps.

	\item Two addressing modes: 16-bit short and 64-bit.

	\item Support for critical latency devices, such as joysticks.

	\item \emph{Carrier Sense Multiple Access with Collision
		Avoidance \\(CSMA-CA)} channel access mode \cite{links:wiki:csma}.

	\item Automatic network establishment by the coordinator.

	\item Fully hand-shaked protocol for transfer reliability.

	\item Power management to ensure low power consumption.

\end{itemize}

\subsection{Conclusions}

  It has been found that some of the concepts defined in the \emph{IEEE}
 paper are of very little practical use. For example, the hierarchy
 specification is only used in \emph{ZigBee} and does not apply to
 \emph{6loWPAN} and the entire MAC layer specification is disregarded
 by the \emph{6loWPAN} user community and many papers where presented
 which propose various improvements, for example \cite{paper:amac}
 (One most highly regarded papers presented at the ACM conference
 on Embedded Networked Sensor Systems in November 2010).

 It does appear that Media Access Control for \WPAN\ as well as IP
 Routing Over Low-power and Lossy networks (ROLL) \cite{ietf:draft:roll}
 are subject to very extensive research at present, as well as time
 synchronisation \cite{paper:ts4,paper:ts2,paper:ts3,paper:ts1,
 Lenzen2010Clock, Lenzen2009Optimal, Sommer2009Gradient, Sommer2008Symmetric}.
 Many papers appeared on this subject and it is of greatest concert
 to the music instruments application, however it was found too complex
 to cover in the course of this project.



\subsection{Higher Layers and the Application Layer}

  The greatest benefit that the OSI model gives, is that any layer
 can be re-implemented without any changes to other layers above or
 below. Also translation between different implementations would a
 trivial task. \emph{6loWPAN} allows for connectivity of low-power
 wireless nodes to the Internet, enabling the future paradigm of
 \emph{"the Internet of Things"}. This has been developed and
 popularised by an alliance of Internet Protocol for Smart Objects
 (IPSO) \cite{links:ipso:homepage}. \emph{6loWPAN} takes care of
 several aspects such as compute resources limits and address space
 requirements due to large amount of participant nodes by providing
 \emph{IPv6} addressing with additional features such as \emph{header
 compression}, reduced functionality as well as other miscellaneous
 enhancements \cite{pubs:ipso:wp3, pubs:ipso:wp1}.

  By enabling Internet connectivity for the sensor nodes, the application
 developers are presented with lesser challenge, because existing algorithms,
 packages and libraries can be used. Care still needs to be taken due
 to limitations such as bandwidth and data structure complexity.

 For example, a Representational State Transfer Application Programming
 Interface (\emph{RESTful API}) \cite{links:wiki:rest} would make a
 system much simpler to integrate into existing environments,
 there won't be a need to develop tools for any particular computer platform.
 \emph{RESTful} approach brings a cross-platform compatibility out-of-the-box
 by utilising web-client software with very little of extra work to be done.
 \emph{RESTful API} is largely used in modern web-application services
 (also known as "the cloud"), it uses human readable URL strings for
 addressing functions on remote servers via HTTP protocol (the handling
 of HTML or XML response is optional and not applicable here, due to above
 mentioned data structure complexity issue).

 Another feature of \emph{IPv6} that is very important for complex wireless
 networks, is \emph{neighborhood discovery}, which eliminates any need
 for what DHCP served in \emph{IPv4}. Regarding the scalability,
 it is true that there is no particular use in this project for 128-bit
 address space that \emph{IPv6} provides. Nevertheless, this network
 may need to extend to a larger number of nodes if a orchestra of these
 is going to be built, also including various stage automation and
 safety sensors all in one area.


\subsection{Alternatives and Concerns}

  It is very appropriate to discuss a few alternative approaches which
 could be taken to implement a radio network specific to the application
 of interactive objects for live performance.

  An alternative to using \WPAN\ radio would be a different, rather
 simpler protocol and, perhaps, in a different frequency band.
  The reason for this is obviously because of application-specific
 concerns. The data transmission for musical performance is constrained
 to very strict real-time latency measures. In other words, when a
 musician is playing on stage the audience will perceive the music as
 very odd if the latency is too high. It would be even worse if several
 participants exhibit random latency, while not playing an improvised
 piece of music.

  Using a radio protocol which is in its own \emph{clear} frequency
 band and uses a minimal communication abstraction stack, should be
 more appropriate for latency optimisation. However, the cross-platform
 networking capabilities would be lost. That might lead to incompatibilities
 with market trends as well as radio band regulations.
  Since \emph{IEEE 802.15.4} is the current global technology trend and
 the 2.4GHz ISM band is accepted world-wide, the alternatives are not
 quite appropriate.

  There are a few solutions for \emph{non-P802.15.4} 2.4GHz ISM radio,
 but these would impose a vendor locking for the transceiver interfaces
 to such manufacturers as \emph{Nordic Semiconductors} \cite{links:nordic:rf2400}
 or \emph{Hope Microelectronics} \cite{links:hoperf:rf2400}.
 There might be even model-specific incompatibilities, while 
 it is quite certain that all \emph{P802.15.4} transceiver
 chips will be compatible in the future.

\break

\section{Study of Earlier Work \\and Commercial Solutions}

  Earlier works in the field of wireless performance devices (i.e. music instruments and
  devices to assist dance and theater performance) were studied. A number of links to
  other sites and publication are listed in the WMI website sections "Links" \cite{wmi:wiki:links}
  and "Bibliography" \cite{wmi:wiki:refs}. Below only the most relevant information is discussed.

\subsection{Publications}

  One publication by Paulo Jorge Bartolomeu (Univ. of Aveiro, 2005)
 \cite{pub:bartolomeu2005} which has also been submitted to the IEEE,
 evaluates the use of \emph{BlueTooth} for a wireless MIDI network.
 Paulo Bartolomeu in his dissertation considered either \emph{ZigBee} or
 \emph{UWB} as a possible alternative, however in 2005 these technologies
 still were immature and expensive. A number of interesting results and
 figures are provided, proving that \emph{BlueTooth} exhibits a definite
 byte latency of, on average, 20ms and a maximum of 100ms, these figures
 were reduced with improved network configuration to an average byte latency
 below 10ms. A quite interesting "command aggregation algorithm" had also
 been proposed in this paper.

 A surprisingly small amount of white papers have been found on this subject, 
 which indeed makes this study more interesting. Though, there are various
 articles available on-line, these are listed on the WMI website. There is
 no significant information to discuss in regards to those articles
 \cite{links:misc:xbeemidi}.
 
 A few short reports demonstrating \emph{BlueTooth} implementations were
 found on CCRMA (Centre for Research in Music and Acoustics at UC Stanford)
 website \cite{links:ccrma}, these appear to be homework reports by CCRMA
 students \cite{pub:ccrma:muggling, pub:ccrma:brbi}. Other related articles
 are listed in the bibliography \cite{pub:ccrma:sm, pub:ccrma:bb} appeared
 in 2005 and only one recent publication describing another \emph{BlueTooth}
 controller for \emph{PureData} \cite{links:wiki:pd} written by a researcher
 from Plymouth University \cite{pub:plym:go}.


\subsection{Commercial Wireless MIDI Products}

  Currently there are four commercial solution available from different
 manufacturers. It is most likely that the radio communication protocol
 used by these devices is unique to each particular manufacturer, since
 no information on compatibility is provided.

\emph{
\begin{itemize}
 	\item Kenton MidiStream \cite{links:kenton:midistream}
	\item M-Audio MidAir \cite{links:maudio:midair}
	\item CME WIDI \cite{links:cme:widi}
	\item PatchMan MidiJet \cite{links:pmm:midijet}
\end{itemize}}

 The \emph{Kenton} devices operate in the two sub-gigahertz ranges, 869MHz
 and 433Mhz, while the other three manufactures use the 2.4GHz ISM band.
 It may be desired to evaluate these devices before proposing any solution
 of a commercial nature; there is no particular interest to obtain any
 of these product otherwise.


\subsubsection{Other Related Hardware}

 The \emph{MiniBee} project by the Canadian group of artists, \emph{Sense/Stage}
 \cite{links:sensestage:homepage}, is of particular interest, it utilizes
 high-level frameworks such as \emph{Pure Data}, \emph{Max MSP} and
 \emph{Super Collider} \cite{links:sensestage:host}. The \emph{MiniBee} hardware
 \cite{links:sensestage:node} is an \emph{Atmel AVR} 8-bit MCU and \emph{Digi
 Xbee} transceiver module, which is not the most optimal design solution.
 Again, \emph{ZigBee} is a patented technology and is not appropriate for an
 open-source project. Also, the \emph{Xbee} module itself is already using an
 MCU (\Chip{HCS08}) to run its software stack \cite{links:xbee:wiki:prog}.

 
 In the area of sensor products, there is one interesting company called
 \emph{Phidgets, Inc.} \cite{links:phidgets:homepage}, it manufactures rather
 wide variety of sensor devices, ranging from base single-board computer (SBC)
 to sensor and actuators sharing same type of 3-pin connector. The SBC board
 runs custom-built \emph{Linux OS} on \emph{ARM9} 266MHz CPU with 64MB of
 SDRAM and 64MB of flash memory. \emph{Phidgets SBC}
 \cite{links:phidgets:products:sbc} has USB and 100MB Ethernet peripherals as
 well as a number of screw terminals and 3-pin connectors for sensors.
 \emph{Phidgets, Inc.} supplies numerous easy-to-use sensor and actuator
 devices. The cost of their products is quite high, but the reason of
 this is the target market. Buyers of these easy-to-use devices are not
 professional electronics engineers and are ready to pay these prices.
 Their website also has source code for reference in a wide selection of
 programming languages and frameworks.
 
 However the top product in the hobbyist marker remains \emph{Arduino}
 \cite{links:arduino:homepage}, of course the low price factor is key to its
 popularity. A huge number of semi-clone products emerged, all featuring
 various additions that some users may find more useful. The most recent
 interesting semi-clone of \emph{the Arduino} is \emph{Seeeduino Film}
 \cite{links:seeed:products:film}, made by \emph{Seeed Studio} in China.
 It appears to be the only product of its class that uses flexible film
 instead of regular circuit board. For example, this could be quite
 useful for integration into a music instrument or attached on performer's
 body in combination with capacitive flex sensors. Even the above mentioned
 \emph{MiniBee} in its design is also a semi-clone of \emph{the Arduino},
 in order to support convenient \emph{Arduino C++ Library}
 \cite{links:arduino:library}, which is in no doubt appreciated by the
 target audience.


\part{Design \& Development} 
\chapter{Operating Systems for Wireless Sensor Networks}
\section{Overview of Embedded Operating Systems} \label{sec:RTOS}

		%% REWRITE !!

   A subject of extensive research for this project has been into the
  area of Real-Time Operating Systems for embedded devices, RTOS for short.
  This family of operating systems is not like any other general
  purpose OS family. The distinct feature that any RTOS has is the task
  management in constrained time-scale, the term \emph{real-time} is relatively
  abstract, it can be classified either as \emph{soft-} or \emph{hard-real-time}.
  The classification is also an abstract matter, some operating systems provide
  various approaches and it is up to the engineer to utilise it correctly for
  any given application. A modern RTOS should also provide a communication
  protocol stack and file storage drivers, support a set of platforms and
  a variety of peripheral devices.
 
 It is commonly known that RTOS is a rather expensive element in embedded design.
 This is because until very recently most commonly used RTOS implementations were
 provided only by specialised commercial vendors, brand names like \emph{Micrium},
 \emph{VxWorks} and perhaps \emph{QNX} are popular in specific application fields
 and there are also more obscure RTOS products offered at premium prices. Another
 more recent commercial RTOS is \emph{Nucleus}, currently owned by \emph{Mentor
 Graphics}. This repor does not focus on commercial products, hence these will
 not be refer to, more information can be found in \cite{wiki:rtos,wiki:rtos:list}.

 %http://en.wikipedia.org/wiki/Real-time_operating_system
 %http://en.wikipedia.org/wiki/List_of_real-time_operating_systems

 Currently the open sources community does present a number of very interesting
 RTOS projects, some of which are briefly introduced below. More details can be
 found on the WMI website \cite{wmi:wiki:rtos} (all the general project URLs are
 listed in that section of the website, and therefore omitted from the text below).
 It may be important to note that many of the project which are listed here are
 originating from major universities, some of which previously gave birth to very
 well know UNIX software and general innovations in the computer science.


 \subsubsection{RTEMS} \label{sec:rtos:rtems}

 The \emph{Real-Time Executive for Multiprocessor Systems} or \emph{RTEMS} is a
 full featured RTOS that supports a variety of open API and interface standards.
 It originates from US military and space exploration organisations. By definition
 \emph{RTEMS} is a highly scalable system, i.e. it is designed for multi-processor 
 deployment and is an interesting system to use for various possible applications,
 \emph{RTEMS} appears to be the oldest open-source RTOS. Unlike other systems below,
 the \emph{RTEMS} doesn't belong to sensor network OS sub-family on which this
 project had been focused. 


\subsubsection{TinyOS} \label{sec:rtos:tinyos}

 \emph{TinyOS} originates from the University of California, Berkeley and
 Stanford and the Intel Research Laboratory at Berkeley \cite{wiki:tinyos,
 links:tinyos:webs, links:tinyos:homepage}. The approach of \emph{TinyOS}
 is very different from all other systems, it uses a higher level abstraction
 language based on \emph{C}, which is called \emph{nesC}. It has a very
 modular semantics with degredd some similarity to hardware description
 languages (HDL), such as \emph{VHDL} or \emph{Verilog}. The name \emph{nesC}
 stands for "Network Embedded Systems C", in analogy with HDL it describes
 application level connectivity for modules, protocols and drivers. Despite
 this high level of abstraction, compiled source code is targeted for devices
 with limited compute resources \cite{links:tinyos:nesc}. The development
 ecosystem of \TinyOS\ also includes a network simulation package \emph{TOSSIM}
 \cite{links:tinyos:tossim}, which is a rather trivial element to design when
 such level of abstraction is in place. At the start of the project \TinyOS\
 was of some interest, however it had been opted out in preference to \Contiki\
 for it's more tradition \emph{"plain C"} language abstractions. However,
 \TinyOS\ had been evaluated during the development phase and more details
 can be found under section \ref{sec:TINYOS}.

 %http://en.wikipedia.org/wiki/TinyOS


\subsubsection{Nano-RK} \label{sec:rtos:nanork}

 \emph{Nano-RK} is a fully preemptive reservation-based RTOS, unlike
 any others in this section it already contained source code for \RFA\
 chip. The testing of \emph{Nano-RK} had been added to the project agenda.
 This OS is developed by Carnegie Mellon University and there are various
 interesting research papers available from the website
 \cite{links:nanork:pubs, pubs:nrk07, pubs:nrk06b, pubs:nrk06a, pubs:nrk06c}.
 Some particular features to mention are an implementation of real-time
 wireless communication protocol for voice signal transmission \cite{pubs:nrk06a}
 and a presence of an audio device driver within \emph{Nano-RK} kernel.

\TrackerList
	 \IssueX{21} Test Nano-RK on \RFA
\TrackerEnd



\subsubsection{Contiki - "The OS of Things"} \label{sec:rtos:contiki}

 {\emph{"Contiki is an open source, highly portable, multi-tasking operating
 system for memory-efficient networked embedded systems and wireless
 sensor networks. Contiki is designed for microcontrollers with small
 amounts of memory. A typical Contiki configuration is 2kB of RAM
 and 40kB of ROM.}}
 
 {\emph{"Contiki provides IP communication, both for IPv4 and IPv6. (\dots)}}
 
 {\emph{"Many key mechanisms and ideas from Contiki have been widely adopted in
 the industry. The uIP embedded IP stack, originally released in 2001, is today
 used by hundreds of companies in systems such as freighter ships, satellites
 and oil drilling equipment. (\dots) Contiki's protothreads, first released
 in 2005, have been used in many different embedded systems, ranging from
 digital TV decoders to wireless vibration sensors.}}
 
 {\emph{"Contiki introduced the idea of using IP communication in low-power
 sensor networks. This subsequently lead to an IETF standard and
 the IPSO Aliance (\dots)}}
 
 {\emph{"Contiki is developed by a group of developers from industry and
 academia lead by Adam Dunkels from the Swedish Institute of Computer
 Science."} \cite{contiki:home}

 Many publications by Adam Dunkels can be found on the SICS website
 \cite{dunkels04contiki, tsiftes09enabling}. He is also an author of
 two recent textbooks on IP WSN subject \cite{dunkels09operating,
 vasseur10interconnecting}. In addition, there is a rich set of simple
 code examples illustrating basic applications. The source code has
 well-defined directory hierarchy, build system and programming style
 \cite{contiki:code:style}.

 \emph{Contiki} provides a very simple API for advanced functions,
 such as real-time task scheduling, \emph{protothread} multitasking,
 filesystem access, event timers, TCP \emph{protosockets} and other
 networking features.

 More detailed description \emph{Contiki} is provided the following
 section (\ref{sec:CONTIKI}).
  

\section{Introducing the Contiki Operating System}\label{sec:CONTIKI}

  \Contiki\ operating system can be seen as collection of processes.
 In the same way as the UNIX operating system is seen as collection of
 files and processes. However, files are not a necessary component, in
 fact \emph{Coffee File System (CFS)} is only needed for some specific
 applications which require a non-volatile storage abstraction. 
  The inter-process communication (IPC) in UNIX has not been implement
 in the very early releases, although Contiki processes wouldn't work
 without its event-driven IPC mechanism\footnote{\emph{Though it is very
 primitive comparing to the usual meaning of IPC as an acronym.}}, which also
 drives the core scheduler. There is no notion of users in \Contiki\,
 however, and at this point comparison with UNIX becomes apparently
 impossible. Nevertheless, such a high-contrast comparison will hopefully
 deliver a good picture of what \ContikiOS\ is and what it is not.

% check the date in the slides from WSN course

  The lead developer of \Contiki\, Adam Dunkels of Swedish Institute of
 Computer Science (SICS), has revolutionised embedded system world by
 implementing \emph{lwIP} in 2000 and later, in 2002, even further optimised
 \emph{uIP} stack. It is known that until \emph{lwIP}, no complete IP stack
 existed which could run on a microcontroller device \cite{dunkels03full}.

  The key technique utilised in implementing \Contiki\ and the \emph{uIP}
 stack is based on a very obscure behaviour which the C language's
 \texttt{switch/case} statement exhibits in certain context. Dunkels
 wrote a few papers \cite{dunkels05protothreads,dunkels06protothreads}
 describing how it had been achieved, and summarises it in his doctoral
 thesis \cite{dunkels07programming}. The threading technique has been
 named \emph{protothreads (PT)}. Processes in \Contiki\ are designed as
 an extension to the \emph{protothreads}. Another extension is called
 the \emph{protosockets}, as the name suggest it provides the abstraction
 of network sockets.

  Most of functionality of all three key elements is delivered by the use
 of C pre-processor macros. The \emph{protothreads} are platform
 independent\footnote{\emph{It is not completely true for Contiki, there
 are few additions which utilise facilities that some hardware platforms provide.}}
 and can be integrated into any C program by including only one header file.
 \emph{Protothreads} can be used with virtually any C compiler toolchain and
 as described in \cite{dunkels06protothreads}, the overhead is rather
 insignificant, considering the powerful functionality; the only limitation
 is due to the use of \emph{\texttt{switch/case}} statements by the \emph{PT}
 macros, these control structures will lead to undefined behaviour of the
 code in the \emph{PT} context.

  \emph{Protosockets} are designed specifically for \emph{uIP} and therefore 
 cannot be included with just one header. The only limitation of the
 \emph{protosockets} is due to design decision, there is no UDP communication
 facility, only TCP is provided. The \emph{protosockets} greatly simplify the
 application code, since all necessary functions are provided, including
 handling of strings. Thought no packet flushing is currently possible, which
 may be desired in some application.

\subsection{Initial Development Phase}

  \Contiki\ source code includes a large set of stable drivers for various
 RCBs which were mentioned previously, such as \emph{Zolertia Z1}, \emph{Tmote
 Sky}, \emph{Atmel Raven} and \Chip{RF230}\emph{-based} boards, \emph{TI}
 \Chip{CC2430} and \emph{MSP430} as well as \emph{ARM Conrtex-MX} and many
 older devices, including \emph{Comondore 64} and \emph{Apple II} computers.

 Details regarding the \Chip{MC1322x} port \cite{links:contiki:port:mc1322x}
 are to follow in \ref{sec:APPDEV}, since \Chip{MC1322x} was chosen as the
 target platfrom for this project. However, the \RFA\ platform had been the
 original preference. Major work during the inital phase had been done to
 port the existing driver code for \Chip{RF230} transceiver to take the
 advantage of the single chip device. This work had been described in the
 interim report \cite{wmi:irep}.


\section{Outlook Trial for the TinyOS} \label{sec:TINYOS}

  It has been discovered that a few members of \TinyOS\ community
 have recently worked on porting \TinyOS\ to run on \Chip{ATmega128RFA1}
 \cite{links:tinyos:rfa1:p1,links:tinyos:rfa1:p2}. Comparing to \Contiki,
 \TinyOS\ has rather superior abstraction layer that provides a seamless
 cross-platform integration methods [\TEP{2}, \TEP{108}, \TEP{109}, \TEP{131}].

 
  Various internals of \TinyOS\ had been studied, however it is 
 certainly a very broad area to be described here. It is best described
 in the \emph{"TinyOS Programming"} book by David Gay and Philip Levis
 (it is available in print as well as a web-edition \cite{levis2009tinyos}).
 Two authors of this book are lead developer of the \TinyOS\ project
 and are currently working at the University of California, Berkley
 and Standford.


  \TinyOS\ currently had been ported to a variety of 8-, 16- and
 32-bit microcontrollers, most of the peripherals (such as I2C and SPI)
 have unified high-level access mechanisms, unlike in \Contiki\footnote{
 \emph{Most of peripheral drivers in Contiki OS are rather specific to
 a certain platform and only some particular drives share a similar
 interface.}}. Most of development documentation is provided in the
 the format of \emph{"TinyOS Enhancement Proposal"} (known as
 \emph{TEP}\footnote{\emph{These documents are commonly referred to as
 TEP\texttt{n}, where \texttt{n} is a number. \TEP{1} defines the format
 of the TEP documents.}}), there is also documentation generated from
 the source code comments (\emph{nescdoc}) as well as various on-line
 resources \cite{tinyos:docs} and the textbook mentioned above. It is
 noticeable that \TinyOS\ documentation is more extensive then what is
 currently present for the \ContikiOS.


\subsection{Programming with TinyOS:\\Compilers and Abstractions}

  The greatest achievement of \emph{TinyOS} is its specialised language,
 benefits of which had been overlooked at the earlier stage of research
 for this project. \emph{"Network Embedded Systems C" (nesC)} is a
 C-based language, it provides abstractions for components, interfaces
 and configurations. It is not an object-oriented language, though
 it is rather component and interface oriented. As it was defined in
 the previous section, \Contiki\ can be seen as a collection of
 processes and events are communicated between those processes, then
 \TinyOS\ can be defined as a collection of components and interfaces
 which are wired together in configuration abstraction layer, tasks,
 events and the data are communicated between the components via the
 interfaces.

  The control structures and all of C constructs are still valid in
 \emph{nesC}. This can be seen as if \emph{nesC}-specific statements
 are replaced by appropriate C source code that defines required behaviour.
 Nevertheless, this is only a brief description of the relation between
 \emph{nesC} and its ancestor. Just as if C could be defined by stating
 a ratio of lines of code in C versus lines of assembly code.

 It is important to note, that current implementation of the compiler
 translates higher-level \emoh{nesC} code into C language. The resulting
 programs are specifically suited for embedded devices. Direct
 compilation would be possible and, if desired, there is an interesting
 platform to look into. \emph{LLVM} ("Low-level Virtual Machine") is a
 new generation compiler technology \cite{links:wiki:llvm}. It is know
 that using \emph{LLVM} and its family member \emph{clang} implementation
 of a new compiler for a C-like language would be simplified (comparing
 to more traditional techniques). One good example of industrial grade
 compiler based on \emph{LLVM} is the \emph{XC} toolchain for \emph{XS-1}
 devices from a UK semiconductor company \emph{XMOS} \cite{links:xmos:tools}.

 Nevertheless, currently \emph{nesC} compiler is known to be fully
 functional and it's task is not as complex as it may seem. Programming
 in any language is always done by applying code patterns, general
 to some degree, to accomplish desired behaviour of a program.
 The purpose of \emph{nesC} abstraction layer can be seen as making
 the details of complex coding - with which high degree of modularity
 can be achieved - rather hidden away from the programmer.


\subsection{Network Protocols in TinyOS}
 
  \TinyOS\ has become widely adopted and there are many researchers
 who contributed significant work in different application areas, one
 such area of great interest to WMI project is sensor node timer
 synchronisation protocols. A number of papers are available \cite{
 Lenzen2010Clock, Lenzen2009Optimal, Sommer2009Gradient, Sommer2008Symmetric}
 and the mainstream repository of \emph{TinyOS} source code already contains
 implementations for some of the proposed protocols [\TEP{132},\TEP{133}].

  Further study and experimentation in this area are required to design
 a system which could cope with real-time constraints of stage control
 applications. It is important to note that \emph{P802.15.4} has addressed
 some real-time application requirements (\ref{sec:P802154}), however the
 status of software support for these features of \emph{P802.15.4} is
 currently uncertain. It is most likely that modification of the MAC layer
 is required to implement dual-mode transmission (real-time and standard
 non-priority). It appears however, that some reaserch
 in this area has been done by the \emph{Nano-RK} developers
 \cite{pubs:nrk06c}\footnote{\emph{More details are also available at:
 \URL{http://nano-rk.org/wiki/RT-Link}}}, thought more work is needed
 to find the most appropriate way to implement it within \TinyOS\ or
 \Contiki.
 
  After several aspects of \emph{TinyOS} were studied, it was clear
 that its current implementation of \emph{6loWPAN} is fully compatible
 with \emph{Contiki} and it had been desired to prove this in practice,
 but this has not been achieved at the time of writing of this paper.
  
  A problem exists however, there was no fully working IP-enabled driver
 for \emph{ATmega128RFA1} transceiver. The IP layers are provided by
 \emph{BLIP} stack (this stands for Berkley Lightweight IP), the stack
 is currently undergoing major development. Details on what is required
 are available on \emph{TinyOS} wiki page \cite{tinyos:wiki:blip20},
 implementing it in \emph{nesC} was not as trivial due to the learning
 curve. Therefore this had be postponed.

  This problem requires further explanation. \emph{TinyOS} has been
 developed since 2001, while \emph{Contiki} first dates back to two
 years later - 2003\footnote{\emph{No exact information has been
 found, these are the earliest dates which appear in the source code.}}. 
 In the early days of WSN research \emph{P802.15.4} and \emph{ZigBee}
 where still emerging and the \emph{6loWPAN} RFCs appeared at IETF more
 recently\footnote{\emph{The first revision of P802.15.4 was released
 in 2003 (current revision is from 2006) and first draft of 6loWPAN
 is dated April 2007.}}. \emph{TinyOS} has originally used its own
 protocol called \emph{ActiveMessage}.

  Currently \TinyOS\ incorporates \emph{BLIP} stack, which initially
 was released by UC Berkley Wireless Embedded Systems research group
 in the summer 2008, know as \emph{bl6lowpan} at that time \cite{ucb:webs:blip}.
 As mentioned below, major code changes are currently still in progress.
 The driver code which fully implements all new features of the \emph{BLIP}
 stack (including point-to-point tunnelling for simple UART wired
 connectivity) is only for the most popular \Chip{CC2420} radio chips.

\subsection{Hardware Resources}

  It is difficult to compare the two sets of hardware platforms which
 \TinyOS\ and \Contiki\ had been ported to, since the status of support
 for some platforms is uncertain. It is probably most appropriate to
 say that these two sets are almost equal, however, some platforms
 which appear in \Contiki\ source code, had not been ported to \TinyOS\
 and vise versa. Two devices of interest to WMI project are \RFA\ and
 \MCX. The issue with the first chip is already described above.
 The second chip has not been ported to \TinyOS\ yet. This should not
 be too difficult task to achieve considering that driver implementation
 could be derived from \Contiki\ code. There also very noticeable
 progress towards \emph{ARM Cortex} support\footnote{\emph{The homepage
 of TinyOS Cortex project can be found at:
 \URL{http://code.google.com/p/tinyos-cortex/}
 The code structure was found to be well organised and most probably
 can be relied upon. However, it has not been desired to take this
 challenge in the near future.}}, hence there is base for \emph{ARM}
 devices (i.e. toolchain support and core architecture code).
 

  There is a large repository of board design files featuring so called
 \emph{Epic Mote} platform, that is base around \Chip{CC2420} and the
 \Chip{MSP430} \cite{links:epic}. The most interesting designed by
 Prabal Duta of UC Berkley \cite{duta:homepage} is shown in Figure
 \ref{fig:mote:quanto}. The reason why it is so interesting is because
 it features \emph{Digi Connect-ME} microprocessor system block that is
 of standard RJ45 form-factor \cite{links:digi:cme}. This tiny device
 has 55MHz \Chip{ARM7TDMI} CPU, 4MB of flash and 8MB of SDRAM as well as
 hardware cryptographic unit and is capable to run Linux kernel and a
 small subset of standard software in userspace, hence the advantage of
 scripting languages can be utilised. Currently it is the smallest
 device available on the market featuring such capabilities and
 the unique form-factor. It is a very appropriate device to use for
 wireless network edge gateway system, for example a hardware crypto
 unit could accelerate secure tunneling of the network traffic. It
 could be considered as a better solution instead of using a large
 Ethernet chip, unless there would be a chip with a combination of
 Ethernet and RF hardware in a single package.

\begin{figure}
\includegraphics[scale=0.5]{../figures/images/epic_mote_quanto.png}
\caption{\emph{Epic Quanto mote featuring Digi Connect-ME}}\label{fig:mote:quanto}
\end{figure}


\subsection{Conclusion}

  Several steps were made in attempt to bring up \emph{TinyOS} on the
 hardware chosen for this project earlier, though it appeared rather
 unmanageable within the give time frame. This is still a subject of
 interest and shall be looked into at a later time. An evidence of
 work carried out with \TinyOS\ code can be viewed on WMI website
 issue tracker.

\TrackerList
 \IssueX{30} Porting Radio Interface to BLIP 2.0
\TrackerEnd

%\URL{https://github.com/errordeveloper/tinyos-wmi/commits/wmi-work?author=errordeveloper}

\chapter{DSP Host System}
\section{DSP Host Hardware}

  A set of hardware platforms suitable for DSP host were considered.
 Among those are the following:
 	\begin{itemize}
		\item \emph{Analog Devices SHARC}
		\item \emph{ARM:} \begin{itemize}
		\item \emph{Marvell Sheeva}
		\item \emph{Texas Instruments OMAP}
		\item \emph{Freescale i.MX}
		\end{itemize}
	\end{itemize}
 All of these processor architectures are fully supported and widely
 used in embedded systems. Various other architecture families were
 considered, but found rather inappropriate due to the price range
 dictated by the target of this project. Multi-core \emph{MISP64}
 devices by \emph{NetLogic} \cite{netlogic:mips64:multicore}, for
 example, have great computational capacity, thought these are too
 expensive for this particular application.

\subsection{x86-based Embedded Devices}

  The list above does not include the \emph{x86}-based CPUs due to a
 fact that it had been very difficult to find a target board using
 Intel or any \emph{x86} CPU from other vendors. There are many boards
 from a large variety of manufacturers, hence the selection process
 becomes particularly time consuming. Quality evaluation would also
 be necessary\footnote{\emph{Due to the scale of production in this
 market, there is a great chance to obtain a device with a failure.}},
 when looking at \emph{x86} devices. Although, there are boards that
 would match the low-power and small form-factor requirements, most
 of these are with a handful of peripherals, for example low quality
 audio and video interfaces and, if these are not present, the board
 may have two or more RS-232 connectors. Another aspect of \emph{x86}
 system design is that it comes with legacy \emph{BIOS} technology,
 while most of \emph{ARM} systems, for example, have rather more
 flexible facilities for booloader and set-up. Also in the class of
 \emph{single board computers (SBC)} there are many boards which do
 not inlude physical connectors for above mentioned peripherals,
 however isuch boards are designed for industrial use require a
 specialised chasis \cite{links:linuxfordevices:guide}.

  Nevertheless, an \emph{x86} machine has been used for most of the
 development work on this project, that is for a rather transparent
 reason.

%% add to bibtex file:
% http://www.linuxfordevices.com/c/a/Linux-For-Devices-Articles/Single-Board-Computer-SBC-Quick-Reference-Guide/

\subsection{OMAP}

  The \emph{OMAP} application processors form \emph{TI} has become
 popular in the open-source community, and in fact \emph{TI} promotes
 Linux and Android as most suitable operating systems for this platform.
 The \emph{OMAP} architecture is based around an \emph{ARM} processor
 (there are various models with different versions of \emph{ARM} core)
 and \emph{TI C64x} DSP block. However, there is a little problem
 associated with how the DSP unit is integrated within the SoC.
 Very limited documentation is provided on how it can be used in the
 \emph{Linux} environment \cite{ti:omap:wiki:dsp}.

 One most outstanding development platform that uses an \emph{OMAP}
 chips with a dual-core \emph{Cortex-A8} CPU clocked at 1GHz - is the
 \emph{PandaBoard} \cite{ti:omap:wiki:pb}. However, it is available
 for back-order only, the maker is producing these boards on demand
 and the lead time would be at least one month. Otherwise this board
 would have been purchased despite the fact that DSP unit would be
 difficult to utilised.

 %http://www.omappedia.org/wiki/DSPBridge_Project
 %http://www.omappedia.org/wiki/PandaBoard

\subsection{Other ARM CPUs}

  \emph{ARM} CPUs are produced by almost any major semiconductor firm,
 so do \emph{Freescale} and \emph{Marvell}. Not all of those companies
 make very high performance \emph{ARM} chips, i.e. clocked near to 1GHz,
 and few make multi-core CPUs. And only some CPUs have an FPU.
 \emph{ARM} floating point instruction set is know as \emph{VFP}
 ("Vector Floating Point"). There are different versions of \emph{VFP},
 though in this context this wasn't a concern.

\subsubsection{Marvell Sheeva}

  \emph{Marvell} has originally started offering \emph{ARM} processors
 since their purchase of Intel's \emph{PXA} devision. Now \emph{Marvell}
 produces a series of high-performance \emph{ARM} chips, most of which
 are multi-core. \emph{Sheeva} is the brand name for these SoC devices.
 The clock frequency ranges from 800 MHz to 1.6 GHz, gigabit Ethernet
 is one of the outstanding features, since \emph{Marvell} specialises
 in the networking and storage IC market. As mentioned above, most
 important feature of \emph{ARM} chips that is necessary for the design
 of DSP host for this project is \emph{VFP}. Some of \emph{ARMADA}
 SoCs include \emph{VFP}, namely \emph{ARMADA XP} series,
 \emph{ARMADA 510} and \emph{610} \cite{links:marvell:armada}.
 \emph{Kirkwood} series are not featuring \emph{VFP}, thought there
 are some outstanding embedded development platforms available.

  \emph{Plug Commputer} (also know as \emph{Seeva Plug}) is small, yet
 very powerful computer in a form-factor of wall socket DC adaptor
 \cite{links:marvell:plug,links:plugcomp:homepage}. There many exciting
 application where these devices could be the best fit, thought due to
 above mentioned lack of \emph{VFP}, this CPU is not well suited as the
 DSP host.

% Marvell Semiconductor, Inc. - ARMAD Processor Family
% http://www.marvell.com/products/processors/armada.html
% http://www.marvell.com/platforms/plug_computer/
% http://www.plugcomputer.org/

\subsubsection{Freescale i.MX}

 \emph{Freescale Semiconducters} offers a range of \emph{ARM} processors
 branded emph{i.MX} \cite{links:freescale:imx}. The datasheets had been
 examined to find out which chips include floating point unit. Among the
 mid-range, \Chip{i.MX31} devices (532 MHz ARM1136JF-S) appear to have
 support for floating point instructions (\emph{VFP}), however the clock
 rate of this SoC would suite only a limited number of applications
 \cite{links:freescale:imx31}. In the upper-range of \emph{Freescale i.MX}
 solutions, there single-core \emph{Conrtex-A8} devices with clock rates
 up to a GHz \cite{links:freescale:imx5}, namely the \Chip{i.MX535}  and
 \Chip{i.MX538}. The second SoC has better video coding and storage
 capabilities.

 I has been noted that a simple development board for \Chip{i.MX535} can
 be purchased for a relatively low price\foonote{\emph{There two basic
 options: one for 150 US dollars, and another for 200 with a touch-screen.}}.

@MISC{links:freescale:imx,
	TITLE = "\href{http://cache.freescale.com/files/32bit/doc/brochure/FLYRIMXPRDCMPR.pdf}
		{{i.MX Product Family Overview}}",
	AUTHOR = {{Freescale Semiconductors}}
}


@MANUAL{links:freescale:imx31,
	TITLE = "\href{http://www.freescale.com/files/32bit/doc/ref_manual/MCIMX31RM.pdf}
		{{i.MX32 Reference Manual}}",
	AUTHOR = {{Freescale Semiconductors}}
}

@MANUAL{links:freescale:imx5,
	TITLE = "\href{http://cache.freescale.com/files/32bit/doc/fact_sheet/IMX5CNFS.pdf}
		{{i.MX5 Fact Sheet}}",
	AUTHOR = {{Freescale Semiconductors}}
}


\subsection{SHARC}

  \emph{Analog Device} is a well-known manufacturer of DSP chips, this
 architercture is different from \emph{TI C64x} and the signal processing
 instructions are handled directly on core. The clock frequencies are
 not being the main performance factor for the \emph{SHARC} device and
 range between 300 and 600 MHz. Linux has been a popular platform for
 advanced \emph{SHARC}-based systems appearing in various application
 markets \cite{links:adi:sharc}. It should noted that the advertisment
 of \emph{SHARC} chips provides less details then products from other
 vendors, hence an extensive evalution would be required for designing
 a system with \emph{SHARC}.

% http://www.analog.com/static/imported-files/product_highlights/SHARC_Proc_Family_(B)_Final.pdf

  \emph{SRARC} processors are commonly used in professional audio
 equipment. One very outstanding product has been released by
 a new British company \emph{Dark Matter Audio}, the device called
 \emph{DMA1} \cite{links:dma1} has been released this year.
 It is an advanced system for interactive music performance with
 very novel design features and powerful \emph{SHARC} DSP engine.
 According to the product announcement it runs Linux OS and also
 uses an \emph{ARM} chip for graphical user interface. It would be
 possible to integrate \emph{DMA1} with current prototype system
 developed in the course of this project. The vendor has provided
 an SDK, thought a request has been sent to obtain more specific
 details needed to design this add-on solution appropriately.


  \emph{SHARC} development hardware is mostly available from the
 silicon vendor itself and very few third-party boards could be found.
 Since this architecture is more specialised, the prices of the 
 development boards are not as low as some of the \emph{ARM} boards,
 although the vendor offers a comprehensive free library of board
 design file \cite{links:adi:freepcb}.

% http://blackfin.uclinux.org/gf/project/stamp/frs


\section{Building Embedded Linux OS}

\subsection{Background}

  Until quite recently, it had been rather more difficult to achieve
 the task of building (custom) embedded Linux system. Traditionally,
 the engineer who desired to do so, would need to follow instructions
 provided in the reference book \emph{"Linux from Scratch"}, commonly
 known as \emph{LFS} \cite{book:lfs}. This text provides details on
 how to utilise various tools for building an embedded Linux kernel
 and the file system from source, it discusses how to tweak various
 features at build time and configure appropriate runtime services.

\subsection{Build Utilities}
 
  More recently, a number of projects emerged, which do a great job
 of extending the flexibility by automating some of the simpler
 tasks, e.g. package dependency tracking and build version control
 (enabling the maintainer to revert to previous builds).
 One of the most outstanding and widely used projects in this areas
 is \emph{OpenEmbedded} \cite{links:oe} it has recently been adopted
 by a major software company \emph{Mentor Graphics}. \emph{Mentor
 Embedded Linux} extends \emph{OpenEmbedded} framework with a set
 of tools which are useful in a large-scale development project
 \cite{links:mentor:linux}.
 
  There are a few alternative approaches which lead to similar results,
 one may wish to use \emph{Gentoo Linux} meta-distribution, there
 is only one source of documentation - \emph{"Gentoo Linux Embedded
 Handbook"} \cite{links:gentoo:embedded}. Generally, \emph{Gentoo}
 framework (named {\emph{Portage}) has all necessary components
 which a system designer may need and there is no big difference
 between \emph{Gentoo} and \emph{OpenEmbedded}. Many internals
 are known to be very similar, tough \emph{Gentoo} doesn't provide
 some specialised tools which are provided in \emph{OpenEmbedded},
 hence \emph{Gentoo} has been originally designed for custom
 desktop and server systems. Major aspect which is different is
 that \emph{Gentoo} is rather bound to package releases, while
 \emph{OpenEmbedded} is more flexible when a development revision
 number has to be specified for a certain source code package.
 The conclusion is that \emph{OpenEmbedded} is most likely to be
 more convenient to use for an embedded target.
 
  A little different approach can be taken with \emph{BuildRoot}
 \cite{links:buildroot:homepage}, which is a build system for
 embedded Linux. Based around the same framework that was design
 for configuring the Linux kernel builds. This tool is a set of
 scripts controlled via a console menu. Although, it appears to
 have a simple to use front-end, the underlying configuration
 system is certainly behind the competition with \emph{OpenEmbedded}.

  A project which have a slightly different orientation is
 \emph{ScratchBox} \cite{links:sbox:homepage}. It provides
 a cross-compilation toolkit for application developers.
 It can be used to provide an software development kit (SDK)
 for a 3rd-party developer with appropriate toolchain and
 hardware emulator. However, \emph{OpenEmbedded} includes
 support for building SDK and \emph{ScratchBox} is rather
 limiting in various ways, i.e. it is a static distribution,
 rather then a build system.
 
  Another project in this area of interest is \emph{Linaro Foundation}
 \cite{links:linaro:about}, the initiative has been originated by
 London-based \emph{Canonical Limited} (the company behind now most
 popular Linux distribution - \emph{Ubuntu}).
 The purpose of \emph{Linaro} is to improve various issues related to
 embedded Linux and provide a higher grade platform for Android and
 other multimedia and consumer oriented distributions. \emph{Linaro} is
 still an emerging project and has not produced any significant output
 \cite{links:linaro:homepage}. It has specific orientation, in terms of
 hardware, being currently involved only with \emph{ARM} platforms,
 and in terms software, it is bound rather strictly to \emph{Ubuntu}
 methodology.

\subsection{Conclusion}

  It had been desired to build an embedded DSP host OS as one of
 additional project targets, however due to various\footnote{\emph{
 Another aspect was due to hardware selection and purchasing issues.}}
 limitations, most importantly - the time frame, the project has not
 achieved this at the time of writing of the final report. The above
 section was included to provide the evidence of research in this area.

% Mentor Embedded Linux - Technical Brief
% by Mentor Graphics Corporation
% http://www.mentor.com/embedded-software/upload/embedded-linux-brief.pdf

% OpenEmbedded - User Manual
% http://docs.openembedded.org/usermanual/usermanual.pdf

% BuildRoot - Home Page
% http://buildroot.uclibc.org/

% Linaro - Foundation Home Page
% http://www.linaro.org/about-linaro/

% ScratchBox - Project Home Page
% http://www.scratchbox.org/

% Gentoo Linux - Embedded Handbook
% http://www.gentoo.org/proj/en/base/embedded/handbook/


\chapter{Application Development}
\section{Working with Contiki}
  
  After an appropriate compiler toolchain and debugger packages for \MCX\
 had been installed on the development host, a few steps were taken to
 simplify the work-flow in \Contiki\ application development environment.
 It may be noted here, that an integrated development environment (IDE)
 could be used and some programmers do prefer to use an IDE, such as
 \emph{Eclipse} \cite{links:contiki:wiki:eclipse}, nevertheless the
 command line tools are known to be the most efficient approach.
 It should be noted that this section is rather brief description of what
 has been done and was not intended to provide a detailed guidance on how
 to reproduce the results.

  Apart from the revision control tools\footnote{\emph{This project used
 git system, however the details on how that has been done are considered
 to be irrelevant to the subject of this report}} and the GCC toolchain
 for ARM \cite{links:mc1322x:gcc}, there are four key command-line tools
 which were utilised during the development process.

\begin{description}
	\item [\MAN{gdb}] - GNU Debugger (source-level) \cite{docs:gdb:manual}
	\item [\emph{\texttt{OpenOCD}}] - On-chip Debugger \cite{links:mc1322x:ocd}
	\item [\emph{\texttt{Vim}}] - a text editor
	\item [\MANX{1}{make}] - GNU make program \cite{docs:make:manual}
\end{description}

% http://mc1322x.devl.org/eclipse.md
% http://www.sics.se/contiki/wiki/index.php/Setting_up_Eclipse_for_Contiki_Development

% http://mc1322x.devl.org/openocd.html

% http://mc1322x.devl.org/toolchain.md

% http://www.gnu.org/software/make/manual/make.pdf
% http://www.gnuarm.com/pdf/gdb.pdf

  To enhance the work-flow \emph{"Makefiles"}\footnote{\emph{These are
 the file which specify a set of rules for the make program on how to
 compile the source code and also perform administrative tasks and run
 debugger or other tools.}} were amended throughout the development
 process. Generally there is one \emph{Makefile} in each subdirectory
 of the source tree, thought most of these inherit rules specified in
 the main \emph{Makefile} (in \Contiki\ there are two of these - one
 at in the root directory and one for each processor architecture).

 Most of the changes were made in \Blame{cpu/mc1322x/Makefile.mc1322x}
 to provide a variety of short command for connecting the debugger
 and sending the program to run on a development board\footnote{
 \emph{For example, to set-up the WPAN router on the Freescale board - run
 \texttt{`make.f1 router'} and to compile 'example.c', load and print
 serial output on the console for the Econotag  -
 \texttt{`make.e1 example.load-print'}; in case if 'example.c' does
 not behave as expected - run \texttt{`make.e1 example.ocd-screen'}.}}
 
\section{Application Prerequisites}

  The first step in application development was to add a driver for
 the second UART (UART2) port. This has been done by copying an
 existing driver code for UART1, though enhancements were required
 at a later stage. The history of changes to the code can be viewed
 in at the repository by utilising the commit log filter. The files
 shown here had been modified.

 \begin{itemize}
 \item \Contrib{cpu/mc1322x/lib/include/uart1.h} \\
 	normal and weak prototypes, register pointers and macros
 \item \Contrib{cpu/mc1322x/lib/uart2.c} \\
 	driver interrupt handlers
 \item \Contrib{cpu/mc1322x/src/default\_lowlevel.h} \\
	prototypes
 \item \Contrib{cpu/mc1322x/src/default\_lowlevel.c} \\
	initialisation functions
 \end{itemize}



\chapter{State Machines}
\begin{tikzpicture}[>=latex, shorten >=1pt, node distance=1in,
			on grid, auto, initial text=\fbox{\tt{Start}}]


\node [E]
(run)
	{\texttt{Listener Initialised}};

\node [C, below of =run, minimum width=10em]
(adv)
	{\emph{Wait for Talker Broadcast}};

\node [E, left of =adv, node distance =6cm]
(wup)
	{\texttt{Wake-up}};

\node [F, below of =adv, text width=10em]
(rep)
	{Attempt to Reply\\to the Advertisement};

\node [C, below of =rep]
(req)
	{\emph{Request Fulfilled?}};

\node [F, below of =req, text width=8em]
(con)
	{Establish Connection};

\node [C, below of =con]
(buf)
	{\emph{Wait for Data Packets}};

\node [F, below of =buf, text width=12em]
(use)
	{\emph{\underline{Submit Data to}\\\underline{the Processing Thread}}};

\node [E, right of =adv, node distance =6cm]
(int)
	{\texttt{Connection Failure}};

\node [F, left of =rep, node distance =3.5cm, text width=7em]
(del)
	{\emph{Decreasing\\Exponential\\Sleep Delay}};

\coordinate [right of =use] (ret);

\path[L, dashed] (run) -- (adv);

\path[L, dashed] (int) -- (adv);

\path[L, dashed] (wup) -- (del);

\path[L] (adv) -- node {Yes, Talker is Present} (rep);

\path[L] (rep) -- (req);

\path[L] (req) -- node [near start] {No, Request Failed} (del);

\path[L] (del) |- (adv);

\path[L] (req) -- node {Yes, Ready to Connect} (con);

\path[L] (con) -- (buf);

\path[L] (buf) -- node {Received} (use);

\path[L] (use) |- (ret) |- node [near end] {Return} (buf);

\end{tikzpicture}

\begin{tikzpicture}[>=latex, shorten >=1pt, node distance=1in,
			on grid, auto, initial text=\fbox{\tt{Start}}]


\node [E]
(run)
	{\texttt{Talker Initialised}};

\node [E, left of =run, node distance =6cm]
(wup)
	{\texttt{Wake-up}};

\node [F, below of =run, text width=8em]
(adv)	
	{Broadcast Advertisement Messages};

\node [C, below of =adv, text width=6em]
(rep)
	{\emph{Wait for a Connection}};

\node [F, below of =rep, text width=12em]
(con)
	{Connect to the Listener};

\node [C, below of =con]
(buf)
	{\emph{Wait for the Data Buffer}};

\node [F, below of =buf, text width=10em]
(now)
	{\emph{\underline{Poll the Stack Thread}\\\underline{to Send Data Now}}};

\node [E, right of =adv, node distance =6cm]
(int)
	{\texttt{Connection Failure}};

\node [F, left of =adv, node distance =3.5cm, text width=7em]
(del)
	{\emph{Increasing\\Exponential\\Sleep Delay}};
\coordinate [right of =now] (ret);

\path[L, dashed] (run) -- (adv);

\path[L, dashed] (int) -- (adv);

\path[L, dashed] (wup) -- (del);

\path[L] (adv) -- (rep);

\path[L] (rep) -- node {Yes, Request Received} (con);

\path[L] (rep) -- node [near start] {No, Listner is not Present} (del);

\path[L] (del) |- (adv);

\path[L] (con) -- (buf);

\path[L] (buf) -- node {Ready} (now);

\path[L] (now) |- (ret) |- node [near end] {Refill} (buf);

\end{tikzpicture}


\chapter{System View}

\section{Key System Components: \\ Specification \& Requirements} \label{sec:SPECS}

  This section outlines the specification which had been proposed
 after initial research phase has been accomplished. Diagram shown
 in figure \ref{fig:sketch1} illustrates the function blocks of the
 complete system that meets this specification, however it is now
 considered suitable only for the development system and not the
 final product design.

\subsection*{Sensor Node}

\emph{Sensor Node} is a device featuring a microcontroller chip
that contains following function blocks and peripherals:

\begin{itemize}
	\item \emph{internal:}
	\begin{itemize}
		\item 2.4Ghz RF transceiver,
		\item A/D converters,
		\item GPIO,
		\item SPI, and
		\item USART
	\end{itemize}
	
	\item \emph{external:}
	\begin{itemize}
		\item digital or analogue sensors,
		\item external connectors for MIDI,
		\item wired remote sensors, and
		\item serial port header
	\end{itemize}
\end{itemize}

The \emph{Sensor Node} runs software which consists of:

\begin{itemize}
	\item \emph{operating system:}
	\begin{itemize}
		\item communication protocol stack,
		\item devices drivers, and
		\item application task management,
		\item service tasks, and
		\item the main program
	\end{itemize}
	\item \emph{bootloader:}
	\begin{itemize}
		\item loads new software
	\end{itemize}
\end{itemize}

\begin{figure}
\centering
\includegraphics[scale=0.4]{../figures/sketch1.pdf}
\caption{\emph{Abstract Sketch of the Development System}} \label{fig:sketch1}
\end{figure}


\pagebreak
\section{Aspects of Final Product Design}

   Many commercial uses of \WPAN in the field of audio control were
 considered. The block diagrams in figure \ref{fig:products} briefly
 illustrate a few interesting solutions. Some of the device proposed
 here can be used in a variety of applications and others are rather
 specific to live sound and stage performance.

  Current implementation can be used to some extend, however one
 very important modification of hardware needs to be considered.
 The connectivity between the host and border router, in current
 development system, is implement via the USB serial interface.
 This involves unnecessary hardware and software components and
 should eliminated from a commercial design.
  One Solution would to utilise SPI bus, which is available on
 most of the SoCs. A transceiver can be connected directly to
 the host SoC and there will need to be a Linux driver for it.
 The second solution is to implement an abstract Linux/Contiki
 framework, such that would use SPI for the datapath and GPIO
 for the interrupts. In this way, a very robust device can be
 designed (see the \emph{"Wired Gateway Board"} diagram in figure
 \ref{fig:products}).

  Another important idea presented in \ref{fig:products}, is the
 \emph{"Tiny Node"}. It proposes a general-purpose boar, which
 would suite a wide range of application. Provided that an
 appropriate physical connection is defined, this board may
 have removable sensor part for each of the possible uses.

\begin{figure}
\centering
\includegraphics[scale=0.8]{../../poster/figures/figure1.pdf}
\caption{\emph{Block diagrams of some possible commercial solutions}} \label{fig:products}
\end{figure}


%%%%%%%%%%%%%%%%%%%%%%%%%%%%%%%%%%%%%%%%%%%%%%%%%%%%%%%%%%%%%%%%%%%%%%%%

\bibliographystyle{acm}

\bibliography{../main,../../contiki/dunkels}

\end{document}
