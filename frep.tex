\newcommand{\Chip}[1]{\emph{\texttt{#1}}}
\newcommand{\Symb}[1]{\emph{\texttt{#1}}}
\newcommand{\MCX}[0]{\Chip{MC1322x}}
\newcommand{\RFA}[0]{\Chip{ATmega128RFA1}}
\newcommand{\Port}[1]{\emph{\texttt{Port #1}}}
\newcommand{\Comp}[1]{\emph{\texttt{#1}}}
\newcommand{\WPAN}[0]{\emph{LR-WPAN}}

\newcommand{\TrackerList}[0]{\subsubsection{\emph{Tracked Issues}}\begin{description}}
\newcommand{\TrackerEnd}[0]{\end{description}}

\newcommand{\Contiki}[0]{\emph{Contiki}}
\newcommand{\ContikiOS}[0]{\emph{Contiki OS}}
\newcommand{\TinyOS}[0]{\emph{TinyOS}}


\documentclass[a4paper]{report}

\usepackage{multirow}
\usepackage{graphicx}

%02:36 <kahrl> errordeveloper: http://physics.wm.edu/~norman/latexhints/conditional_macros.html
%01:37 <kahrl> so \makeatletter \ifx \PRINTING \@empty \renewcommand{\href}{FAKE_VERSION_OF_HREF} \fi \makeatother
%01:39 <kahrl> or \makeatletter \ifx \PRINTING \@empty \relax \else \renewcommand{\href}{FAKE_VERSION_OF_HREF} \fi \makeatother

% `latex '\def\NOHREF{}\input{foo.tex}'`

\ifdefined \NOHREF

	\usepackage{url}
	\newcommand{\URL}[1]{\[ \texttt{\emph{#1}} \]}
	%% \newcommand{\href}[2]{#2 (\texttt{\emph{\url{#1}}})} %% not needed probably
	
	\newcommand{\Href}[2]{{#2}} %% fake version for printing
\else

\usepackage{hyperref}
\newcommand{\Href}[2]{\href{#1}{#2}} %% use this for proper href
\newcommand{\URL}[1]{\[ \Href{#1}{\texttt{\emph{#1}}} \]}

\fi



%% use \Href  and for printing we use \cite{something}
%% so we end up with link to the URL and bibliography
%% reference number as well. If \Href is fake there
%% will be no URL in the text.
\newcommand{\AltRef}[3]{\Href{#2}{#1} \cite{#3}}



\newcommand{\Issue}[1]{\Href{http://wmi.new-synth.info/issues/#1}{Issue \##1}}
\newcommand{\IssueX}[1]{\item[\Href{http://wmi.new-synth.info/issues/#1}{\Issue{#1}}]:}

\newcommand{\TEP}[1]{\AltRef{\emph{TEP#1}}{http://www.tinyos.net/tinyos-2.x/doc/pdf/tep#1.pdf}{links:tinyos:tep#1}}

\newcommand{\RFC}[1]{\AltRef{\emph{RFC#1}}{http://tools.ietf.org/pdf/rfc#1.pdf}{ietf:rfc#1}}

%% usage:
%	\MAN{gcc}
%	\MAN{2, open}

%\newcommand{\MAN}[1]{\AltRef{\emph{\texttt{#1}}}{http://manpages.debian.net/cgi-bin/man.cgi?query=#1&sektion=0&format=pdf}{doc:linux:man:0:#1}}
%\newcommand{\MANX}[2]{\AltRef{\emph{\texttt{#2}}}{http://manpages.debian.net/cgi-bin/man.cgi?query=#2&sektion=#1&format=pdf}{doc:linux:man:#1:#2}}

\newcommand{\MAN}[1]{\AltRef{\emph{\texttt{#1}}}{http://manpages.debian.net/cgi-bin/man.cgi?query=#1&sektion=0&format=ascii}{doc:linux:man:0:#1}}
\newcommand{\MANX}[2]{\AltRef{\emph{\texttt{#2}}}{http://manpages.debian.net/cgi-bin/man.cgi?query=#2&sektion=#1&format=ascii}{doc:linux:man:#1:#2}}

%% For this commands we need to use \Href explicitly
%% so if Href is proper - the display the path and link it
%% otherwise - just display the path. Because there's no
%% reason to show various refences to this in the bibtex

\newcommand{\Blob}[1]{\Href{https://github.com/errordeveloper/contiki-wmi/blob/wmi-work/#1}{\texttt{#1}}}
\newcommand{\Tree}[1]{\Href{https://github.com/errordeveloper/contiki-wmi/tree/wmi-work/#1}{\texttt{#1}}}
\newcommand{\Blame}[1]{\Href{https://github.com/errordeveloper/contiki-wmi/blame/wmi-work/#1}{\texttt{#1}}}
\newcommand{\Logs}[1]{\Href{https://github.com/errordeveloper/contiki-wmi/commits/wmi-work/#1}{\texttt{#1}}}
\newcommand{\Contrib}[1]{\Href{https://github.com/errordeveloper/contiki-wmi/commits/wmi-work/#1?author=errordeveloper}{\texttt{#1}}}

\usepackage{tikz}
\usetikzlibrary{automata,positioning,shapes,arrows}

\tikzstyle{C} = [diamond,
			draw, fill=blue!20,
			text width=4.5em, text badly centered,
			node distance=3cm, inner sep=0pt]

\tikzstyle{F} = [rectangle,
			draw, fill=green!20,
			text width=5em, text centered,
			rounded corners, minimum height=2.5em]

\tikzstyle{E} = [ellipse,
			draw, fill=red!20,
			node distance=3cm, minimum height=3em]

\tikzstyle{L} = [draw, -latex']

\usepackage{pstricks-add}

%\DeclareGraphicsExtensions{.eps,.epsi,.pdf,.png,.jpg}

\title{WMI: Wireless Music Instruments\\ \emph{BSc Final Year Project} \\ Final Report}
\author{Ilya Dmitrichenko \\ \\ Department of Computing\\ London Metropolitan University}

\date{\today}



\begin{document}
\maketitle


\abstract
{

   This is the end of year report for a BSc degree project which had originally
   targeted a design of a complete wireless system for streaming of music control
   data collected from sensor devices and MIDI hardware to a host machine which
   would synthesise an audio signal to be delivered into loudspeakers or recorded
   on disk. Despite rather simple design idea, the research and development had
   taken extended time period and the emphasis of this report is on a variety of
   aspects related to wireless sensor network design and implementation. These
   aspects include: cross-platform operating system architecture, development
   hardware, application design and the abstraction layer requirements for
   interfacing hardware functions to the software as well as other miscellaneous
   subjects intersecting these areas. Two popular operating systems are presented
   in this report - \emph{Contiki} and \emph{TinyOS}. Two common microcontroller
   architectures had been used in the course of this project - 32-bit \emph{ARM}
   and 8-bit \emph{AVR}. \emph{IEEE 802.15.4} devices where considered as the
   only suitable radio harware option, due to the wide use and availability of
   the chips which implement this protocol. Another key assumption has been that
   the complice to the Internet Protocol (IP) is neccessary. A simple prototype
   has been implemented and a set of solutions proposed for a complete product
   design. Several advanced subjects for further work are also addressed in this
   report.

}

\tableofcontents
\listoffigures
%\listoftables

%%%%%%%%%%%%%%%%%%%%%%%%%%%%%%%%%%%%%%%%%%%%%%%%%%%%%%%%%%%%%%%%%%%%%%%%

\part{Introduction \& Research}
\chapter{General Background}

\section{Motivation}

  Recent advances in low-power microcontroller and radio frequency data communications
 technology, together with enormous growth in the general purpose embedded computer
 market challenges several industries around the world. However, the music instrument
 industry haven't fully taken all the advantages of the most recent electronic devices
 and standards.
 
 Wireless sensor networks will very soon appear in numerous application areas,
 which may include practically any market (consumer, industrial, medical and many other).
 
 This project aims to implement a networked system for stage performance which uses
 wireless sensor devices as a creative user interface. Despite the title, this also
 concerns any theatrical or dance performance as well as various monitoring and
 control uses in entertainment systems.



\section{Organisation}

  The very initial research and preparation phases of this project had taken place
 prior the start of the academic year \emph{2010/11}. When the project has started
 in October, one of the first steps taken was a creation of website, which provides
 a number of important facilities for project management, file keeping and issue
 tacking. The website can be found at the URL below:
 {\URL{http://wmi.new-synth.info}}
 It will be referred to throughout this report as \emph{The WMI Website}. A current
 source code repository is also available on-line and linked to the website as well
 as work in progress notes and other supplementary information and data.

\subsection{Report Structure}

\subsection{Typographic Conventions}

  Unfamiliar names and acronyms mentioned in the report are typeset in \emph{italic},
 as well as vendor brands when referred to a product from that vendor. Names of device
 models as well as commands and programming language keywords or statements are typed
 in \emph{\texttt{bold-italic}}. Filenames are \texttt{mono-spaced} and in the PDF
 version of the document are hyper-linked to the source code repository\footnote{%
 \emph{The path is aways relative to the source code root directory, unless specified
 otherwise.}}. There will appear a special sub-section titled \emph{"Tracked Issues"}
 in some sections, it shows reference to issue tickets on the website\footnote{\emph{%
 The tickets are all enumerated in one sequence starting from 1 and classified by the
 type of issue each ticket relates to (i.e. "bug", "feature", "task"). Best effort
 was made to file these issue and there are very few remaining undocumented.}}.
 The tickets can be accessed via URL of the following form (replace the \Symb{`\#'}
 symbol with the given number): \URL{http://wmi.new-synth.info/issues/\#}

\chapter{Field \& Market Trends}

 This chapter intends to review the variety of available
 technologies and discuss why some desitions and preference
 had been made.

 \emph{The terminology of the OSI 7-layer model will be used,
 therefore table \ref{tab:osi} will illustrate the stacking
 of all layers and in comparison to the Internet Protocol
 model (presuming that the reader is already familiar with
 this terminology, hence it is for only for reference).}

\begin{table}[h]
\begin{center}
 \begin{tabular}{|c||c|}
 \hline
 {\bf OSI model}&{\bf IP model}\\
 \hline \hline
 \ \ \  Application \ \ \   & \multirow{2}{*}{ \ \ \   Application \ \ \ }   \\
 \cline{1-1}
 Presentation & \\
 \hline
 Session & \multirow{2}{*}{TCP, UDP,\  \dots }\\
 \cline{1-1}
 Transport & \\
 \hline
 Network & IP\\
 \hline
 Data Link & Data Link\\
 \hline
 Physical & Physical\\
 \hline
 \end{tabular}
 \end{center}
\caption{\emph{Open System Interconnection Model
	and The Internet Protocol}}\label{tab:osi}
\end{table}


\section{Music Control Protocols}

   One of original targets of motivation to work on low-power
  wireless network for music control, had been a desire to
  experiment in the area of high-level representation of control
  data specific to live music performance and audio in general.
  Started by considering to extend not so recently proposed OSC
  ("Open Sound Control") protocol \cite{paper:osc11} with a
  set features which it appears to be missing. This is a subject
  to more extensive experimentation and therefore has not been
  included in the scope of the project itself.

   OSC is effectively only an application layer data format and
  is mostly used with UDP. It had been proposed by a group of
  researchers at the Centrer for New Music \& Audio Technologies
  (CNMAT), UC Berkley \cite{links:cnmat}. It was first presented
  in 1997 \cite{paper:osc97} and has received a rather limited
  adoption. Although, it was observed that there is a tendency
  towards a wider adoption of OSC - a number of interesting
  hardware product had been released which use OSC as primary
  protocol. Some of these devices are listed below.

  \begin{itemize}
  	\item \emph{Livid Block64 \cite{links:livid:block64}}
	%http://www.lividinstruments.com/hardware_ohm64.php
	\item \emph{Jazz Mutant Lemur \cite{links:jazzmutant:lemur}}
	%http://www.jazzmutant.com/lemur_overview.php
	\item \emph{Monome \cite{links:monome}}
	%http://monome.org
	%http://docs.monome.org/doku.php?id=tech:protocol:osc
  \end{itemize}

  It has to be said that OSC seems to be intended as a candidate
  to replace the MIDI protocol (which had been define in 1983 and
  therefore considered in need of a replacement). The greatest
  limitation seen in MIDI today is the size of values it can
  represent, i.e. most control values a bound in 0-127 range.
  Nevertheless, wireless transmission of MIDI was chosen to
  be the initial target for this project. To avoid going
  outside of the scope of this report, it shall be defined
  that MIDI is quite likely to be most appropriate to implement
  at this stage, since OSC format is certainly not suitable.
  The reason for this is that microcontrollers do not have
  the capabilities to handle large amount of data represented
  as character strings. Therefore a new protocol needs to be
  designed, which would overcome most these limitations; though,
  that is already in the scope of another project.

%http://cnmat.berkeley.edu/

%http://archive.cnmat.berkeley.edu/ICMC97/postscript/0118030b.org-OSC-paper.ps

\section{Low-power Digital Radio}

  This section gives an overview of currently available low-power
 wireless communication technologies in general terms, then some of
 the important concepts are briefly introduced to support further
 discussion of these devices and software with appropriate terminology.

  One of the main subjects of the initial research was wireless data
 communication standards for sensing and control applications and it
 should be noted that the transmission of audio signals has not been
 taken into consideration. Also as outlined in the requirenments,
 radio protocols such as UWB (e.g. WUSB, WiMAX) and \emph{IEEE 802.11}
 (i.e. WiFi) are not low-power and therefore are not applicable for the
 purpose of this project. Although, short-range\footnote{\emph{Recent
 amendments in P802.15.4a specify alternative physical layer options
 that include sub-gigahertz UWB modes \cite{links:wiki:p:wpan}.}} UWB
 could be of great use for its potential throughput capabilities, the
 cost and availability of transceivers are yet unknown.
 
  The main interest is in low-power radio of 2.4GHz range. The semiconductor
 market is currently flooded with a variety of inexpensive devices that
 implement either \emph{IEEE 802.15.4} standard or patented protocols such
 as \emph{ZigBee}, \emph{RF4CE} or other that are based on \emph{P802.15.4}.

 A few more different low-power wireless networking technologies exist, such
 as \emph{DASH7} and \emph{ANT}. \emph{DASH7} is an active RFID protocol for
 extended range and it is very specific for certain applications, it operates
 in 434MHz band \cite{links:wiki:p:dash7}. \emph{ANT} is a proprietary standard
 which uses 2.4GHz band \cite{links:wiki:p:ant}. Both of these technologies
 are support only by small groups of silicon chip vendors.

 Multiple standards exist which are using the same hardware functions provided
 by \emph{P802.14.5}-compliant devices\footnote{\emph{This implies that all of
 these protocols are effectively defined only by software.}}, some of these are
 \emph{ZigBee}, \emph{RF4CE} and \emph{6loWPAN} \cite{links:wiki:p:6lowpan}.
 \emph{RF4CE} belongs to the \emph{ZigBee} \cite{links:wiki:p:zigbee} family
 together with a number of other application area specific variations.
 \emph{6loWPAN} \cite{links:wiki:p:6lowpan} is most interesting patent-free
 protocol and it is transparent to existing software, since it is compliant
 to the Internet Protocol. %\IETF

  Nevertheless, all of these technologies are not yet widely available in
 the consumer market, where \emph{BlueTooth} \cite{links:wiki:p:mrwpan} and
 simplistic sub-gigahertz serial radio transceivers or infra-red are commonly
 found. Although, it is most likely that \emph{P802.15.4} transceivers will
 soon dominate low-power wireless application markets.

 Further in this report \emph{P802.15.4} will be referred to as \WPAN\ 
 (stands for Low-Rate\footnote{\emph{Medium-rate (MR-WPAN) are the BlueTooth
 (P802.15.1) networks. Also there is HR-WPAN (P802.15.3) standard defining
 high-rate (UWB) networks. All classes of WPAN together with WLAN are
 are referred to as short-range wireless networks.}} Wireless Personal
 Area Network). A number of amendments to \emph{IEEE 802.15.4} had been
 published, the latest version is \emph{P802.15.4-2006}. In this report
 it shall be looked at as a de-facto solution \cite{links:wiki:p:wpan,
 links:wiki:p:lrwpan, links:ieee:802:15}.

 \emph{Some general concepts of the \WPAN\ hardware described below
 are considered to be absolutely complete for understanding the
 system operation from the software design perspective.}

\subsection{\emph{IEEE 802.15.4} - Low-Rate WPAN}

  This standard was first proposed by the IEEE in 2003 and has
 evolved since. As far as the concepts essestial for application
 development are concerned, there is no major difference between
 the revisions.

 It is important to understand at this point that the concept of
 low-power consumption applies to all layers, so the application
 layer indeed is required to co-operate in order to preserve the
 energy. However, the task of this project is to get maximum
 throughput on \WPAN\ network and attempt to reduce the latency
 and maximize quality of service, hence power preservation
 techniques are considered very briefly. However, the low-power
 requirement for the design is met, since the avarage power rating
 of the device remains relatively low without applying these
 techniques.
 

 \WPAN\ defines two layers of the OSI model, the \emph{Physical (PHY)}
 and \emph{Media Access Control (MAC)} layers. Network and Applications
 layers are defined by other standards mentioned above.


\subsubsection{Physical Layer (PHY)}

\begin{flushright} \small{
\emph{Source: "ZigBee Wireless Networks and Transceivers" \\
	by Shahin Farahani (2008) \cite{b:zigbee}}}
	\end{flushright}


% This layer functions for transmission and reception of radio packets,
%provides control facilities for channel selection and power management.

 "The PHY layer is the closest layer to hardware and directly controls
 and communicates with the radio transceiver.
 The PHY layer is responsible for the following:
 \begin{itemize}

        \item Activating and deactivating the radio transceiver.

	\item Transmitting and receiving data.

	\item Selecting the channel frequency.

	\item Performing energy detection (ED).\\
	\small{\emph{The ED is the task of estimating the signal energy within the
	frequency band of interest. This estimate is used to understand
	whether or not a channel is clear and can be used for transmission.}}

	\item Performing Clear Channel Assessment (CCA).

	\item Generating a Link Quality Indication (LQI).\\
	\small{\emph{The LQI is an indication of the quality of the data packets
	received by the receiver. The signal strength can be used as
	an indication of signal quality.}}"
	\end{itemize}

 The \WPAN\ standard defines use
 for a number of bands in different geographical regions, the 2.4GHz
 band can be used anywhere in the world.  The details regarding RF
 modulation techniques and various regulations are outside of the
 subject of this report. There are 27 channels in different bands, 
 2.4GHz band has been assigned with channel numbers 11 to 26.

  Power regulations apply depending on geographical region, the
 measures are transceiver output power and duty cycle. The global
 ISM band can be utilized at 100\% duty on approximately 10mW level.

 Two modulation techniques can be used in 2.4GHz band:
 \begin{description}
 \item[\emph{Offset-QPSK}]- Offset Quadrature Phase-Shift Keying
 \cite{links:wiki:qpsk}
 \item[\emph{DSSS}]- Direct-Sequence Spread Spectrum (was used in 2003 revision)
 \cite{links:wiki:dsss}
 \end{description}

 Alternative modulations techniques are defined in amendment
 \emph{P802.15.4a}, these include Ultra-Wide Band \emph{(UWB)}
 and Chirp Spread Spectrum \emph{(CSS)}.

 Common basic data rate is 250kbps and distance coverage is from 10m
 to 100m, but higher limits can be achieved (up to 2Mbps).

\subsubsection{Media Access Control (MAC)}

\begin{flushright} \small{
\emph{Source: "ZigBee Wireless Networks and Transceivers" \\
	by Shahin Farahani (2008) \cite{b:zigbee}}}
	\end{flushright}

  "The MAC provides the interface between the PHY
 and the next higher layer above the MAC."

% {This layer is above physical and therefore it controls what is being
% transmitted and how it is done.
% Some of the most important concepts are:
%	\begin{itemize}
% 		\item Super Frames and Timeslots\\
%		\small{\emph{These provide mechanism for real-time data transmission.}
%		\item Encryption\\
%		\small{\emph{Most of commercially available devices use 128-bit AES}
%		\dots Beacon Frames\\
%		\small{\emph{Essentially beacons can be seen as advertisement messages.}

 "The IEEE 802.15.4 defines four MAC frame structures:
\begin{itemize}
       \item \texttt{Beacon Frame}
	\small{\emph{ --- used by a coordinator to transmit beacons.\\
       			The beacons are used to synchronize the clock\\
			of all the devices within the same network.}}
       \item \texttt{Data Frame}
	\small{\emph{ --- used to transmit data.}}

       \item \texttt{Acknowledge Frame}
	\small{\emph{ --- used to acknowledge the successful reception of a packet.}}
       \item \texttt{MAC Command Frame}
	\small{\emph{ --- are used for commands such as requesting the data\\
			  and association or disassociation with a network.}}

\end{itemize}

It is important to understand that frames from each network layer are
encapsulated into each other, i.e. the MAC frame are encapsulated into
PHY frames on transmission and on reception these data structure is
being decoupled.


\subsubsection{\WPAN\ Classification: Nodes and Topologies}

  The \WPAN\ device hierarchy is defined in terms of \emph{full-}
  and \emph{reduced-function} devices (FFD and RFD for short).
  
  \begin{itemize}
 	\item Routers (FFD) 
	\begin{itemize}
		\item Network Coordinator
		\item Branch Coordinator
		\item Border Router
	\end{itemize}
	\item Participant Clients (RFD or FFD)
  \end{itemize}

  \WPAN\ Topologies are:
  \begin{itemize}
  	\item Point-to-Point
	\item Star
	\item Mesh
  \end{itemize}

\subsubsection{Feature Outline}
\emph{Source: IEEE 802.15 Task Group 4 Website}\cite{links:ieee:802:15:4}.

\begin{itemize}
	\item Data rates of 250 kbps, 40 kbps, and 20 kbps.

	\item Two addressing modes: 16-bit short and 64-bit.

	\item Support for critical latency devices, such as joysticks.

	\item CSMA-CA channel access.

	\item Automatic network establishment by the coordinator.

	\item Fully hand-shaked protocol for transfer reliability.

	\item Power management to ensure low power consumption.

\end{itemize}

\subsection{Conclusions}

  It has been found that some of the concepts defined in the \emph{IEEE}
 paper are of very little practical use. For example, the hierarchy
 specification is only used in \emph{ZigBee} and does not apply to
 \emph{6loWPAN} and the entire MAC layer specification is disregarded
 by the \emph{6loWPAN} user community and many papers where presented
 which propose various improvements, for example \cite{papers:amac}
 (One most highly regarded papers presented at the ACM conference
 on Embedded Networked Sensor Systems in November 2010).

 It does appear that Media Access Control for \WPAN\ as well as IP
 Routing Over Low-power and Lossy networks (ROLL) \cite{rfc:drafts:roll:survey07}
 are subject to very extensive research at present, as well time
 synchronisation (to a lesser extend) \cite{paper:ts1,paper:ts2,
 paper:ts3}. The last subject is of greatest concert to the music
 instruments application, however it was found too complex to cover
 in the course of this project.

 %http://www.cs.berkeley.edu/~stevedh/pubs/p1-Dutta.pdf
 %http://tools.ietf.org/html/draft-ietf-roll-protocols-survey-07



\subsection{Higher Layers and the Application Layer}

  The greatest benefit that the OSI model gives, is that any layer
 can be re-implemented without any changes to other layers above or
 below. Also translation between different implementations would a
 trivial task. \emph{6loWPAN} allows for connectivity of low-power
 wireless nodes to the Internet, enabling the future paradigm of
 \emph{"the Internet of Things"}. This has been developed and
 popularised by an alliance of Internet Protocol for Smart Objects
 (IPSO) \cite{links:ipso:homepage}. \emph{6loWPAN} takes care of
 several aspects such as compute resources limits and address space
 requirements due to large amount of participant nodes by providing
 \emph{IPv6} addressing with additional features such as \emph{header
 compression}, reduced functionality as well as other miscellaneous
 enhancements \cite{pubs:ipso:wp3, pubs:ipso:wp1}.

  By enabling Internet connectivity for the sensor nodes, the application
 developers are presented with lesser challenge, because existing algorithms,
 packages and libraries can be used. Care still needs to be taken due
 to limitations such as bandwidth and data structure complexity.

 For example, a Representational State Transfer Application Programming
 Interface (\emph{RESTful API}) \cite{links:wiki:rest} would make a
 system much simpler to integrate into existing environments,
 there won't be a need to develop tools for any particular computer platform.
 \emph{RESTful} approach brings a cross-platform compatibility out-of-the-box
 by utilising web-client software with very little of extra work to be done.
 \emph{RESTful API} is largely used in modern web-application services
 (also known as "the cloud"), it uses human readable URL strings for
 addressing functions on remote servers via HTTP protocol (the handling
 of HTML or XML response is optional and not applicable here, due to above
 mentioned data structure complexity issue).

 Another feature of \emph{IPv6} that is very important for complex wireless
 networks, is \emph{neighborhood discovery}, which eliminates any need
 for what DHCP served in \emph{IPv4}. Regarding the scalability,
 it is true that there is no particular use in this project for 128-bit
 address space that \emph{IPv6} provides. Nevertheless, this network
 may need to extend to a larger number of nodes if a orchestra of these
 is going to be built, also including various stage automation and
 safety sensors all in one area.


\subsection{Alternatives and Concerns}

  It is very appropriate to discuss a few alternative approaches which
 could be taken to implement a radio network specific to the application
 of interactive objects for live performance.

  An alternative to using \WPAN\ radio would be a different, rather
 simpler protocol and, perhaps, in a different frequency band.
  The reason for this is obviously because of application-specific
 concerns. The data transmission for musical performance is constrained
 to very strict real-time latency measures. In other words, when a
 musician is playing on stage the audience will perceive the music as
 very odd if the latency is too high. It would be even worse if several
 participants exhibit random latency, while not playing an improvised
 piece of music.

  Using a radio protocol which is in its own \emph{clear} frequency
 band and uses a minimal communication abstraction stack, should be
 more appropriate for latency optimisation. However, the cross-platform
 networking capabilities would be lost. That might lead to incompatibilities
 with market trends as well as radio band regulations.
  Since \emph{IEEE 802.15.4} is the current global technology trend and
 the 2.4GHz ISM band is accepted world-wide, the alternatives are not
 quite appropriate.

  There are a few solutions for \emph{non-P802.15.4} 2.4GHz ISM radio,
 but these would impose a vendor locking for the transceiver interfaces
 to such manufacturers as \emph{Nordic Semiconductors} \cite{links:nordic:rf2400}
 or \emph{Hope Microelectronics} \cite{links:hoperf:rf2400}.
 There might be even model-specific incompatibilities, while 
 it is quite certain that all \emph{P802.15.4} transceiver
 chips will be compatible in the future.

\break

\section{Study of Earlier Work \\and Commercial Solutions}

  Earlier works in the field of wireless performance devices (i.e. music instruments and
  devices to assist dance and theater performance) were studied. A number of links to
  other sites and publication are listed in the WMI website sections "Links" \cite{wmi:wiki:links}
  and "Bibliography" \cite{wmi:wiki:refs}. Below only the most relevant information is discussed.

\subsection{Publications}

  One publication by Paulo Jorge Bartolomeu (Univ. of Aveiro, 2005)
 \cite{pub:bartolomeu2005} which has also been submitted to the IEEE,
 evaluates the use of \emph{BlueTooth} for a wireless MIDI network.
 Paulo Bartolomeu in his dissertation considered either \emph{ZigBee} or
 \emph{UWB} as a possible alternative, however in 2005 these technologies
 still were immature and expensive. A number of interesting results and
 figures are provided, proving that \emph{BlueTooth} exhibits a definite
 byte latency of, on average, 20ms and a maximum of 100ms, these figures
 were reduced with improved network configuration to an average byte latency
 below 10ms. A quite interesting "command aggregation algorithm" had also
 been proposed in this paper.

 A surprisingly small amount of white papers have been found on this subject, 
 which indeed makes this study more interesting. Though, there are various
 articles available on-line, these are listed on the WMI website. There is
 no significant information to discuss in regards to those articles
 \cite{links:misc:xbeemidi}.
 
 A few short reports demonstrating \emph{BlueTooth} implementations were
 found on Stanford University CCRMA (Centre for Research in Music and
 Acoustics) website, these appear to be homework reports by CCRMA
 students \cite{pub:ccrma:muggling, pub:ccrma:brbi}. A few more
 interesting articles are listed in the bibliography \cite{pub:ccrma:sm,
 pub:ccrma:bb}. One more recent publication describing a \emph{BlueTooth}
 controller for \emph{PureData} \cite{links:wiki:pd} is by a researcher
 at Plymouth University \cite{pub:plym:go}.


\subsection{Commercial Wireless MIDI Products}

  Currently there are four commercial solution available from different
 manufacturers. It is most likely that the radio communication protocol
 used by these devices is unique to each particular manufacturer, since
 no compatibility information is provided.

\emph{
\begin{itemize}
 	\item Kenton MidiStream \cite{links:kenton:midistream}
	\item M-Audio MidAir \cite{links:maudio:midair}
	\item CME WIDI \cite{links:cme:widi}
	\item PatchMan MidiJet \cite{links:pmm:midijet}
\end{itemize}}

 The \emph{Kenton} devices operate in the two sub-gigahertz ranges, 869MHz
 and 433Mhz, while the other three manufactures use the 2.4GHz ISM band.
 It may be desired to evaluate these devices before proposing any solution
 of a commercial nature; there is no particular interest to obtain any
 of these product otherwise.


\subsubsection{Various Available Hardware}

 \emph{MiniBee} project by Canadian group of artists \emph{Sense/Stage}
 \cite{links:sensestage:homepage} is of particular interest, it utilizes
 high-level frameworks such as \emph{Pure Data}, \emph{Max MSP} and
 \emph{Super Collider} \cite{links:sensestage:host}. More details about
 these software frameworks will be provided later in the report, as one
 of them is also used in this project. The \emph{MiniBee} hardware
 \cite{links:sensestage:node} is using \emph{Atmel AVR} 8-bit MCU and
 \emph{Digi Xbee} transceiver module, which is not the most optimal
 design solution. Again, \emph{ZigBee} is a patented technology and
 is not appropriate for an \emph{open-source} project. Also, the
 \emph{Xbee} module itself is already using an MCU to run its software stack.
 
 In the area of sensor products, there is one interesting company called
 \emph{Phidgets, Inc.} \cite{links:phidgets:homepage}, it manufactures rather
 wide variety of sensor devices, ranging from base single-board computer (SBC)
 to sensor and actuators sharing same type of 3-pin connector. The SBC board
 runs custom-built \emph{Linux OS} on \emph{ARM9} 266MHz CPU with 64MB of
 SDRAM and 64MB of flash memory. \emph{Phidgets SBC}
 \cite{links:phidgets:products:sbc} has USB and 100MB Ethernet peripherals as
 well as a number of screw terminals and 3-pin connectors for sensors.
 \emph{Phidgets, Inc.} supplies numerous easy-to-use sensor and actuator
 devices. The cost of their products is quite high, but the reason of
 this is the target market. Buyers of these easy-to-use devices are not
 professional electronics engineers and are ready to pay these prices.
 Their website also has source code for reference in a wide selection of
 programming languages and frameworks.
 
 However the top product in the hobbyist marker remains to be \emph{Arduino}
 \cite{links:arduino:homepage}, of course the low price factor is key of its
 popularity. A huge number of semi-clone products emerged, all featuring
 various additions that some users may find more useful. The most recent
 interesting semi-clone of \emph{the Arduino} is \emph{Seeeduino Film}
 \cite{links:seeed:products:film}, made by \emph{Seeed Studio} in China.
 It appears to be the only product of its class that uses flexible film
 instead of regular circuit board. For example, this could be quite
 useful for integration into a music instrument or attached on performer's
 body in combination with capacitive flex sensors. Even the above mentioned
 \emph{MiniBee} in its design is also a semi-clone of \emph{the Arduino},
 in order to support convenient \emph{Arduino C++ Library}
 \cite{links:arduino:library}, which is in no doubt appreciated by the
 target audience.


\part{Design \& Development} 
\chapter{Operating Systems for Wireless Sensor Networks}
\section{Overview of Embedded Operating Systems} \label{sec:RTOS}

		%% REWRITE !!

   A subject of extensive research for this project has been into the
  area of Real-Time Operating Systems for embedded devices, RTOS for short.
  This family of operating systems is not like any other general
  purpose OS family. The distinct feature that any RTOS has is the task
  management in constrained time-scale, the term \emph{real-time} is relatively
  abstract, it can be classified either as \emph{soft-} or \emph{hard-real-time}.
  The classification is also an abstract matter, some operating systems provide
  various approaches and it is up to the engineer to utilise it correctly for
  any given application. A modern RTOS should also provide a communication
  protocol stack and file storage drivers, support a set of platforms and
  a variety of peripheral devices.
 
 It is commonly known that RTOS is a rather expensive element in embedded design.
 This is because until very recently most commonly used RTOS implementations were
 provided only by specialised commercial vendors, brand names like \emph{Micrium},
 \emph{VxWorks} and perhaps \emph{QNX} are popular in specific application fields
 and there are also more obscure RTOS products offered at premium prices. Another
 more recent commercial RTOS is \emph{Nucleus}, currently owned by \emph{Mentor
 Graphics}. This repor does not focus on commercial products, hence these will
 not be refer to, more information can be found in \cite{wiki:rtos,wiki:rtos:list}.

 %http://en.wikipedia.org/wiki/Real-time_operating_system
 %http://en.wikipedia.org/wiki/List_of_real-time_operating_systems

 Currently the open sources community does present a number of very interesting
 RTOS projects, some of which are briefly introduced below. More details can be
 found on the WMI website \cite{wmi:wiki:rtos} (all the general project URLs are
 listed in that section of the website, and therefore omitted from the text below).
 It may be important to note that many of the project which are listed here are
 originating from major universities, some of which previously gave birth to very
 well know UNIX software and general innovations in the computer science.


 \subsubsection{RTEMS} \label{sec:rtos:rtems}

 The \emph{Real-Time Executive for Multiprocessor Systems} or \emph{RTEMS} is a
 full featured RTOS that supports a variety of open API and interface standards.
 It originates from US military and space exploration organisations. By definition
 \emph{RTEMS} is a highly scalable system, i.e. it is designed for multi-processor 
 deployment and is an interesting system to use for various possible applications,
 \emph{RTEMS} appears to be the oldest open-source RTOS. Unlike other systems below,
 the \emph{RTEMS} doesn't belong to sensor network OS sub-family on which this
 project had been focused. 


\subsubsection{TinyOS} \label{sec:rtos:tinyos}

 \emph{TinyOS} originates from the University of California, Berkeley and
 Stanford and the Intel Research Laboratory at Berkeley \cite{wiki:tinyos,
 links:tinyos:webs, links:tinyos:homepage}. The approach of \emph{TinyOS}
 is very different from all other systems, it uses a higher level abstraction
 language based on \emph{C}, which is called \emph{nesC}. It has a very
 modular semantics with degredd some similarity to hardware description
 languages (HDL), such as \emph{VHDL} or \emph{Verilog}. The name \emph{nesC}
 stands for "Network Embedded Systems C", in analogy with HDL it describes
 application level connectivity for modules, protocols and drivers. Despite
 this high level of abstraction, compiled source code is targeted for devices
 with limited compute resources \cite{links:tinyos:nesc}. The development
 ecosystem of \TinyOS\ also includes a network simulation package \emph{TOSSIM}
 \cite{links:tinyos:tossim}, which is a rather trivial element to design when
 such level of abstraction is in place. At the start of the project \TinyOS\
 was of some interest, however it had been opted out in preference to \Contiki\
 for it's more tradition \emph{"plain C"} language abstractions. However,
 \TinyOS\ had been evaluated during the development phase and more details
 can be found under section \ref{sec:TINYOS}.

 %http://en.wikipedia.org/wiki/TinyOS


\subsubsection{Nano-RK} \label{sec:rtos:nanork}

 \emph{Nano-RK} is a fully preemptive reservation-based RTOS, unlike
 any others in this section it already contained source code for \RFA\
 chip. The testing of \emph{Nano-RK} had been added to the project agenda.
 This OS is developed by Carnegie Mellon University and there are various
 interesting research papers available from the website
 \cite{links:nanork:pubs, pubs:nrk07, pubs:nrk06b, pubs:nrk06a, pubs:nrk06c}.
 Some particular features to mention are an implementation of real-time
 wireless communication protocol for voice signal transmission \cite{pubs:nrk06a}
 and a presence of an audio device driver within \emph{Nano-RK} kernel.

\TrackerList
	 \IssueX{21} Test Nano-RK on \RFA
\TrackerEnd



\subsubsection{Contiki - "The OS of Things"} \label{sec:rtos:contiki}

 {\emph{"Contiki is an open source, highly portable, multi-tasking operating
 system for memory-efficient networked embedded systems and wireless
 sensor networks. Contiki is designed for microcontrollers with small
 amounts of memory. A typical Contiki configuration is 2kB of RAM
 and 40kB of ROM.}}
 
 {\emph{"Contiki provides IP communication, both for IPv4 and IPv6. (\dots)}}
 
 {\emph{"Many key mechanisms and ideas from Contiki have been widely adopted in
 the industry. The uIP embedded IP stack, originally released in 2001, is today
 used by hundreds of companies in systems such as freighter ships, satellites
 and oil drilling equipment. (\dots) Contiki's protothreads, first released
 in 2005, have been used in many different embedded systems, ranging from
 digital TV decoders to wireless vibration sensors.}}
 
 {\emph{"Contiki introduced the idea of using IP communication in low-power
 sensor networks. This subsequently lead to an IETF standard and
 the IPSO Aliance (\dots)}}
 
 {\emph{"Contiki is developed by a group of developers from industry and
 academia lead by Adam Dunkels from the Swedish Institute of Computer
 Science."} \cite{contiki:home}

 Many publications by Adam Dunkels can be found on the SICS website
 \cite{dunkels04contiki, tsiftes09enabling}. He is also an author of
 two recent textbooks on IP WSN subject \cite{dunkels09operating,
 vasseur10interconnecting}. In addition, there is a rich set of simple
 code examples illustrating basic applications. The source code has
 well-defined directory hierarchy, build system and programming style
 \cite{contiki:code:style}.

 \emph{Contiki} provides a very simple API for advanced functions,
 such as real-time task scheduling, \emph{protothread} multitasking,
 filesystem access, event timers, TCP \emph{protosockets} and other
 networking features.

 More detailed description \emph{Contiki} is provided the following
 section (\ref{sec:CONTIKI}).
  

\section{Introducing the Contiki Operating System}\label{sec:CONTIKI}

  \Contiki\ operating system can be seen as collection of processes.
 In the same way as the UNIX operating system is seen as collection of
 files and processes. However, files are not a necessary component, in
 fact \emph{Coffee File System (CFS)} is only needed for some specific
 applications which require a non-volatile storage abstraction. 
  The inter-process communication (IPC) in UNIX has not been implement
 in the very early releases, although Contiki processes wouldn't work
 without its event-driven IPC mechanism\footnote{\emph{Though it is very
 primitive comparing to the usual meaning of IPC as an acronym.}}, which also
 drives the core scheduler. There is no notion of users in \Contiki\,
 however, and at this point comparison with UNIX becomes apparently
 impossible. Nevertheless, such a high-contrast comparison will hopefully
 deliver a good picture of what \ContikiOS\ is and what it is not.

% check the date in the slides from WSN course

  The lead developer of \Contiki\, Adam Dunkels of Swedish Institute of
 Computer Science (SICS), has revolutionised embedded system world by
 implementing \emph{lwIP} in 2000 and later, in 2002, even further optimised
 \emph{uIP} stack. It is known that until \emph{lwIP}, no complete IP stack
 existed which could run on a microcontroller device \cite{dunkels03full}.

  The key technique utilised in implementing \Contiki\ and the \emph{uIP}
 stack is based on a very obscure behaviour which the C language's
 \texttt{switch/case} statement exhibits in certain context. Dunkels
 wrote a few papers \cite{dunkels05protothreads,dunkels06protothreads}
 describing how it had been achieved, and summarises it in his doctoral
 thesis \cite{dunkels07programming}. The threading technique has been
 named \emph{protothreads (PT)}. Processes in \Contiki\ are designed as
 an extension to the \emph{protothreads}. Another extension is called
 the \emph{protosockets}, as the name suggest it provides the abstraction
 of network sockets.

  Most of functionality of all three key elements is delivered by the use
 of C pre-processor macros. The \emph{protothreads} are platform
 independent\footnote{\emph{It is not completely true for Contiki, there
 are few additions which utilise facilities that some hardware platforms provide.}}
 and can be integrated into any C program by including only one header file.
 \emph{Protothreads} can be used with virtually any C compiler and,
 as described in \cite{dunkels06protothreads}, the overhead is rather
 insignificant, considering the powerful functionality; the only limitation
 is due to the use of \emph{\texttt{switch/case}} statements by the \emph{PT}
 macros, these control structures will lead to undefined behaviour of the
 code in the \emph{PT} context.

  \emph{Protosockets} are designed specifically for \emph{uIP} and therefore 
 cannot be included with just one header. The only limitation of the
 \emph{protosockets} is due to design decision, there is no UDP communication
 facility, only TCP is provided. The \emph{protosockets} greatly simplify the
 application code, since all necessary functions are provided, including
 handling of strings. Thought no packet flushing is currently possible, which
 may be desired in some application.

\subsection{Initial Development Phase}

  \Contiki\ source code includes a large set of stable drivers for various
 RCBs which were mentioned previously, such as \emph{Zolertia Z1}, \emph{Tmote
 Sky}, \emph{Atmel Raven} and \Chip{RF230}\emph{-based} boards, \emph{TI}
 \Chip{CC2430} and \emph{MSP430} as well as \emph{ARM Conrtex-MX} and many
 older devices, including \emph{Comondore 64} and \emph{Apple II} computers.

 Details regarding the \Chip{MC1322x} port \cite{links:contiki:port:mc1322x}
 are to follow in \ref{sec:APPDEV}, since \Chip{MC1322x} was chosen as the
 target platfrom for this project. However, the \RFA\ platform had been the
 first preference. Major work during the inital phase had been done to
 port the existing driver code for \Chip{RF230} transceiver to take the
 advantage of the single chip device. This work had been described in the
 interim report \cite{wmi:irep}.


  A considerably long period of work during the first semester had been
 towards porting of \ContikiOS\ to run on \RFA\ chip. The details were
 included in the interim report \cite{docs:irep}. However, if the porting
 was to be completed, the time it would have taken may have exceeded the
 allocated period for the project (one academic year). Therefore it was
 uderstood that some reconsideration needed to be taken and two major
 solutions were proposed: either changing the development hardware or
 trying a different OS with the same hardware.

\subsubsection{Problems Ecountered while Proting the Driver Code}

  It had been not a trivial task to compare a set of implementations
 of the driver code, nevertheless some common fragments of code had
 been identified. The interinm report has included the details of what
 was acheived at the time, though most of work in this area \dots

\section{Outlook Trial for the TinyOS} \label{sec:TINYOS}

  It has been discovered that a few members of \TinyOS\ community
 have recently worked on porting \TinyOS\ to run on \Chip{ATmega128RFA1}
 \cite{tinyos:arch:rfa1-p1,tinyos:arch:rfa1-p2}. Comparing to \Contiki,
 \TinyOS\ has rather superior abstraction layer that provides a seamless
 cross-platform integration \cite{tinyos:tepXXX,tinyos:tepYYY,tinyos:tepZZZ}.
 
  Various internals of \TinyOS\ had been studied, however it is 
 certainly a very broad area to be described here. It is best described
 in the \emph{"TinyOS Programming"} book by David Gay and Philip Levis
 (it is available in print as well as a web-edition \cite{tinyos:book}).
 Two authors of this book are lead developer of the \TinyOS\ project
 and are currently working at the University of California, Berkley
 and Standford.


  \TinyOS\ currently had been ported to a variety of 8-, 16- and
 32-bit microcontrollers, most of the peripherals (such as I2C and SPI)
 have unified high-level access mechanisms, unlike in \Contiki\footnote{
 \emph{Most of peripheral drivers in Contiki OS are rather specific to
 a certain platform and only some particular drives share a similar
 interface.}}. Most of development documentation is provided in the
 the format of \emph{"TinyOS Enhancement Proposal"} (known as
 \emph{TEP}\footnote{\emph{These documents are commonly referred to as
 TEP\texttt{n}, where \texttt{n} is a number. \TEP{1} defines the format
 of the TEP documents.}}), there is also documentation generated from
 the source code comments (\emph{nescdoc}) as well as various on-line
 resources \cite{tos:wiki:docs} and the textbook mentioned above.
 It is noticeable that \TinyOS\ documentation is more extensive the
 what is currently presented for the \ContikiOS.


\subsection{Programming TinyOS: Compilers and Abstractions}

  The greatest achievement of \emph{TinyOS} is its specialised language,
 benefits of which had been overlooked at the earlier stage of research
 for this project. \emph{"Network Embedded Systems C" (nesC)} is a
 C-based language, it provides abstractions for components, interfaces
 and configurations. It is not an object-oriented language, though
 it is rather component and interface oriented. As it was defined in
 the previous section that \Contiki\ can be seen as a collection of
 processes and events are communicated between those processes, then
 \TinyOS\ can be defined as a collection of components and interfaces
 which are wired together in configuration abstraction layer, tasks,
 events and the data are communicated between the components via the
 interfaces.

  The control structures and all of C constructs are still valid in
 \emph{nesC}. This can be seen as if \emph{nesC}-specific statements
 are replaced by appropriate C source code that defines required behaviour.
 Nevertheless, this is only a brief description of the relation between
 \emph{nesC} and its ancestor. Just as if C could be defined by stating
 a ratio of lines of code in C versus lines of assembly code.

 It is important to note that current implementation of the compiler
 translates higher-level nesC code into C language. The resulting
 programs are specifically suited for embedded devices. Direct
 compilation would be possible and, if desired, there is an interesting
 platform to look into. \emph{LLVM} ("Low-level Virtual Machine") is a
 new generation compiler technology. It is know that using \emph{LLVM}
 and its family member \emph{clang} implementation of a new compiler
 for a C-like language would be simplified (comparing to more traditional
 techniques). One good example of industrial grade compiler based on
 \emph{LLVM} is the \emph{XC} toolchain for \emph{XS-1} devices from
 a UK semiconductor company \emph{XMOS} \cite{paper:xmos:docs:xcc}.

 Nevertheless, currently \emph{nesC} compiler is known to be fully
 functional and it's task is not as complex as it may seem. Programming
 in any language is always done by applying code patterns, general
 to some degree, to accomplish desired behaviour of a program.
 The purpose of \emph{nesC} abstraction layer can be seen as making
 the details of complex coding - with which high degree of modularity
 can be achieved - rather hidden away from the programmer.


\subsection{Network Protocols in TinyOS}
 
  \TinyOS\ has become widely adopted and there are many researchers
 who contributed significant work in different application areas, one
 such area of great interest to WMI project is sensor node timer
 synchronisation protocols. A number of papers are available and the
 mainstream repository of \emph{TinyOS} source code already contains
 implementations for some of those protocols.
 \emph{( A few details with references needed )}
  Further study and experimentation in this area are required to design
 a system which could cope with real-time constraints of stage control
 applications. It is important to note that \emph{P802.15.4} has addressed
 some real-time application requirements, however the status of software
 support for these features of \emph{P802.15.4} is currently uncertain.
 \emph{( Perhaps mention NanoRX ? )}

  After several aspects of \emph{TinyOS} were studied, it was clear
 that its current implementation of \emph{6loWPAN} is fully compatible
 with \emph{Contiki} and it had been desired to prove this in practice,
 but this has not been achieved at the time of writing of this paper.
  
  A problem exists however, there was no fully working IP-enabled driver
 for \emph{ATmega128RFA1} transceiver. The IP layers are provided by
 \emph{BLIP} stack (this stands for Berkley Lightweight IP), the stack
 is currently undergoing major development. Details on what is required
 are available on \emph{TinyOS} wiki page \cite{tinyos:wiki:blip-2-0},
 implementing it in \emph{nesC} was not as trivial due to the learning
 curve. Therefore this task had be postponed.

  This problem requires further explanation. \emph{TinyOS} has been
 developed since 2001, while \emph{Contiki} first dates back to two
 years later - 2003\footnote{\emph{No exact information has been
 found, these are the earliest dates which appear in the source code.}}. 
 In the early days of WSN research \emph{P802.15.4} and \emph{ZigBee}
 where still emerging, the \emph{6loWPAN} RFCs appeared at IETF more
 recently\footnote{\emph{The first revision of P802.15.4 was released
 in 2003 (current revision is from 2006) and first draft of 6loWPAN
 is dated April 2007.}}. \emph{TinyOS} has originally used its own
 protocol called \emph{ActiveMessage}.

  Currently \TinyOS\ incorporates \emph{BLIP} stack, which initially
 was released by UC Berkley Wireless Embedded Systems research group
 in the summer 2008, know as \emph{bl6lowpan} at that time \cite{ucb:webs:blip}.
 As mentioned below, major code changes are currently still in progress.
 The driver code which fully implements all new features of the \emph{BLIP}
 stack (including point-to-point tunnelling for simple UART wired
 connectivity) is only for the most popular \Chip{CC2420} radio chips.

\subsection{Hardware Resources}

  It is difficult to compare the two sets of what hardware platforms
 \TinyOS\ and \Contiki\ had been ported to, since the status of support
 for some platforms is uncertain. It is probably most appropriate to
 say that these two sets are almost equal, however, some platforms
 which appear in \Contiki\ source code, had not been ported to \TinyOS\
 and vise versa. Two devices of interest to WMI project are \RFA\ and
 \MCX. The issue with the first chip is already described above.
 The second chip has not been ported to \TinyOS\ yet. This should not
 be too difficult task to achieve considering that driver implementation
 could be derived from \Contiki\ code. There also very noticeable
 progress towards \emph{ARM Cortex} support\footnote{\emph{The homepage
 of TinyOS Cortex project can be found at:
 \URL{http://code.google.com/p/tinyos-cortex/}
 The code structure was found to be well organised and most probably
 can be relied upon. However it has not been desired to take this
 challenge in the near future.}}, hence there is base for \emph{ARM}
 devices (i.e. toolchain support and core architecture code).
 

  There is a large repository of board design files featuring so called
 \emph{Epic Mote} platform, that is base around \Chip{CC2420} and the
 \Chip{MSP430} \cite{epic:homepage}. The most interesting designed by
 Prabal Duta of UC Berkley \cite{duta:homepage} is shown in Figure
 \ref{fig:mote:quanto}. The reason why it is so interesting is because
 it features \emph{Digi Connect-ME} microprocessor system block that is
 of standard RJ45 form-factor \cite{digi:connect-me}. This tiny device
 has 55MHz \Chip{ARM7TDMI} CPU, 4MB of flash and 8MB of SDRAM as well as
 hardware cryptographic unit and is capable to run Linux kernel and a
 small subset of standard software in userspace, hence the advantage of
 scripting languages can be utilised. Currently it is the smallest
 device available on the market featuring such capabilities and
 the unique form-factor. It is a very appropriate device to use for
 wireless network edge gateway system, for example a hardware crypto
 unit could accelerate secure tunneling of the network traffic. It
 could be considered as a better solution instead of using a large
 Ethernet chip, unless there would be a chip with a combination of
 Ethernet and RF hardware in a single package.

% INCLUDE THIS IMAGE:
% http://www.cs.berkeley.edu/~prabal/projects/epic/epic-quanto.jpg


\subsection{Conclusion}

  Several steps were made in attempt to bring up \emph{TinyOS} on the
 hardware chosen for this project earlier, though it appeared rather
 unmanageable withing the give time frame. This is still a subject
 of interest and shall be looked into at a later time. An evidence of
 work carried out with \TinyOS\ code can be viewed on WMI website
 issue tracker.

\Tracker
 \IssueX{30} Porting Radio Interface to BLIP 2.0
 \end{description}

%\URL{https://github.com/errordeveloper/tinyos-wmi/commits/wmi-work?author=errordeveloper}

% number of bug tracker issues were filed on the WMI website



%% to be added to the bibtex file:

 %http://www.cs.berkeley.edu/~prabal/projects/epic/
 %http://www.cs.berkeley.edu/~prabal/
 %http://www.digi.com/products/wireless-wired-embedded-solutions/solutions-on-module/digi-connect/digiconnectme.jsp
 %http://smote.cs.berkeley.edu:8000/tracenv/wiki/blip
 %http://docs.tinyos.net/index.php/BLIP_2.0_Platform_Support_Guide
 %http://standards.ieee.org/getieee802/download/802.15.4-2006.pdf
 %http://standards.ieee.org/getieee802/download/802.15.4-2003.pdf
 %http://tools.ietf.org/pdf/rfc4944.pdf

\chapter{Wireless Node Hardware}
\section{Overview of Hardware Options: \\Platforms and Cores} \label{sec:hwintro}

%There had been an extended research and early decision was
%to use \emph{Atmel} \RFA\, however this has changed for the
%reason described in the previous chapter.

 This chapter discusses various alternatives, which had been
 considered for the target hardware platform. To narrow this
 selection, an MCU with an integrated radio was of primary
 interest. Currently, the semiconductor market presents a
 wide range of single-chip MCUs with \WPAN\ interfaces. There
 is a dedicated section on the WMI website for more information
 and references to product pages \cite{wmi:wiki:chips1,
 wmi:wiki:chips2}.

\subsubsection{32-bit ARM}

 Two different integrated \WPAN\ microcontrollers based on 
 \emph{ARM} core architecture were found. Most notable
 chips are \emph{Freescale \texttt{MC1322x}}, which are
 classified as a \emph{platform-in-package}. \emph{Freescale}
 combined all transceiver components in one chip, providing
 a very simple design solution. Only one external RF component
 that \MCX\ requires is the antenna. Alternative \emph{Cortex-M3}
 product is \emph{Ember \texttt{EM300}-series}, these don't
 achieve the same scale of integration on the transceiver side
 and still requires an external RF circuit.

 The \Chip{MC1322x} gives a great benefit to a designer, as it
 is known that layout of a radio circuit for frequencies in
 gigahertz range is particularly challenging task and, indeed,
 this is a specialist area of board design.

 More details on \MCX\ will appear in section \ref{sec:MCX},
 including a block diagram of the architecture (Figure \ref{fig:mc1322x}).

\subsubsection{Other Microcontrollers}

 \emph{ST Microelectronics} and \emph{Ember} offer 16-bit single-chip
 \WPAN\ microcontrollers based on \emph{XAP2} processor architecture.

 \emph{NEC} (now \emph{Renesas}) has a few 16-bit RF chips available,
 but the selection of development tools and libraries for that platform
 is rather poor.

 \emph{Freescale} has a another set of 8-bit \WPAN\ microcontrollers.
 \Chip{MC13213} family currently consists of just three devices with
 16k/32k/60 kB of flash memory and \emph{Motorola 68HC08} core.

 A UK company \emph{Jennic} (now acquired by \emph{NXP}) produces a
 range of wireless micorocontrollers and sub-assembly modules. These
 are utilising \emph{OpenRISC} 32-bit processor core combined with a
 proprietary transceiver. The \emph{OpenRISC} core had been developed
 as a community effort of \emph{OpenCores.org}, however \emph{Jennic}
 software source code is not available and a rather low quality binary
 library distribution is offered. This software had been analysed, but
 the results are irrelevant to this report.

 \emph{Texas Instruments} currently offer a \Chip{CC2530} based on
 \emph{8051} MCU with 2.4 GHz radio, which is most like a product for
 designers who prefer to use \emph{8051}. \emph{Telit Communications}
 sells sub-assembly RF and USB dongles featuring the \Chip{CC2530} chip.
 Another interesting product from \emph{Texas} is \Chip{CC8520} which
 allows low-power compressed audio transmission in the ISM band, however
 as it was mention earlier, transmission of audio is not of interest
 to this project.

 At the time of writing of this report the product range of \emph{TI}
 wireless ICs has expanded. It now has \Chip{CC2511} \emph{SoC}
 that is based on \Chip{CC2530} with a USB interface added and
 yet another integrated family - \Chip{CC430}. These chips are using
 \emph{TI's} \Chip{MSP430} 16-bit core and a sub-gigahertz \Chip{CC1101}
 transceivers.

\subsubsection{8-bit Atmel AVR ATmega128RFA1}

 The \Chip{ATmega128RFA1} \cite{atmel:atmega128rfa1:datasheet} chip
 has 16MHz clock and contains 128kB of flash, 4kB of EEPROM and 16kB
 of SRAM memories, 32 general purpose registers, 35 GPIO lines, 8
 channels of 10-bit ADC and 6 configurable timers, counters and PWM
 as well as USART, SPI and JTAG interfaces. \Chip{ATmega1281} core
 of the \Chip{ATmega128RFA1} is a generic 8-bit microcontroller.
 Apart from integrated \Chip{AT86RFA231} \cite{atmel:at86rf231:datasheet}
 low-power 2.4GHz \emph{P802.15.4} transceiver, there are no other
 distinct features, it is a stock-standard AVR chip, though the first
 Wireless MCU from \emph{Atmel}.

 A choice of software stacks is provided \cite{atmel:avr2070,
 atmel:avr2025, atmel:avr2102, atmel:zbpro}, although the use
 of these packages is a subject to licensing and architectural
 issues \cite{wmi:wiki:atmelsw}.

\subsection{Boards and Peripherals}

 \emph{Note: Product URLs are provided on dedicated page of the WMI
 website \cite{wmi:wiki:devhw}.}
 \newline

 For a radio controller board (RCB) one of the major factors is the
 type of antenna. Type of sensor connectors and serial interface,
 as well as presence of JTAG header are also taken into consideration.

\subsubsection{\texttt{"32-bit"}}

 \emph{Freescale} offers a variety of evaluation boards featuring
 \Chip{MC1322x} chips and some interesting sensor ICs. There is
 a number of \Chip{MC1322x} RCBs from 3rd-party suppliers, an
 up-to-date list is available from \emph{Contiki} \Chip{MC1322x}
 website \cite{links:contiki:rcb:mc1322x}. More information on
 this platform will appear below \ref{sec:MCX}.

\subsubsection{\texttt{"16-bit"}}

 Two interesting devices which are using TI MSP430 16-bit MCU with
 a discrete transceiver chip is \emph{Zolertia Z1} board and the
 \emph{Epic Mote}. The \emph{Zolertia} PCB also features a 3-axis
 accellerometer and two 3-pin sensor connectors (compatible with
 \emph{Phidgits} sensors). USB serial interface chip and two anennas
 are other distinct features of this device. A chip antenna is
 soldered on to the \emph{Z1} PCB and second micro-FL socket is
 provied for alternative antennae options. A version of \emph{Z1}
 with extra peripherals (JTAG, battery holder and external antenna)
 as well as compact plastic enclosure will be soon available
 from \emph{Zolertia} sore \cite{links:zolertia:store}.
 
\subsubsection{\texttt{"8-bit"}}

 \emph{Atmel} offers a range of RCB evaluation kits
 \cite{links:atmel:rcb} and there a few \emph{ZigBit}
 modules which feature an external RF power amplifier,
 although those could be great boards to use, the
 3rd-party solutions are more suitable for the budget
 of this project. Two development board options are
 briefly described below.

 First \Chip{ATmega128RFA1} board that had been looked at 
 was \emph{Dresden Elektronik Radio Controller Board RCB128RFA1}
 \cite{links:de:rcb}. In combination with \emph{Sensor Terminal
 Board} \cite{links:de:stb}, it is a very suitable development 
 and prototyping solution. The \emph{RCB} has external antenna
 connector and the MCU chip is enclosed in a metal shielding
 and a battery holder is also included. The \emph{STB} boards
 has screw terminals for sensors and JTAG, ISP, USB and DC
 power connectors, as well as an external 32kB SRAM chip.

 The \emph{SparFun} board pictured in Figure
 \ref{fig:sparkfun:atmega128rfa1:image}
 has a much more compact design, therefore it could fit
 into a slim enclosure with \emph{I/O} breakout connections
 or, otherwise, just a few sensors placed inside of the
 enclosure for a simpler application. The circuit diagram
 for this device is shown in Figure
 \ref{fig:sparkfun:atmega128rfa1:circuit}.
 A pair of these boards has been purchased and used during
 the first phase of development (\ref{sec:RFA}).
 This preference was due to the buget concerns.

\subsection{Popular Solutions}

 The \emph{Xbee} \cite{links:digi:xbee} modules from \emph{Digi}
 are very popular among hobbyists, however it is rather a
 drop-in solution, and is not appropriate for a new design.
 \emph{Xbee} modules can also be used without external MCU
 \cite{links:misc:xbeemidi}, but the functionality is fixed
 to what is implemented in the firmware. Programable modules
 are also available \cite{links:xbee:wiki:prog}, however very
 limited information has been found. One recently published
 book \cite{noble2009programming} does suggest to use the
 \emph{Xbee} modules, althogh author mentions no alternatives.
 \emph{IcludeTech WiMi} \cite{links:includetech:wimi}
 is a development board which uses an \emph{Xbee} module with a whip
 antenna and a \emph{Microchip PIC} MCU. It would be appropriate as a
 drop-in module for a MIDI device which is being produced already and
 wireless connectivity needs to be offered as an expansion option,
 engineers who are familiar with \emph{PIC} microcontrollers may find
 these devices suitable for their designs.

\begin{figure}
\centering
\includegraphics[scale=2.8]{../figures/images/sparkfun_atmega128rfa1_img00c.png}
\caption{\emph{SparkFun} \Chip{ATmega128RFA1} \emph{Development Board}} \label{fig:sparkfun:atmega128rfa1:image}
\end{figure}
\pagebreak


\section{Initial Hardware Consideration} \label{sec:RFA}

  At the initial stage of this project the hardware platform was chosen
 based on previous experience of using development tools for \emph{Atmel
 AVR} microcontrollers. The original orientation was towards using a chip
 with integrated radio transceiver, \emph{Atmel} advertises the \RFA\ chip
 as having the lowest power consumption and most appropriate link budget.

\subsubsection{Development Boards for ATmega128RFA1}

  As it was already mentioned, two choises of development boards were considered:

	\begin{itemize} \em
		\item Dresden Elektronik \cite{links:de:rcb,links:de:stb}
		\item SparkFun (Figure \ref{fig:sparkfun:atmega128rfa1:image})
	\end{itemize}

  \emph{DE} hardware features a useful set of components, including a spear 32 kB
 of memory and robust screew terminals for I/O connections. However, the board
 from \emph{SparkFun} has been chosen for it's low price.
  It has been found later that this board was inconvenient to use in various
 ways, for example the layout of serial port pins on the side of the board
 could be design to fit standard serial USB cables\footnote{\emph{SparkFun
 sells few different serial USB adaptors all of which have the same layout.
 These are know to be very popular among most of hobbists and professional
 engineers and therefore are considered to be a de-facto standard.}}.
 Also the layout of transciever side of the PCB was found to be quite primitive
 and, most importantly, has not included a suitable ground plane nor it has a
 shield. Nevertheless, this board is of a rather small footprint and could be
 used in a prototype product.

%% inlude: CIRCUIT & BOARD LAYOUTS
%% copy the section of interim report
%% where some stuff is said regarding
%% the circuit ... or may be not ???

\section{Selecting the Development \\Platform for Contiki OS} \label{sec:MCX}

  Examining the code for various hardware platforms in the source
 tree of \Contiki, it was understood that source code for
 \Chip{MC1322x} devices from \emph{Freescale} is organised in a
 much clearer way. Current implementation appears to support most
 of important features of these chips and, in fact, these are
 very robust devices. As mentioned earlier, \Chip{MC1322x} is an
 \Chip{ARM7TDMI} microcontroller with fully integrated radio and
 the only component external to the chip itself is the antenna.
 Apart from this, the \emph{Freescale} device has an outstanding
 set of peripheral and rather large amount of memory\footnote{\emph{%
 An average size of compiled code for the Talker program was
 reported to be 79 kB.}} (128KB of flash, 96kB of RAM and 80kB
 of ROM) the processor clock frequency is 24MHz. Comparing to
 the minimum amount of memory required to run \Contiki\ (2kB of RAM
 and 40kB of ROM) there is a very large headroom available for the
 application and some additional drivers. This also makes it a great
 development platform, during debugging stages compiler optimisation
 often needs to be disabled and additional memory may be also
 populated with debug data.

\subsection{MC1322x Architecture Overview}

 %Below is the diagram of \MCX\ \emph{Platform in Package (PiP)}.

\begin{figure}
\centering
\includegraphics[angle=-90,scale=0.76]{../figures/images/freescale_mc1322x_bd.png}
\caption{\emph{The block diagram of Freescale MC1322x}} \label{fig:mc1322x}
\end{figure}

  In addition to standard peripherals (UART, 12-bit A/D converters,
 SPI and I2C) this device has \emph{Synchronous Serial Interface (SSI)}
 that would allow communication with I2S devices as well as other
 synchronous serial peripherals. Being a 32-bit microcontroller
 it can be used for some basic audio signal processing. For example,
 with \emph{Analog Devices} \Chip{ADMP441} omnidirectional microphone
 with 24-bit I2S digital output \cite{datasheet:adi:admp441} an
 acoustic measurement sensor node can be designed. According to the
 datasheet, \Chip{ADMP441} has a very linear frequency response in
 the band between 100Hz to 15kHz. Such wireless node would be
 suitable for field noise measurement at popular music festivals
 or construction sites and industrial areas.
 Another sensor that could be included on such board would be an
 accelerometer, according to \emph{Freescale Application
 Note AN3751} \cite{appnote:freescale:AN3751} by utilising standard
 DSP techniques on accelerometer signal data, various vibration
 frequency analysis results can be produced. Another feature of
 the \MCX\ is it's accelerated MAC unit, called \emph{MACA}.

\subsection{MC1322x Development Hardware}

  The homepage of \Chip{MC1322x} port of \Contiki\ has a detailed
 overview of what development boards are available
 \cite{links:contiki:rcb:mc1322x}.
 Two boards which were chosen for this project are \emph{Freescale
 1322xUSB} and \emph{Redbee Econotag}.

  A minimum of two devices is required to implement a \WPAN\ network.
 At the early stage, the \emph{SparkFun} boards were supposed to be
 used as the sensor node and the gateway. When the decision had been
 made to use \MCX\ platform, the \emph{Freescale} dongle was found to
 be the most appropriate device to run the \Contiki\ \emph{Border Router}
 program, which implements end-to-end IP communication\footnote{\emph{The
 source code is located in: \Blob{examples/ipv6/rpl-border-router/}}}.

\subsubsection{Freescale USB Dongle}

 \emph{1322xUSB} is a very small USB device that doesn't provide the
 access to any of the pin and therefore is well suited as a network
 gateway. The circuit diagram is shown in figure \ref{fig:freescale-usb}
 to demonstrate the minimalistic layout of this board. One the bottom
 side of the board, there is a header for a JTAG connector and a few
 test points. It has appeared that the board has already been flashed
 in the factory, and the firmware had to be erased when the board was
 used for the first time to run the \Contiki\ code. The information on
 how to erase the firmware was obtained from the datasheet
 \cite{links:freescale:1322xusb}. This device was found rather primitive,
 its only benefit is the size, though it also serves as the minimal reference design.

\begin{figure}
\centering
\includegraphics[scale=0.5]{../figures/images/freescale_1322xusb.png}
\caption{\emph{The circuit layout of Freescale 1322xUSB}} \label{fig:freescale-usb}
\end{figure}

\subsubsection{Redbee Econotag}

 \emph{Econotag} by \emph{Redbee LLC} is highly regarded open-source
 development board. It is specifically design to meet all important
 requirenements of a development board with very constrained bill of
 materials. The board provides access to all of the I/O pins and
 and is very convenient for programming and debugging. Both of these
 functions are achived with the help of dual-port USB UART chip,
 \emph{FTDI} \Chip{FT22232HL} \cite{datasheet:ftdi:dual}. One of the
 two ports is connected to the primary UART interface and the second
 port provides access to the JTAG. The \Chip{FT2232HL} IC also facilitates
 bit-bang software reset, the \emph{Freescale} dongle cannot be reset
 in this way.

\subsection{Hardware Programming Facilities}

  To program the \Chip{MC1322x}, the first UART port is used. It also
 used for standard serial port output. The program is localed into
 RAM and doesn't wear-off the flash memory each time. Thought, for
 flashing the \MCX\ devices, a two-stage procedure is needed. First,
 a self-flasher code has to be loaded, which in turns loads the
 main program code into the flash memory.
 
  \Contiki\ source code contains the self-flasher implementations as
 well a set of utilities for programming the \MCX\ devices\footnote{%
 \emph{These are located in \Blob{cpu/mc1322x/tools/}.}} However, this
 is not the only way to update the software. Dunkels describes the design
 of \Contiki \emph{Executable and Linkable Format} \cite{dunkels06runtime}
 loader.  There are a few implication of using ELF loader, e.g. there
 has to be a certain amount of spare memory for the code to be downloaded
 into. However, it gives a great advantage of reprogramming over-the-air,
 without a need for physical access to the node.


\begin{figure}
\centering
\includegraphics[angle=90]{../figures/econotag_circuit_sheet1_monochrome.pdf}
\caption{\emph{Econotag schematics (sheet 1)}}\label{fig:econotag:circuit:s1}
\end{figure}

\begin{figure}
\centering
\includegraphics[angle=90]{../figures/econotag_circuit_sheet2_monochrome.pdf}
\caption{\emph{Econotag schematics (sheet 2)}}\label{fig:econotag:circuit:s2}
\end{figure}

\pagebreak
\subsection{Econotag Circuit}


  The schematics show in figures \ref{fig:econotag:circuit:s1},
  \ref{fig:econotag:circuit:s2} and
  the board layout in \ref{fig:econotag:layout} had been taken
  from the design package \cite{links:econotag:design} that is
  distribute by \emph{RedWire} under \emph{Creative Commons}
  license. These are included here to approve the simplicity
  of circuit layout with highly integrated \MCX\ chip.

   Comparing to the \emph{SparkFun} board, the \emph{Econotag}
  demonstrates much superior development solution with the
  inclusion of dual USB chip, that provides software reset and
  JTAG facilitates. Also many other minor aspects of this board
  provide a greater benefit then the earlier board. Althought,
  the layout shown above does not include the ground plane, by
  by visual comparison, the \emph{Econotag} has a very solid
  ground plane, unlike the \emph{SparkFun} board.

   The area where more experiments are still to be done, is the
  power consumption measurement. It is know however, that the
  \emph{Econtag} will run of two cell batteries for up to 48
  hours without utilising any of the sleep modes, however this
  information still has to be proven.


\begin{figure}
\centering
\includegraphics[angle=90,scale=1.75]{../figures/econotag_board_top.pdf}
\includegraphics[angle=90,scale=1.75]{../figures/econotag_board_bot.pdf}
\caption{\emph{Econotag development board layout (top and bottom)}} \label{fig:econotag:layout}
\end{figure}

\pagebreak
\subsection{Photographs of the Hardware in Use}

  The photograph in figure \ref{fig:photo:1} shows the top view
 of the \emph{Econotag} with an \Chip{ADXL345}\footnote{\emph{%
 More details on the use of the sensor are to follow.}}.
 break-out board soldered to the SPI pins. The second image
 (\ref{fig:photo:2}) is the bottom view of the Econotag with
 MIDI cable soldered to the UART2 pins, the DIN5 MIDI connector
 and the \emph{Freescale} dongle. It can observed the size of
 the dongle is very close to the size of DIN5 connector,
 therefore a device can be designed which would be of a size
 little bigger then the DIN5 connector body, though the power
 would need to be supplied. The standard midi connector
 has 5 pins and only two are used by most of the devices,
 hence, theoretically, this could be implemented. However,
 in practice, a battery will be required, since the modification
 of MIDI device to provide a power supply is not an option.
 The lack of bus power is another reason to consider MIDI
 being a legacy protocol.

\begin{figure}
\centering
\includegraphics[scale=0.8]{../figures/images/photos/1c.png}
\caption{\emph{Econotag (top view)}}\label{fig:photo:1}
\end{figure}

\begin{figure}
\centering
\includegraphics[scale=0.8,angle=90]{../figures/images/photos/2c.png}
\caption{\emph{Freescale USB dongle and the Econotag (bottom view)}}\label{fig:photo:2}
\end{figure}

\chapter{DSP Host System}
\section{DSP Host Hardware}

  A set of hardware platforms suitable for DSP host were considered.
 Among those are the following:
 	\begin{itemize}
		\item \emph{Analog Devices SHARC}
		\item \emph{ARM:} \begin{itemize}
		\item \emph{Marvell Sheeva}
		\item \emph{Texas Instruments OMAP}
		\item \emph{Freescale i.MX}
		\end{itemize}
	\end{itemize}
 All of these processor architectures are fully supported and widely
 used in embedded systems. Various other architecture families were
 considered, but found rather inappropriate due to the price range
 dictated by the target of this project. Multi-core \emph{MISP64}
 devices by \emph{NetLogic} \cite{netlogic:mips64:multicore}, for
 example, have great computational capacity, thought these are too
 expensive for this particular application.

\subsection{x86-based Embedded Devices}

  The list above does not include \emph{x86}-based CPUs due to the
 fact that it had been very difficult to find a target board using
 Intel or any \emph{x86} CPU from other vendors. There are many boards
 from a large variety of manufacturers, hence the selection process
 becomes particularly time consuming. Quality evaluation would also
 be necessary\footnote{\emph{Due to the scale of production in this
 market, there is a great chance to obtain a device with a failure.}},
 when looking at \emph{x86} devices. Although, there are boards that
 would match the low-power and small form-factor requirements, most
 of these are with a handful of peripherals, for example low quality
 audio and video interfaces and, if these are not present, the board
 may have two or more RS-232 connectors. Another aspect of \emph{x86}
 system design is that it comes with legacy \emph{BIOS} technology,
 while most of \emph{ARM} systems, for example, have rather more
 flexible facilities for booloader and set-up. Also in the class of
 \emph{single board computers (SBC)} there are many boards which do
 not inlude physical connectors for above mentioned peripherals, such
 boards require a specialised chasis. 

  Nevertheless, an \emph{x86} machine has been used for most of the
 development work on this project, that is for a rather transparent
 reason.

\subsection{OMAP}

  The \emph{OMAP} application processors form \emph{TI} has become
 popular in the open-source community, and in fact \emph{TI} promotes
 Linux and Android as most suitable operating systems for this platform.
 The \emph{OMAP} architecture is based around an \emph{ARM} processor
 (there are various models with different versions of \emph{ARM} core)
 and \emph{TI C64x} DSP block. However, there is a little problem
 associated with how the DSP unit is integrated with the CPU.
 Very limited documentation is provided on how it can be used in the
 \emph{Linux} environment \cite{ti:omap:wiki:dsp}.

 One most outstanding development platform that uses an \emph{OMAP}
 chips with dual-core \emph{Cortex-A8} CPU clocked at 1GHz, is the
 \emph{PandaBoard} \cite{ti:omap:wiki:pb}, however it is available
 for back-order only, the maker is producing these boards on demand
 and the lead time would be at least one month. Otherwise this board
 would have been purchased despite the fact that DSP unit would be
 difficult to utilised.

 %http://www.omappedia.org/wiki/DSPBridge_Project
 %http://www.omappedia.org/wiki/PandaBoard

\subsection{Other ARM CPUs}

  \emph{ARM} CPUs are produced by almost any major semiconductor firm,
 so do \emph{Freescale} and \emph{Marvell}. Not all of those companies
 make very high performance \emph{ARM} chips, i.e. clocked near to 1GHz,
 and few make multi-core CPUs. And only some CPUs have an FPU.
 \emph{ARM} floating point instruction set is know as \emph{VFP}
 ("Vector Floating Point"). There are different versions of \emph{VFP},
 though in this context this wasn't a concern.

\subsubsection{Marvell Sheeva}

  \emph{Marvell} has originally started offering \emph{ARM} processors
 since their purchase of Intel's \emph{PXA} devision. Now \emph{Marvell}
 produces a series of high-performance \emph{ARM} chips, most of which
 are multi-core. \emph{Sheeva} is the brand name for these SoC devices.
 The clock frequency ranges from 800 MHz to 1.6 GHz, gigabit Ethernet
 is one of the outstanding features, since \emph{Marvell} specialises
 in the networking and storage IC market. As mentioned above, most
 important feature of \emph{ARM} chips that is necessary for the design
 of DSP host for this project is \emph{VFP}. Some of \emph{ARMADA}
 SoCs include \emph{VFP}, namely \emph{ARMADA XP} series,
 \emph{ARMADA 510} and \emph{610} \cite{links:marvell:armada}.
 \emph{Kirkwood} series are not featuring \emph{VFP}, thought there
 are some outstanding embedded development platforms available.

  \emph{Plug Commputer} (also know as \emph{Seeva Plug}) is small, yet
 very powerful computer in a form-factor of wall socket DC adaptor
 \cite{links:marvell:plug,links:plugcomp:homepage}. There many exciting
 application where these devices could be the best fit, thought due to
 above mentioned lack of \emph{VFP}, this CPU is not well suited as the
 DSP host.

% Marvell Semiconductor, Inc. - ARMAD Processor Family
% http://www.marvell.com/products/processors/armada.html
% http://www.marvell.com/platforms/plug_computer/
% http://www.plugcomputer.org/

\subsubsection{Freescale i.MX}
% Freescale i.MX family info:
% http://cache.freescale.com/files/32bit/doc/brochure/FLYRIMXPRDCMPR.pdf
%^% NO DETAILS ABOUT NEON/VFP specified !!

% i.MX31 - mid range (532 MHz ARM1136JF-S)
% i.MX31 has VFP:
% http://www.freescale.com/webapp/sps/site/prod_summary.jsp?code=i.MX31
% http://www.freescale.com/files/32bit/doc/ref_manual/MCIMX31RM.pdf
%^% NOT MUCH APART FROM THE VFP.

% i.MX53 - top end of the family (1 GHz ARM Cortex-A8)
% i.MX535 and i.MX538 Applications Processors Fact Sheet 
% http://www.freescale.com/webapp/sps/site/overview.jsp?code=IMX53_FAMILY
% http://cache.freescale.com/files/32bit/doc/fact_sheet/IMX5CNFS.pdf
%^% IT SAYS THAT A BASIC DEV BOARD IS JUST $150, WITH TOUCHSCREEN - $200
% i.MX538 has NEON and VFP + SATA:
% http://www.freescale.com/webapp/sps/site/prod_summary.jsp?code=i.MX538
% http://www.freescale.com/files/32bit/doc/ref_manual/iMX53RM.pdf

\subsection{SHARC}

  The DSP instruction can be used directly and a library is provided
 which \dots blah \dots blah.

%% add to bibtex file:
% http://www.linuxfordevices.com/c/a/Linux-For-Devices-Articles/Single-Board-Computer-SBC-Quick-Reference-Guide/

\section{Building Embedded Linux OS}

\subsection{Background}

  Until quite recently, it had been rather more difficult to achieve
 the task of building (custom) embedded Linux system. Traditionally,
 the engineer who desired to do so, would need to follow instructions
 provided in the reference book \emph{"Linux from Scratch"}, commonly
 known as \emph{LFS} \cite{book:lfs}. This text provides details on
 how to utilise various tools for building an embedded Linux kernel
 and the file system from source, it discusses how to tweak various
 features at build time and configure appropriate runtime services.

\subsection{Build Utilities}
 
  More recently, a number of projects emerged, which do a great job
 of extending the flexibility by automating some of the simpler
 tasks, e.g. package dependency tracking and build version control
 (enabling the maintainer to revert to previous builds).
 One of the most outstanding and widely used projects in this areas
 is \emph{OpenEmbedded} \cite{links:oe} it has recently been adopted
 by a major software company \emph{Mentor Graphics}. \emph{Mentor
 Embedded Linux} extends \emph{OpenEmbedded} framework with a set
 of tools which are useful in a large-scale development project
 \cite{links:mentor:linux}.
 
  There are a few alternative approaches which lead to similar results,
 one may wish to use \emph{Gentoo Linux} meta-distribution, there
 is only one source of documentation - \emph{"Gentoo Linux Embedded
 Handbook"} \cite{links:gentoo:embedded}. Generally, \emph{Gentoo}
 framework (named {\emph{Portage}) has all necessary components
 which a system designer may need and there is no big difference
 between \emph{Gentoo} and \emph{OpenEmbedded}. Many internals
 are known to be very similar, tough \emph{Gentoo} doesn't provide
 some specialised tools which are provided in \emph{OpenEmbedded},
 hence \emph{Gentoo} has been originally designed for custom
 desktop and server systems. Major aspect which is different is
 that \emph{Gentoo} is rather bound to package releases, while
 \emph{OpenEmbedded} is more flexible when a development revision
 number has to be specified for a certain source code package.
 The conclusion is that \emph{OpenEmbedded} is most likely to be
 more convenient to use for an embedded target.
 
  A little different approach can be taken with \emph{BuildRoot}
 \cite{links:buildroot:homepage}, which is a build system for
 embedded Linux. Based around the same framework that was design
 for configuring the Linux kernel builds. This tool is a set of
 scripts controlled via a console menu. Although, it appears to
 have a simple to use front-end, the underlying configuration
 system is certainly behind the competition with \emph{OpenEmbedded}.

  A project which have a slightly different orientation is
 \emph{ScratchBox} \cite{links:sbox:homepage}. It provides
 a cross-compilation toolkit for application developers.
 It can be used to provide an software development kit (SDK)
 for a 3rd-party developer with appropriate toolchain and
 hardware emulator. However, \emph{OpenEmbedded} includes
 support for building SDK and \emph{ScratchBox} is rather
 limiting in various ways, i.e. it is a static distribution,
 rather then a build system.
 
  Another project in this area of interest is \emph{Linaro Foundation}
 \cite{links:linaro:homepage}, the initiative has been originated by
 London-based \emph{Canonical Limited} (the company behind now most
 popular Linux distribution - \emph{Ubuntu}).
 The purpose of \emph{Linaro} is to improve various issues related to
 embedded Linux and provide a higher grade platform for Android and
 other multimedia and consumer oriented distributions. \emph{Linaro} is
 still an emerging project and has not produced any significant output
 \cite{links:linaro:homepage}. It has specific orientation, in terms of
 hardware, being currently involved only with \emph{ARM} platforms,
 and in terms software, it is bound rather strictly to \emph{Ubuntu}
 methodology\footnote{\emph{That implies it is being directed by the
 founder company, Canonical.}}.

\section{Conclusion}

  It had been desired to build an embedded DSP host OS as one of
 additional project targets, however due to various\footnote{\emph{%
 Another aspect was due to hardware selection and purchasing issues.}}
 limitations, most importantly - the time frame, the project has not
 achieved this at the time of writing of the final report. The above
 chapter was included to provide the evidence of research in this area.

\section{Synthesis Software}

  Another target which has not been achieved, since it was dependant
 on the design of the DSP host, was to use visual DSP modeling package
 \emph{Pure Data} \cite{links:wiki:pd, links:sw:pd}. It would be a
 simple task and there is no particular concern of how to integrate
 the current implementation with \emph{Pure Data}\footnote{\emph{%
 The author has extensive experience of using this package}}. Another
 reason for disregarding this target, was that the emphasis of the
 entire project has changed towards various aspects of wireless
 sensor network implementation and audio synthesis is outside the
 general scope of this report.



\chapter{Application Development}
\section{Working with Contiki}
  
  After an appropriate compiler toolchain and debugger packages for \MCX\
 had been installed on the development host, a few steps were taken to
 simplify the work-flow in \Contiki\ application development environment.
 It may be noted here, that an integrated development environment (IDE)
 could be used and some programmers do prefer to use an IDE, such as
 \emph{Eclipse} \cite{links:contiki:eclipse, links:mc1322x:eclipse},
 nevertheless the command line tools are known to be the most efficient approach.
 It should be noted that this section is rather brief description of what
 has been done and was not intended to provide a detailed guidance on how
 to reproduce the results.

  Apart from the revision control tools\footnote{\emph{This project used
 git system, however the details on how that has been done are considered
 to be irrelevant to the subject of this report.}}, a text editor and the
 GCC toolchain for ARM \cite{links:mc1322x:gcc}, there are three essential
 command-line tools which were utilised during the development process:

\begin{description}
	\item [\MAN{gdb}] - GNU Debugger (source-level) \cite{docs:gdb:manual}
	\item [\MANX{1}{make}] - GNU make program \cite{docs:make:manual}
	\item [\emph{\texttt{OpenOCD}}] - On-chip Debugger \cite{links:mc1322x:ocd}
\end{description}


  To enhance the work-flow \emph{"Makefiles"}\footnote{\emph{These are
 the file which specify a set of rules for the make program on how to
 compile the source code and also perform administrative tasks and run
 debugger or other tools.}} were amended throughout the development
 process. Generally there is one \emph{Makefile} in each subdirectory
 of the source tree, thought most of these inherit rules specified in
 the main \emph{Makefile} (in \Contiki\ there are two of these - one
 in the root directory and one for each processor architecture).

 Most of the changes were made in \Blame{cpu/mc1322x/Makefile.mc1322x}
 to provide a few of shorter commands for setting-up the debugger
 and sending the program to run on a development board\footnote{
 \emph{For example, to set-up the WPAN router on the Freescale board - run
 \texttt{`make.f1 router'} and to compile 'example.c', load and print
 serial output on the console for the Econotag  -
 \texttt{`make.e1 example.load-print'}; in case if 'example.c' does
 not behave as expected - run \texttt{`make.e1 example.ocd-screen'}.}}
 
\section{Application Prerequisites}

  The first step in application development was to add a driver for
 the second UART (UART2) port. This has been done by copying an
 existing driver code for UART1, though enhancements were required
 at a later stage. The history of changes to the code can be viewed
 in at the repository by utilising the commit log filter. The files
 shown here had been modified.

 \begin{itemize}
 \item \Contrib{cpu/mc1322x/lib/include/uart1.h} \\
 	\emph{normal and weak prototypes, register pointers and macros}
 \item \Contrib{cpu/mc1322x/lib/uart2.c} \\
 	\emph{driver interrupt handlers}
 \item \Contrib{cpu/mc1322x/src/default\_lowlevel.h} \\
	\emph{prototypes}
 \item \Contrib{cpu/mc1322x/src/default\_lowlevel.c} \\
	\emph{initialisation functions}
 \end{itemize}

  A set of macros was defined to aid the application code, though
 most of those macros were used only for initial debugging.

  Soon, it has been understood that the original implementation
 was missing handler functions for the UART interrupts. With
 the help of communication on the \Contiki\ mailing list, the
 appropriate methods were realised. One of the key techniques
 was to use \emph{"weak"} function linking attribute. Two macros
 named \texttt{U2\_RXI\_POLLHANDLER()} and \texttt{U2\_TXI\_CALL()}
 are appearing in the program listing \ref{code:talker}. These
 macros define the function which is called by the interrupt
 routine in \Blob{cpu/mc1322x/lib/uart2.c}\footnote{\emph{For
 the definition of the macros, see: \begin{itemize}
	\item \Blob{cpu/mc1322x/lib/include/uart1.h}
	\item \Blob{projects/wmi/mc1322x/midi/uart2-midi.h}
 	\end{itemize}}}.

\TrackerList\em
\IssueX{27} Compiling various example programs
\IssueX{31} Wireless transmission of MIDI is proven working
\IssueX{32} Requirenments for the MIDI UART driver
\TrackerEnd

\section{Designing the MIDI Talker Program}

  A variety of constructs had been tried for handling the
 FIFO  buffer into the \emph{uIP} packet buffer. This was
 rather challenging and many complicated bugs appeared
 throughout out the process. The JTAG debugger had been
 utilised a number of times to find these bugs.

  The macro named \texttt{PSOCK\_GENERATOR\_SEND()} had
 been found as the most direct way to transfer the data
 from FIFO into the packet buffer. This macro is defined
 and documented in \Blob{core/net/psock.h}, it is a part
 of \Contiki\ \emph{protosockets} library.

\subsection{Protosockets}

  This section gives more detail on \emph{protosockets},
 which were introduced in \ref{sec:CONTIKI}.

  The \emph{protosockets}  is a set of C macros (and just
 a few C functions) defined and documented in:
 \begin{itemize}	
	\item\Blob{core/net/psock.h}
	\item\Blob{core/net/psock.c}
 \end{itemize}

 These are indeed simple methods for a TCP application to
 utilise. Nevertheless, it is commonly known that UDP
 communication is most appropriate for a type of program
 that is discussed here, the results included included in
 this chapter will show that TCP can be used for a simple
 program like this. If UDP was to be utilised, additional
 algorithms for retransmission and stream data ordering
 would need to be developed. As discussed in section
 \ref{sec:MIDI}, it should be rather more appropriate to
 develop a library for \emph{RTP-MIDI} instead of a simple
 UDP application. There is a great benefit to use TCP at
 this stage, as it provides the guarantees of data being
 transmitted correctly and in a sequential order on the
 socket level.

 For detailed documentation of the function which will
 appear in the program listing, see the \emph{Contiki}
 \cite{contiki:docs} and the \emph{uIP} \cite{uip:docs}
 manuals. It would be misleading to included extracts
 of documentation here, since lower-level functions
 and algorithms would need to be listed.

\section{The Source Code}

  The Talker program is considered to be complete
 and most of the possible bugs had been eliminated
 from this code, therefore it shall be presented
 here. Some lines had been omitted for clarity and
 complete version can be viewed in the repository
 (\Blob{projects/wmi/mc1322x/midi/psock-taker.c}).


  The Listener program, however, was not finished
 at the time of submission of this report. The
 current version is missing an algorithm which
 would process the packet buffer before it had
 been filled-up. This is considered a major design
 problem, which should be solved within a short
 period of time after the submission of this report
 (see \Blob{projects/wmi/mc1322x/midi/psock-listener.c}
 for the current implementation).


\subsection{Further Enhancements}

  The flow diagram of the Talker program is show in
 figure \ref{sm:talker} and the Listener flow is
 shown in \ref{sm:listener}. There is problem with
 this approach - the Listener has to know the address
 of the Talker, which doesn't make this network
 very flexible. It would be rather appropriate to
 implement a self-advertising device. That can be
 achieved by adding message broadcast thread, as shown
 in figures \ref{sm:advtalker} and \ref{sm:advlistener}.
 These diagrams also include an additional sleep state,
 which may be considered an optional element. The
 sleep delay calculation would need to be defined in
 such way that there is possibility of nodes not detecting
 each other. It should be also noted that these flow
 charts demonstrate the behaviour of wireless nodes,
 thought a similar flow can be used for the host program.


\pagebreak
\lstinputlisting[caption={\emph{The Talker program}},
label=code:talker,language=C]
{sw/code_talker.c}


\section{Implementation Decisions}

  Two programs where written to test transmission of MIDI. The concept of
 server/client communication could be applied, nevertheless it had been
 found more appropriate to name the parties as a \emph{Talker} and a
 \emph{Listener}. At the time of writing of this report the \emph{Talker}
 program had been fully tested and debugged. The second program can be
 considered only as a prototype and does require further work.

  The \emph{protosocket} macros had been utilised for the implementation,
 hence the communication protocol imposed by \emph{protosocket} was TCP.
 There are various improvements that are still to be considered.
 The specification of TCP includes a notion for \emph{urgent data}
 (\RFC{0793}), however this feature has received a very limited adoption
 and is known to be handled differently by various implementations of TCP
 \cite{ietf:draft:tcpm:urgdata}. It has been found that the \emph{uIP}
 stack does implement some mechanisms for the \emph{urgent data}, however
 the documentation on how it can be applied in \Contiki\ application code,
 could not be found. 


\subsection{UDP Multicasting}

  A variety of packages exist which utilise UDP multicasting for streaming
 of the MIDI data on LAN. These include \emph{ipMIDI} \cite{links:ipmidi} for
 Microsoft Windows and Apple Mac OS X platforms as well as compatible
 packages for Linux - \emph{multimidicast} \cite{links:multimidicast} and
 \emph{qmidinet} \cite{links:qmidinet}. This two Linux packages had been
 tested with current implementation of the \emph{Talker} program and have
 shown a suitable performance. The multicasting approach is considered to
 be quite appropriate for streaming of MIDI signals on IP networks, since
 no configuration is required, i.e. none of the participants have to set-up
 connection between each other and only need to send data to the multicast
 IP address\footnote{\emph{For compatibility with ipMIDI, the multicast IP
 address should be 225.0.0.37 and the port numbers from 21928 and above.}}.
 However the \emph{ipMIDI} is not a formal protocol, it simply streams the
 MIDI signals to a UDP socket, i.e. there are no any error correcting
 mechanisms or hand-shaking.


\subsection{MIDI Networking in Linux}

  On Linux, networking of MIDI clients can be implemented in user-space
 with \emph{ALSA ("Advanced Linux Sound Architecture")} utilities package
 \cite{links:linux:alsa}.
 The documentation on \emph{ALSA Sequencer Network} is limited to the
 comments in the source code of the \MAN{aseqnet} program. It is design
 using UDP and a specific data structure. The two programs mentioned in
 the previous section do not communicate with \MAN{aseqnet}, but use
 the \emph{ALSA Sequencer} interface directly.


\subsection{MIDI payload for Real-time Transport Protocol}

 Two relevant Internet standards had been published by the IETF in 2006:
 \begin{itemize}
  \item\emph{"RTP Payload Format for MIDI"} (\RFC{4695})
  \item\emph{An Implementation Guide for RTP MIDI} (\RFC{4696})
 \end{itemize}

 This two papers had been studied and considered as a possible solution
 for encapsulating MIDI data stream on a \emph{6loWPAN} networks, however
 the implementation would require a library for the \emph{Real-time
 Transport Protocol} as well a large set of algorithms to handle various
 aspects of the \emph{RFC4695}. It would be a very challenging task to
 design \emph{RTP-MIDI} library which could run on a microcontroller.

 This protocol is currently implemented as part of \emph{Apple CoreMidi}
 \cite{links:wiki:rtpmidi} and an open-source library exists for Linux
 \cite{links:rtpmidi}. However, there had been major adoption for the
 \emph{RTP-MIDI} in the field.



\chapter{State Machines}
\begin{tikzpicture}[>=latex, shorten >=1pt, node distance=1in,
			on grid, auto, initial text=\fbox{\tt{Start}}]


\node [E]
(run)
	{\texttt{Listener Initialised}};

\node [C, below of =run, minimum width=10em]
(adv)
	{\emph{Wait for Talker Broadcast}};

\node [E, left of =adv, node distance =6cm]
(wup)
	{\texttt{Wake-up}};

\node [F, below of =adv, text width=10em]
(rep)
	{Attempt to Reply\\to the Advertisement};

\node [C, below of =rep]
(req)
	{\emph{Request Fulfilled?}};

\node [F, below of =req, text width=8em]
(con)
	{Establish Connection};

\node [C, below of =con]
(buf)
	{\emph{Wait for Data Packets}};

\node [F, below of =buf, text width=12em]
(use)
	{\emph{\underline{Submit Data to}\\\underline{the Processing Thread}}};

\node [E, right of =adv, node distance =6cm]
(int)
	{\texttt{Connection Failure}};

\node [F, left of =rep, node distance =3.5cm, text width=7em]
(del)
	{\emph{Decreasing\\Exponential\\Sleep Delay}};

\coordinate [right of =use] (ret);

\path[L, dashed] (run) -- (adv);

\path[L, dashed] (int) -- (adv);

\path[L, dashed] (wup) -- (del);

\path[L] (adv) -- node {Yes, Talker is Present} (rep);

\path[L] (rep) -- (req);

\path[L] (req) -- node [near start] {No, Request Failed} (del);

\path[L] (del) |- (adv);

\path[L] (req) -- node {Yes, Ready to Connect} (con);

\path[L] (con) -- (buf);

\path[L] (buf) -- node {Received} (use);

\path[L] (use) |- (ret) |- node [near end] {Return} (buf);

\end{tikzpicture}

\begin{tikzpicture}[>=latex, shorten >=1pt, node distance=1in,
			on grid, auto, initial text=\fbox{\tt{Start}}]


\node [E]
(run)
	{\texttt{Talker Initialised}};

\node [E, left of =run, node distance =6cm]
(wup)
	{\texttt{Wake-up}};

\node [F, below of =run, text width=8em]
(adv)	
	{Broadcast Advertisement Messages};

\node [C, below of =adv, text width=6em]
(rep)
	{\emph{Wait for a Connection}};

\node [F, below of =rep, text width=12em]
(con)
	{Connect to the Listener};

\node [C, below of =con]
(buf)
	{\emph{Wait for the Data Buffer}};

\node [F, below of =buf, text width=10em]
(now)
	{\emph{\underline{Poll the Stack Thread}\\\underline{to Send Data Now}}};

\node [E, right of =adv, node distance =6cm]
(int)
	{\texttt{Connection Failure}};

\node [F, left of =adv, node distance =3.5cm, text width=7em]
(del)
	{\emph{Increasing\\Exponential\\Sleep Delay}};
\coordinate [right of =now] (ret);

\path[L, dashed] (run) -- (adv);

\path[L, dashed] (int) -- (adv);

\path[L, dashed] (wup) -- (del);

\path[L] (adv) -- (rep);

\path[L] (rep) -- node {Yes, Request Received} (con);

\path[L] (rep) -- node [near start] {No, Listner is not Present} (del);

\path[L] (del) |- (adv);

\path[L] (con) -- (buf);

\path[L] (buf) -- node {Ready} (now);

\path[L] (now) |- (ret) |- node [near end] {Refill} (buf);

\end{tikzpicture}


\chapter{System View}
\section{Key System Components: \\ Specification \& Requirements} \label{sec:SPECS}

  This section outlines the specification which had been proposed
 after initial research phase has been accomplished. Diagram shown
 in figure \ref{fig:sketch1} illustrates the function blocks of the
 complete system that meets this specification, however it is now
 considered suitable only for the development system and not the
 final product design.

\subsection*{Sensor Node}

\emph{Sensor Node} is a device featuring a microcontroller chip
that contains following function blocks and peripherals:

\begin{itemize}
	\item \emph{internal:}
	\begin{itemize}
		\item 2.4Ghz RF transceiver,
		\item A/D converters,
		\item GPIO,
		\item SPI, and
		\item USART
	\end{itemize}
	
	\item \emph{external:}
	\begin{itemize}
		\item digital or analogue sensors,
		\item external connectors for MIDI,
		\item wired remote sensors, and
		\item serial port header
	\end{itemize}
\end{itemize}

The \emph{Sensor Node} runs software which consists of:

\begin{itemize}
	\item \emph{operating system:}
	\begin{itemize}
		\item communication protocol stack,
		\item devices drivers, and
		\item application task management,
		\item service tasks, and
		\item the main program
	\end{itemize}
	\item \emph{bootloader:}
	\begin{itemize}
		\item loads new software
	\end{itemize}
\end{itemize}

\begin{figure}
\centering
\includegraphics[scale=0.4]{../figures/sketch1.pdf}
\caption{\emph{Abstract Sketch of the Development System}} \label{fig:sketch1}
\end{figure}


\pagebreak
\section{Aspects of Final Product Design}

   Many commercial uses of \WPAN\ in the field of audio control were
 considered. The block diagrams in figure \ref{fig:products} briefly
 illustrate a few interesting solution. Some of the device proposed
 here can be used in a variety of applications and others are rather
 specific for the live performance instruments.

  Current implementation can be used to some extend, however one
 very important modification of hardware needs to be considered.
 The connection between \WPAN\ gateway is currenly using USB,
 however it would be more appropriate to design a board with
 direct SPI link between the host CPU and the transciever.

\begin{figure}
\centering
\includegraphics[scale=0.8]{../../poster/figures/figure1.pdf}
\caption{\emph{TEST}} \label{fig:products}
\end{figure}


\chapter{Looking Ahead}

  The major emphasis of this project had been in the area of development
 tools, platforms and code architecture for embedded software design.
 Though, originally, it was intended as an application development project,
 study of various alternatives of how the simple task can be achieved, lead
 to extended research in embedded operating systems and various programming
 tools, including: compiler toolchains, cross-platform integration techniques
 as well as the languages. Some of the most relevant aspects had been covered 
 in this report, leaving the unexplored depth behind.

  One of initial interest areas had been in the field of networking protocols
 for wireless sensing and control. Current best practice suggests that IP-enabled
 networking technology is most suitable for the global adoption, hence simple
 to integrate with various existing applications. It is rather difficult to
 imagine where, in the modern technology world, a non-IP network may be useful.
 Certainly, non-IP networking technology is available on the market, however if
 such devices are going to be used in any application, there are several limitations
 imposed on integration into various existing infrastructures. That implies
 physical connectivity as well as software support. One example is the popular
 \emph{ZigBee} family of protocols, which is gaining some popularity, thought
 no global scale of deployment has been observed. Not too long ago, there used
 to exist quite a few non-IP networking protocols, such as \emph{AppleTalk}
 and \emph{IPX}, for example. The majority of the published papers are oriented
 towards the use of IP-enabled networks for wireless sensing, just a few praise
 proprietary technologies.

  The project, which had been described here, has achieved its most
 fundamental goals, i.e. it has been proven that the MIDI data is
 handled as expected on a \emph{P802.15.4/6loWPAN} network and the
 platform had been established for future experimentation.
 The bigger project is now to commence.


%%%%%%%%%%%%%%%%%%%%%%%%%%%%%%%%%%%%%%%%%%%%%%%%%%%%%%%%%%%%%%%%%%%%%%%%

\bibliographystyle{acm}

\bibliography{../main,../../contiki/dunkels,../../timesync/papers,../../tinyos/tep}

\end{document}
